% This file is automatically generated by the "core dump appdocs" CLI command.  Any manual edits will be lost.
\section{AddQueueMember}
\subsection{Synopsis}
\begin{verbatim}
Dynamically adds queue members
\end{verbatim}
\subsection{Description}
\begin{verbatim}
   AddQueueMember(queuename[|interface[|penalty[|options[|membername]]]]):
Dynamically adds interface to an existing queue.
If the interface is already in the queue it will return an error.
  This application sets the following channel variable upon completion:
     AQMSTATUS    The status of the attempt to add a queue member as a 
                     text string, one of
           ADDED | MEMBERALREADY | NOSUCHQUEUE 
Example: AddQueueMember(techsupport|SIP/3000)

\end{verbatim}


\section{ADSIProg}
\subsection{Synopsis}
\begin{verbatim}
Load Asterisk ADSI Scripts into phone
\end{verbatim}
\subsection{Description}
\begin{verbatim}
  ADSIProg(script): This application programs an ADSI Phone with the given
script. If nothing is specified, the default script (asterisk.adsi) is used.

\end{verbatim}


\section{AgentLogin}
\subsection{Synopsis}
\begin{verbatim}
Call agent login
\end{verbatim}
\subsection{Description}
\begin{verbatim}
  AgentLogin([AgentNo][|options]):
Asks the agent to login to the system.  Always returns -1.  While
logged in, the agent can receive calls and will hear a 'beep'
when a new call comes in. The agent can dump the call by pressing
the star key.
The option string may contain zero or more of the following characters:
      's' -- silent login - do not announce the login ok segment after agent logged in/off

\end{verbatim}


\section{AgentMonitorOutgoing}
\subsection{Synopsis}
\begin{verbatim}
Record agent's outgoing call
\end{verbatim}
\subsection{Description}
\begin{verbatim}
  AgentMonitorOutgoing([options]):
Tries to figure out the id of the agent who is placing outgoing call based on
comparison of the callerid of the current interface and the global variable 
placed by the AgentCallbackLogin application. That's why it should be used only
with the AgentCallbackLogin app. Uses the monitoring functions in chan_agent 
instead of Monitor application. That have to be configured in the agents.conf file.

Return value:
Normally the app returns 0 unless the options are passed.

Options:
	'd' - make the app return -1 if there is an error condition	'c' - change the CDR so that the source of the call is 'Agent/agent_id'
	'n' - don't generate the warnings when there is no callerid or the
	      agentid is not known.
             It's handy if you want to have one context for agent and non-agent calls.

\end{verbatim}


\section{AGI}
\subsection{Synopsis}
\begin{verbatim}
Executes an AGI compliant application
\end{verbatim}
\subsection{Description}
\begin{verbatim}
  [E|Dead]AGI(command|args): Executes an Asterisk Gateway Interface compliant
program on a channel. AGI allows Asterisk to launch external programs
written in any language to control a telephony channel, play audio,
read DTMF digits, etc. by communicating with the AGI protocol on stdin
and stdout.
  This channel will stop dialplan execution on hangup inside of this
application, except when using DeadAGI.  Otherwise, dialplan execution
will continue normally.
  A locally executed AGI script will receive SIGHUP on hangup from the channel
except when using DeadAGI. This can be disabled by setting the AGISIGHUP channel
variable to "no" before executing the AGI application.
  Using 'EAGI' provides enhanced AGI, with incoming audio available out of band
on file descriptor 3

  Use the CLI command 'agi show' to list available agi commands
  This application sets the following channel variable upon completion:
     AGISTATUS      The status of the attempt to the run the AGI script
                    text string, one of SUCCESS | FAILED | NOTFOUND | HANGUP

\end{verbatim}


\section{AlarmReceiver}
\subsection{Synopsis}
\begin{verbatim}
Provide support for receiving alarm reports from a burglar or fire alarm panel
\end{verbatim}
\subsection{Description}
\begin{verbatim}
  AlarmReceiver(): Only 1 signalling format is supported at this time: Ademco
Contact ID. This application should be called whenever there is an alarm
panel calling in to dump its events. The application will handshake with the
alarm panel, and receive events, validate them, handshake them, and store them
until the panel hangs up. Once the panel hangs up, the application will run the
system command specified by the eventcmd setting in alarmreceiver.conf and pipe
the events to the standard input of the application. The configuration file also
contains settings for DTMF timing, and for the loudness of the acknowledgement
tones.

\end{verbatim}


\section{AMD}
\subsection{Synopsis}
\begin{verbatim}
Attempts to detect answering machines
\end{verbatim}
\subsection{Description}
\begin{verbatim}
  AMD([initialSilence][|greeting][|afterGreetingSilence][|totalAnalysisTime]
      [|minimumWordLength][|betweenWordsSilence][|maximumNumberOfWords]
      [|silenceThreshold])
  This application attempts to detect answering machines at the beginning
  of outbound calls.  Simply call this application after the call
  has been answered (outbound only, of course).
  When loaded, AMD reads amd.conf and uses the parameters specified as
  default values. Those default values get overwritten when calling AMD
  with parameters.
- 'initialSilence' is the maximum silence duration before the greeting. If
   exceeded then MACHINE.
- 'greeting' is the maximum length of a greeting. If exceeded then MACHINE.
- 'afterGreetingSilence' is the silence after detecting a greeting.
   If exceeded then HUMAN.
- 'totalAnalysisTime' is the maximum time allowed for the algorithm to decide
   on a HUMAN or MACHINE.
- 'minimumWordLength'is the minimum duration of Voice to considered as a word.
- 'betweenWordsSilence' is the minimum duration of silence after a word to 
   consider the audio that follows as a new word.
- 'maximumNumberOfWords'is the maximum number of words in the greeting. 
   If exceeded then MACHINE.
- 'silenceThreshold' is the silence threshold.
This application sets the following channel variable upon completion:
    AMDSTATUS - This is the status of the answering machine detection.
                Possible values are:
                MACHINE | HUMAN | NOTSURE | HANGUP
    AMDCAUSE - Indicates the cause that led to the conclusion.
               Possible values are:
               TOOLONG-<%d total_time>
               INITIALSILENCE-<%d silenceDuration>-<%d initialSilence>
               HUMAN-<%d silenceDuration>-<%d afterGreetingSilence>
               MAXWORDS-<%d wordsCount>-<%d maximumNumberOfWords>
               LONGGREETING-<%d voiceDuration>-<%d greeting>

\end{verbatim}


\section{Answer}
\subsection{Synopsis}
\begin{verbatim}
Answer a channel if ringing
\end{verbatim}
\subsection{Description}
\begin{verbatim}
  Answer([delay]): If the call has not been answered, this application will
answer it. Otherwise, it has no effect on the call. If a delay is specified,
Asterisk will wait this number of milliseconds before returning to
the dialplan after answering the call.

\end{verbatim}


\section{Authenticate}
\subsection{Synopsis}
\begin{verbatim}
Authenticate a user
\end{verbatim}
\subsection{Description}
\begin{verbatim}
  Authenticate(password[|options[|maxdigits]]): This application asks the caller
to enter a given password in order to continue dialplan execution. If the password
begins with the '/' character, it is interpreted as a file which contains a list of
valid passwords, listed 1 password per line in the file.
  When using a database key, the value associated with the key can be anything.
Users have three attempts to authenticate before the channel is hung up.
  Options:
     a - Set the channels' account code to the password that is entered
     d - Interpret the given path as database key, not a literal file
     m - Interpret the given path as a file which contains a list of account
         codes and password hashes delimited with ':', listed one per line in
         the file. When one of the passwords is matched, the channel will have
         its account code set to the corresponding account code in the file.
     r - Remove the database key upon successful entry (valid with 'd' only)
     maxdigits  - maximum acceptable number of digits. Stops reading after
         maxdigits have been entered (without requiring the user to
         press the '#' key).
         Defaults to 0 - no limit - wait for the user press the '#' key.

\end{verbatim}


\section{BackGround}
\subsection{Synopsis}
\begin{verbatim}
Play an audio file while waiting for digits of an extension to go to.
\end{verbatim}
\subsection{Description}
\begin{verbatim}
  Background(filename1[&filename2...][|options[|langoverride][|context]]):
This application will play the given list of files while waiting for an
extension to be dialed by the calling channel. To continue waiting for digits
after this application has finished playing files, the WaitExten application
should be used. The 'langoverride' option explicitly specifies which language
to attempt to use for the requested sound files. If a 'context' is specified,
this is the dialplan context that this application will use when exiting to a
dialed extension.  If one of the requested sound files does not exist, call processing will be
terminated.
  Options:
    s - Causes the playback of the message to be skipped
          if the channel is not in the 'up' state (i.e. it
          hasn't been answered yet). If this happens, the
          application will return immediately.
    n - Don't answer the channel before playing the files.
    m - Only break if a digit hit matches a one digit
          extension in the destination context.
This application sets the following channel variable upon completion:
 BACKGROUNDSTATUS    The status of the background attempt as a text string, one of
               SUCCESS | FAILED

\end{verbatim}


\section{BackgroundDetect}
\subsection{Synopsis}
\begin{verbatim}
Background a file with talk detect
\end{verbatim}
\subsection{Description}
\begin{verbatim}
  BackgroundDetect(filename[|sil[|min|[max]]]):  Plays  back  a  given
filename, waiting for interruption from a given digit (the digit must
start the beginning of a valid extension, or it will be ignored).
During the playback of the file, audio is monitored in the receive
direction, and if a period of non-silence which is greater than 'min' ms
yet less than 'max' ms is followed by silence for at least 'sil' ms then
the audio playback is aborted and processing jumps to the 'talk' extension
if available.  If unspecified, sil, min, and max default to 1000, 100, and
infinity respectively.

\end{verbatim}


\section{Bridge}
\subsection{Synopsis}
\begin{verbatim}
Bridge two channels
\end{verbatim}
\subsection{Description}
\begin{verbatim}
Usage: Bridge(channel[|options])
	Allows the ability to bridge two channels via the dialplan.
The current channel is bridged to the specified 'channel'.
The following options are supported:
   p - Play a courtesy tone to 'channel'.
BRIDGERESULT dial plan variable will contain SUCCESS, FAILURE, LOOP, NONEXISTENT or INCOMPATIBLE.

\end{verbatim}


\section{Busy}
\subsection{Synopsis}
\begin{verbatim}
Indicate the Busy condition
\end{verbatim}
\subsection{Description}
\begin{verbatim}
  Busy([timeout]): This application will indicate the busy condition to
the calling channel. If the optional timeout is specified, the calling channel
will be hung up after the specified number of seconds. Otherwise, this
application will wait until the calling channel hangs up.

\end{verbatim}


\section{ChangeMonitor}
\subsection{Synopsis}
\begin{verbatim}
Change monitoring filename of a channel
\end{verbatim}
\subsection{Description}
\begin{verbatim}
ChangeMonitor(filename_base)
Changes monitoring filename of a channel. Has no effect if the channel is not monitored
The argument is the new filename base to use for monitoring this channel.

\end{verbatim}


\section{ChanIsAvail}
\subsection{Synopsis}
\begin{verbatim}
Check channel availability
\end{verbatim}
\subsection{Description}
\begin{verbatim}
  ChanIsAvail(Technology/resource[&Technology2/resource2...][|options]): 
This application will check to see if any of the specified channels are
available. The following variables will be set by this application:
  ${AVAILCHAN}     - the name of the available channel, if one exists
  ${AVAILORIGCHAN} - the canonical channel name that was used to create the channel
  ${AVAILSTATUS}   - the status code for the available channel
  Options:
    s - Consider the channel unavailable if the channel is in use at all
    t - Simply checks if specified channels exist in the channel list
        (implies option s) 

\end{verbatim}


\section{ChannelRedirect}
\subsection{Synopsis}
\begin{verbatim}
Redirects given channel to a dialplan target.
\end{verbatim}
\subsection{Description}
\begin{verbatim}
ChannelRedirect(channel|[[context|]extension|]priority):
  Sends the specified channel to the specified extension priority

\end{verbatim}


\section{ChanSpy}
\subsection{Synopsis}
\begin{verbatim}
Listen to a channel, and optionally whisper into it
\end{verbatim}
\subsection{Description}
\begin{verbatim}
  ChanSpy([chanprefix][|options]): This application is used to listen to the
audio from an Asterisk channel. This includes the audio coming in and
out of the channel being spied on. If the 'chanprefix' parameter is specified,
only channels beginning with this string will be spied upon.
  While spying, the following actions may be performed:
    - Dialing # cycles the volume level.
    - Dialing * will stop spying and look for another channel to spy on.
    - Dialing a series of digits followed by # builds a channel name to append
      to 'chanprefix'. For example, executing ChanSpy(Agent) and then dialing
      the digits '1234#' while spying will begin spying on the channel
      'Agent/1234'.
  Note: The X option supersedes the three features above in that if a valid
        single digit extension exists in the correct context ChanSpy will
        exit to it. This also disables choosing a channel based on 'chanprefix'
        and a digit sequence.
  Options:
    b             - Only spy on channels involved in a bridged call.
    g(grp)        - Match only channels where their ${SPYGROUP} variable is set to
                    contain 'grp' in an optional : delimited list.
    q             - Don't play a beep when beginning to spy on a channel, or speak the
                    selected channel name.
    r[(basename)] - Record the session to the monitor spool directory. An
                    optional base for the filename may be specified. The
                    default is 'chanspy'.
    v([value])    - Adjust the initial volume in the range from -4 to 4. A
                    negative value refers to a quieter setting.
    w             - Enable 'whisper' mode, so the spying channel can talk to
                    the spied-on channel.
    W             - Enable 'private whisper' mode, so the spying channel can
                    talk to the spied-on channel but cannot listen to that
                    channel.
    o             - Only listen to audio coming from this channel.
    X             - Allow the user to exit ChanSpy to a valid single digit
                    numeric extension in the current context or the context
                    specified by the SPY_EXIT_CONTEXT channel variable. The
                    name of the last channel that was spied on will be stored
                    in the SPY_CHANNEL variable.

\end{verbatim}


\section{ClearHash}
\subsection{Synopsis}
\begin{verbatim}
Clear the keys from a specified hashname
\end{verbatim}
\subsection{Description}
\begin{verbatim}
ClearHash(<hashname>)
  Clears all keys out of the specified hashname

\end{verbatim}


\section{Congestion}
\subsection{Synopsis}
\begin{verbatim}
Indicate the Congestion condition
\end{verbatim}
\subsection{Description}
\begin{verbatim}
  Congestion([timeout]): This application will indicate the congestion
condition to the calling channel. If the optional timeout is specified, the
calling channel will be hung up after the specified number of seconds.
Otherwise, this application will wait until the calling channel hangs up.

\end{verbatim}


\section{ContinueWhile}
\subsection{Synopsis}
\begin{verbatim}
Restart a While loop
\end{verbatim}
\subsection{Description}
\begin{verbatim}
Usage:  ContinueWhile()
Returns to the top of the while loop and re-evaluates the conditional.

\end{verbatim}


\section{ControlPlayback}
\subsection{Synopsis}
\begin{verbatim}
Play a file with fast forward and rewind
\end{verbatim}
\subsection{Description}
\begin{verbatim}
  ControlPlayback(file[|skipms[|ff[|rew[|stop[|pause[|restart|options]]]]]]]):
This application will play back the given filename. By default, the '*' key
can be used to rewind, and the '#' key can be used to fast-forward.
Parameters:
  skipms  - This is number of milliseconds to skip when rewinding or
            fast-forwarding.
  ff      - Fast-forward when this DTMF digit is received.
  rew     - Rewind when this DTMF digit is received.
  stop    - Stop playback when this DTMF digit is received.
  pause   - Pause playback when this DTMF digit is received.
  restart - Restart playback when this DTMF digit is received.
Options:
  o(#) - Start at # ms from the beginning of the file.
This application sets the following channel variables upon completion:
  CPLAYBACKSTATUS -  This variable contains the status of the attempt as a text
                     string, one of: SUCCESS | USERSTOPPED | ERROR
  CPLAYBACKOFFSET -  This contains the offset in ms into the file where
                     playback was at when it stopped.  -1 is end of file.

\end{verbatim}


\section{DateTime}
\subsection{Synopsis}
\begin{verbatim}
Says a specified time in a custom format
\end{verbatim}
\subsection{Description}
\begin{verbatim}
DateTime([unixtime][|[timezone][|format]])
  unixtime: time, in seconds since Jan 1, 1970.  May be negative.
              defaults to now.
  timezone: timezone, see /usr/share/zoneinfo for a list.
              defaults to machine default.
  format:   a format the time is to be said in.  See voicemail.conf.
              defaults to "ABdY 'digits/at' IMp"

\end{verbatim}


\section{DBdel}
\subsection{Synopsis}
\begin{verbatim}
Delete a key from the database
\end{verbatim}
\subsection{Description}
\begin{verbatim}
  DBdel(family/key): This application will delete a key from the Asterisk
database.
  This application has been DEPRECATED in favor of the DB_DELETE function.

\end{verbatim}


\section{DBdeltree}
\subsection{Synopsis}
\begin{verbatim}
Delete a family or keytree from the database
\end{verbatim}
\subsection{Description}
\begin{verbatim}
  DBdeltree(family[/keytree]): This application will delete a family or keytree
from the Asterisk database

\end{verbatim}


\section{DeadAGI}
\subsection{Synopsis}
\begin{verbatim}
Executes AGI on a hungup channel
\end{verbatim}
\subsection{Description}
\begin{verbatim}
  [E|Dead]AGI(command|args): Executes an Asterisk Gateway Interface compliant
program on a channel. AGI allows Asterisk to launch external programs
written in any language to control a telephony channel, play audio,
read DTMF digits, etc. by communicating with the AGI protocol on stdin
and stdout.
  This channel will stop dialplan execution on hangup inside of this
application, except when using DeadAGI.  Otherwise, dialplan execution
will continue normally.
  A locally executed AGI script will receive SIGHUP on hangup from the channel
except when using DeadAGI. This can be disabled by setting the AGISIGHUP channel
variable to "no" before executing the AGI application.
  Using 'EAGI' provides enhanced AGI, with incoming audio available out of band
on file descriptor 3

  Use the CLI command 'agi show' to list available agi commands
  This application sets the following channel variable upon completion:
     AGISTATUS      The status of the attempt to the run the AGI script
                    text string, one of SUCCESS | FAILED | NOTFOUND | HANGUP

\end{verbatim}


\section{Dial}
\subsection{Synopsis}
\begin{verbatim}
Place a call and connect to the current channel
\end{verbatim}
\subsection{Description}
\begin{verbatim}
  Dial(Technology/resource[&Tech2/resource2...][|timeout][|options][|URL]):
This application will place calls to one or more specified channels. As soon
as one of the requested channels answers, the originating channel will be
answered, if it has not already been answered. These two channels will then
be active in a bridged call. All other channels that were requested will then
be hung up.
  Unless there is a timeout specified, the Dial application will wait
indefinitely until one of the called channels answers, the user hangs up, or
if all of the called channels are busy or unavailable. Dialplan executing will
continue if no requested channels can be called, or if the timeout expires.

  This application sets the following channel variables upon completion:
    DIALEDTIME   - This is the time from dialing a channel until when it
                   is disconnected.
    ANSWEREDTIME - This is the amount of time for actual call.
    DIALSTATUS   - This is the status of the call:
                   CHANUNAVAIL | CONGESTION | NOANSWER | BUSY | ANSWER | CANCEL
                   DONTCALL | TORTURE | INVALIDARGS
  For the Privacy and Screening Modes, the DIALSTATUS variable will be set to
DONTCALL if the called party chooses to send the calling party to the 'Go Away'
script. The DIALSTATUS variable will be set to TORTURE if the called party
wants to send the caller to the 'torture' script.
  This application will report normal termination if the originating channel
hangs up, or if the call is bridged and either of the parties in the bridge
ends the call.
  The optional URL will be sent to the called party if the channel supports it.
  If the OUTBOUND_GROUP variable is set, all peer channels created by this
application will be put into that group (as in Set(GROUP()=...).
  If the OUTBOUND_GROUP_ONCE variable is set, all peer channels created by this
application will be put into that group (as in Set(GROUP()=...). Unlike OUTBOUND_GROUP,
however, the variable will be unset after use.

  Options:
    A(x) - Play an announcement to the called party, using 'x' as the file.
    C    - Reset the CDR for this call.
    d    - Allow the calling user to dial a 1 digit extension while waiting for
           a call to be answered. Exit to that extension if it exists in the
           current context, or the context defined in the EXITCONTEXT variable,
           if it exists.
    D([called][:calling]) - Send the specified DTMF strings *after* the called
           party has answered, but before the call gets bridged. The 'called'
           DTMF string is sent to the called party, and the 'calling' DTMF
           string is sent to the calling party. Both parameters can be used
           alone.
    f    - Force the callerid of the *calling* channel to be set as the
           extension associated with the channel using a dialplan 'hint'.
           For example, some PSTNs do not allow CallerID to be set to anything
           other than the number assigned to the caller.
    g    - Proceed with dialplan execution at the current extension if the
           destination channel hangs up.
    G(context^exten^pri) - If the call is answered, transfer the calling party to
           the specified priority and the called party to the specified priority+1.
           Optionally, an extension, or extension and context may be specified. 
           Otherwise, the current extension is used. You cannot use any additional
           action post answer options in conjunction with this option.
    h    - Allow the called party to hang up by sending the '*' DTMF digit.
    H    - Allow the calling party to hang up by hitting the '*' DTMF digit.
    i    - Asterisk will ignore any forwarding requests it may receive on this
           dial attempt.
    k    - Allow the called party to enable parking of the call by sending
           the DTMF sequence defined for call parking in features.conf.
    K    - Allow the calling party to enable parking of the call by sending
           the DTMF sequence defined for call parking in features.conf.
    L(x[:y][:z]) - Limit the call to 'x' ms. Play a warning when 'y' ms are
           left. Repeat the warning every 'z' ms. The following special
           variables can be used with this option:
           * LIMIT_PLAYAUDIO_CALLER   yes|no (default yes)
                                      Play sounds to the caller.
           * LIMIT_PLAYAUDIO_CALLEE   yes|no
                                      Play sounds to the callee.
           * LIMIT_TIMEOUT_FILE       File to play when time is up.
           * LIMIT_CONNECT_FILE       File to play when call begins.
           * LIMIT_WARNING_FILE       File to play as warning if 'y' is defined.
                                      The default is to say the time remaining.
    m([class]) - Provide hold music to the calling party until a requested
           channel answers. A specific MusicOnHold class can be
           specified.
    M(x[^arg]) - Execute the Macro for the *called* channel before connecting
           to the calling channel. Arguments can be specified to the Macro
           using '^' as a delimeter. The Macro can set the variable
           MACRO_RESULT to specify the following actions after the Macro is
           finished executing.
           * ABORT        Hangup both legs of the call.
           * CONGESTION   Behave as if line congestion was encountered.
           * BUSY         Behave as if a busy signal was encountered.
           * CONTINUE     Hangup the called party and allow the calling party
                          to continue dialplan execution at the next priority.
           * GOTO:<context>^<exten>^<priority> - Transfer the call to the
                          specified priority. Optionally, an extension, or
                          extension and priority can be specified.
           You cannot use any additional action post answer options in conjunction
           with this option. Also, pbx services are not run on the peer (called) channel,
           so you will not be able to set timeouts via the TIMEOUT() function in this macro.
    n    - This option is a modifier for the screen/privacy mode. It specifies
           that no introductions are to be saved in the priv-callerintros
           directory.
    N    - This option is a modifier for the screen/privacy mode. It specifies
           that if callerID is present, do not screen the call.
    o    - Specify that the CallerID that was present on the *calling* channel
           be set as the CallerID on the *called* channel. This was the
           behavior of Asterisk 1.0 and earlier.
    O([x]) - "Operator Services" mode (Zaptel channel to Zaptel channel
             only, if specified on non-Zaptel interface, it will be ignored).
             When the destination answers (presumably an operator services
             station), the originator no longer has control of their line.
             They may hang up, but the switch will not release their line
             until the destination party hangs up (the operator). Specified
             without an arg, or with 1 as an arg, the originator hanging up
             will cause the phone to ring back immediately. With a 2 specified,
             when the "operator" flashes the trunk, it will ring their phone
             back.
    p    - This option enables screening mode. This is basically Privacy mode
           without memory.
    P([x]) - Enable privacy mode. Use 'x' as the family/key in the database if
           it is provided. The current extension is used if a database
           family/key is not specified.
    r    - Indicate ringing to the calling party. Pass no audio to the calling
           party until the called channel has answered.
    S(x) - Hang up the call after 'x' seconds *after* the called party has
           answered the call.
    t    - Allow the called party to transfer the calling party by sending the
           DTMF sequence defined in features.conf.
    T    - Allow the calling party to transfer the called party by sending the
           DTMF sequence defined in features.conf.
    U(x[^arg]) - Execute via Gosub the routine 'x' for the *called* channel before connecting
           to the calling channel. Arguments can be specified to the Gosub
           using '^' as a delimeter. The Gosub routine can set the variable
           GOSUB_RESULT to specify the following actions after the Gosub returns.
           * ABORT        Hangup both legs of the call.
           * CONGESTION   Behave as if line congestion was encountered.
           * BUSY         Behave as if a busy signal was encountered.
           * CONTINUE     Hangup the called party and allow the calling party
                          to continue dialplan execution at the next priority.
           * GOTO:<context>^<exten>^<priority> - Transfer the call to the
                          specified priority. Optionally, an extension, or
                          extension and priority can be specified.
           You cannot use any additional action post answer options in conjunction
           with this option. Also, pbx services are not run on the peer (called) channel,
           so you will not be able to set timeouts via the TIMEOUT() function in this routine.
    w    - Allow the called party to enable recording of the call by sending
           the DTMF sequence defined for one-touch recording in features.conf.
    W    - Allow the calling party to enable recording of the call by sending
           the DTMF sequence defined for one-touch recording in features.conf.

\end{verbatim}


\section{Dictate}
\subsection{Synopsis}
\begin{verbatim}
Virtual Dictation Machine
\end{verbatim}
\subsection{Description}
\begin{verbatim}
  Dictate([<base_dir>[|<filename>]])
Start dictation machine using optional base dir for files.

\end{verbatim}


\section{Directory}
\subsection{Synopsis}
\begin{verbatim}
Provide directory of voicemail extensions
\end{verbatim}
\subsection{Description}
\begin{verbatim}
  Directory(vm-context[|dial-context[|options]]): This application will present
the calling channel with a directory of extensions from which they can search
by name. The list of names and corresponding extensions is retrieved from the
voicemail configuration file, voicemail.conf.
  This application will immediately exit if one of the following DTMF digits are
received and the extension to jump to exists:
    0 - Jump to the 'o' extension, if it exists.
    * - Jump to the 'a' extension, if it exists.

  Parameters:
    vm-context   - This is the context within voicemail.conf to use for the
                   Directory.
    dial-context - This is the dialplan context to use when looking for an
                   extension that the user has selected, or when jumping to the
                   'o' or 'a' extension.

  Options:
    e - In addition to the name, also read the extension number to the
        caller before presenting dialing options.
    f - Allow the caller to enter the first name of a user in the directory
        instead of using the last name.

\end{verbatim}


\section{DISA}
\subsection{Synopsis}
\begin{verbatim}
DISA (Direct Inward System Access)
\end{verbatim}
\subsection{Description}
\begin{verbatim}
DISA(<numeric passcode>[|<context>[|<cid>[|mailbox[|options]]]]) or
DISA(<filename>[||||options])
The DISA, Direct Inward System Access, application allows someone from 
outside the telephone switch (PBX) to obtain an "internal" system 
dialtone and to place calls from it as if they were placing a call from 
within the switch.
DISA plays a dialtone. The user enters their numeric passcode, followed by
the pound sign (#). If the passcode is correct, the user is then given
system dialtone within <context> on which a call may be placed. If the user
enters an invalid extension and extension "i" exists in the specified
context, it will be used.

If you need to present a DISA dialtone without entering a password, simply
set <passcode> to "no-password".

Be aware that using this may compromise the security of your PBX.

The arguments to this application (in extensions.conf) allow either
specification of a single global passcode (that everyone uses), or
individual passcodes contained in a file.

The file that contains the passcodes (if used) allows a complete
specification of all of the same arguments available on the command
line, with the sole exception of the options. The file may contain blank
lines, or comments starting with "#" or ";".

<context> specifies the dialplan context in which the user-entered extension
will be matched. If no context is specified, the DISA application defaults
the context to "disa". Presumably a normal system will have a special
context set up for DISA use with some or a lot of restrictions.

<cid> specifies a new (different) callerid to be used for this call.

<mailbox[@context]> will cause a stutter-dialtone (indication "dialrecall")
to be used, if the specified mailbox contains any new messages.

The following options are available:
  n - the DISA application will not answer initially.
  p - the extension entered will be considered complete when a '#' is entered.

\end{verbatim}


\section{DumpChan}
\subsection{Synopsis}
\begin{verbatim}
Dump Info About The Calling Channel
\end{verbatim}
\subsection{Description}
\begin{verbatim}
   DumpChan([<min_verbose_level>])
Displays information on channel and listing of all channel
variables. If min_verbose_level is specified, output is only
displayed when the verbose level is currently set to that number
or greater. 

\end{verbatim}


\section{EAGI}
\subsection{Synopsis}
\begin{verbatim}
Executes an EAGI compliant application
\end{verbatim}
\subsection{Description}
\begin{verbatim}
  [E|Dead]AGI(command|args): Executes an Asterisk Gateway Interface compliant
program on a channel. AGI allows Asterisk to launch external programs
written in any language to control a telephony channel, play audio,
read DTMF digits, etc. by communicating with the AGI protocol on stdin
and stdout.
  This channel will stop dialplan execution on hangup inside of this
application, except when using DeadAGI.  Otherwise, dialplan execution
will continue normally.
  A locally executed AGI script will receive SIGHUP on hangup from the channel
except when using DeadAGI. This can be disabled by setting the AGISIGHUP channel
variable to "no" before executing the AGI application.
  Using 'EAGI' provides enhanced AGI, with incoming audio available out of band
on file descriptor 3

  Use the CLI command 'agi show' to list available agi commands
  This application sets the following channel variable upon completion:
     AGISTATUS      The status of the attempt to the run the AGI script
                    text string, one of SUCCESS | FAILED | NOTFOUND | HANGUP

\end{verbatim}


\section{Echo}
\subsection{Synopsis}
\begin{verbatim}
Echo audio, video, or DTMF back to the calling party
\end{verbatim}
\subsection{Description}
\begin{verbatim}
  Echo(): This application will echo any audio, video, or DTMF frames read from
the calling channel back to itself. If the DTMF digit '#' is received, the
application will exit.

\end{verbatim}


\section{EndWhile}
\subsection{Synopsis}
\begin{verbatim}
End a while loop
\end{verbatim}
\subsection{Description}
\begin{verbatim}
Usage:  EndWhile()
Return to the previous called While

\end{verbatim}


\section{Exec}
\subsection{Synopsis}
\begin{verbatim}
Executes dialplan application
\end{verbatim}
\subsection{Description}
\begin{verbatim}
Usage: Exec(appname(arguments))
  Allows an arbitrary application to be invoked even when not
hardcoded into the dialplan.  If the underlying application
terminates the dialplan, or if the application cannot be found,
Exec will terminate the dialplan.
  To invoke external applications, see the application System.
  If you would like to catch any error instead, see TryExec.

\end{verbatim}


\section{ExecIf}
\subsection{Synopsis}
\begin{verbatim}
Executes dialplan application, conditionally
\end{verbatim}
\subsection{Description}
\begin{verbatim}
Usage:  ExecIF (<expr>|<app>|<data>)
If <expr> is true, execute and return the result of <app>(<data>).
If <expr> is true, but <app> is not found, then the application
will return a non-zero value.

\end{verbatim}


\section{ExecIfTime}
\subsection{Synopsis}
\begin{verbatim}
Conditional application execution based on the current time
\end{verbatim}
\subsection{Description}
\begin{verbatim}
  ExecIfTime(<times>|<weekdays>|<mdays>|<months>?appname[|appargs]):
This application will execute the specified dialplan application, with optional
arguments, if the current time matches the given time specification.

\end{verbatim}


\section{ExitWhile}
\subsection{Synopsis}
\begin{verbatim}
End a While loop
\end{verbatim}
\subsection{Description}
\begin{verbatim}
Usage:  ExitWhile()
Exits a While loop, whether or not the conditional has been satisfied.

\end{verbatim}


\section{ExtenSpy}
\subsection{Synopsis}
\begin{verbatim}
Listen to a channel, and optionally whisper into it
\end{verbatim}
\subsection{Description}
\begin{verbatim}
  ExtenSpy(exten[@context][|options]): This application is used to listen to the
audio from an Asterisk channel. This includes the audio coming in and
out of the channel being spied on. Only channels created by outgoing calls for the
specified extension will be selected for spying. If the optional context is not
supplied, the current channel's context will be used.
  While spying, the following actions may be performed:
    - Dialing # cycles the volume level.
    - Dialing * will stop spying and look for another channel to spy on.
  Note: The X option superseeds the two features above in that if a valid
        single digit extension exists in the correct context it ChanSpy will
        exit to it.
  Options:
    b             - Only spy on channels involved in a bridged call.
    g(grp)        - Match only channels where their ${SPYGROUP} variable is set to
                    contain 'grp' in an optional : delimited list.
    q             - Don't play a beep when beginning to spy on a channel, or speak the
                    selected channel name.
    r[(basename)] - Record the session to the monitor spool directory. An
                    optional base for the filename may be specified. The
                    default is 'chanspy'.
    v([value])    - Adjust the initial volume in the range from -4 to 4. A
                    negative value refers to a quieter setting.
    w             - Enable 'whisper' mode, so the spying channel can talk to
                    the spied-on channel.
    W             - Enable 'private whisper' mode, so the spying channel can
                    talk to the spied-on channel but cannot listen to that
                    channel.
    o             - Only listen to audio coming from this channel.
    X             - Allow the user to exit ChanSpy to a valid single digit
                    numeric extension in the current context or the context
                    specified by the SPY_EXIT_CONTEXT channel variable. The
                    name of the last channel that was spied on will be stored
                    in the SPY_CHANNEL variable.

\end{verbatim}


\section{ExternalIVR}
\subsection{Synopsis}
\begin{verbatim}
Interfaces with an external IVR application
\end{verbatim}
\subsection{Description}
\begin{verbatim}
  ExternalIVR(command[|arg[|arg...]]): Forks an process to run the supplied command,
and starts a generator on the channel. The generator's play list is
controlled by the external application, which can add and clear entries
via simple commands issued over its stdout. The external application
will receive all DTMF events received on the channel, and notification
if the channel is hung up. The application will not be forcibly terminated
when the channel is hung up.
See doc/externalivr.txt for a protocol specification.

\end{verbatim}


\section{Festival}
\subsection{Synopsis}
\begin{verbatim}
Say text to the user
\end{verbatim}
\subsection{Description}
\begin{verbatim}
  Festival(text[|intkeys]):  Connect to Festival, send the argument, get back the waveform,play it to the user, allowing any given interrupt keys to immediately terminate and return
the value, or 'any' to allow any number back (useful in dialplan)

\end{verbatim}


\section{Flash}
\subsection{Synopsis}
\begin{verbatim}
Flashes a Zap Trunk
\end{verbatim}
\subsection{Description}
\begin{verbatim}
  Flash(): Sends a flash on a zap trunk.  This is only a hack for
people who want to perform transfers and such via AGI and is generally
quite useless oths application will only work on Zap trunks.

\end{verbatim}


\section{FollowMe}
\subsection{Synopsis}
\begin{verbatim}
Find-Me/Follow-Me application
\end{verbatim}
\subsection{Description}
\begin{verbatim}
  FollowMe(followmeid|options):
This application performs Find-Me/Follow-Me functionality for the caller
as defined in the profile matching the <followmeid> parameter in
followme.conf. If the specified <followmeid> profile doesn't exist in
followme.conf, execution will be returned to the dialplan and call
execution will continue at the next priority.

  Options:
    s    - Playback the incoming status message prior to starting the follow-me step(s)
    a    - Record the caller's name so it can be announced to the callee on each step
    n    - Playback the unreachable status message if we've run out of steps to reach the
           or the callee has elected not to be reachable.
Returns -1 on hangup

\end{verbatim}


\section{ForkCDR}
\subsection{Synopsis}
\begin{verbatim}
Forks the Call Data Record
\end{verbatim}
\subsection{Description}
\begin{verbatim}
  ForkCDR([options]):  Causes the Call Data Record to fork an additional
cdr record starting from the time of the fork call
If the option 'v' is passed all cdr variables will be passed along also.

\end{verbatim}


\section{GetCPEID}
\subsection{Synopsis}
\begin{verbatim}
Get ADSI CPE ID
\end{verbatim}
\subsection{Description}
\begin{verbatim}
  GetCPEID: Obtains and displays ADSI CPE ID and other information in order
to properly setup zapata.conf for on-hook operations.

\end{verbatim}


\section{Gosub}
\subsection{Synopsis}
\begin{verbatim}
Jump to label, saving return address
\end{verbatim}
\subsection{Description}
\begin{verbatim}
Gosub([[context|]exten|]priority[(arg1[|...][|argN])])
  Jumps to the label specified, saving the return address.

\end{verbatim}


\section{GosubIf}
\subsection{Synopsis}
\begin{verbatim}
Conditionally jump to label, saving return address
\end{verbatim}
\subsection{Description}
\begin{verbatim}
GosubIf(condition?labeliftrue[(arg1[|...])][:labeliffalse[(arg1[|...])]])
  If the condition is true, then jump to labeliftrue.  If false, jumps to
labeliffalse, if specified.  In either case, a jump saves the return point
in the dialplan, to be returned to with a Return.

\end{verbatim}


\section{Goto}
\subsection{Synopsis}
\begin{verbatim}
Jump to a particular priority, extension, or context
\end{verbatim}
\subsection{Description}
\begin{verbatim}
  Goto([[context|]extension|]priority): This application will set the current
context, extension, and priority in the channel structure. After it completes, the
pbx engine will continue dialplan execution at the specified location.
If no specific extension, or extension and context, are specified, then this
application will just set the specified priority of the current extension.
  At least a priority is required as an argument, or the goto will return a -1,
and the channel and call will be terminated.
  If the location that is put into the channel information is bogus, and asterisk cannot
find that location in the dialplan,
then the execution engine will try to find and execute the code in the 'i' (invalid)
extension in the current context. If that does not exist, it will try to execute the
'h' extension. If either or neither the 'h' or 'i' extensions have been defined, the
channel is hung up, and the execution of instructions on the channel is terminated.
What this means is that, for example, you specify a context that does not exist, then
it will not be possible to find the 'h' or 'i' extensions, and the call will terminate!

\end{verbatim}


\section{GotoIf}
\subsection{Synopsis}
\begin{verbatim}
Conditional goto
\end{verbatim}
\subsection{Description}
\begin{verbatim}
  GotoIf(condition?[labeliftrue]:[labeliffalse]): This application will set the current
context, extension, and priority in the channel structure based on the evaluation of
the given condition. After this application completes, the
pbx engine will continue dialplan execution at the specified location in the dialplan.
The channel will continue at
'labeliftrue' if the condition is true, or 'labeliffalse' if the condition is
false. The labels are specified with the same syntax as used within the Goto
application.  If the label chosen by the condition is omitted, no jump is
performed, and the execution passes to the next instruction.
If the target location is bogus, and does not exist, the execution engine will try 
to find and execute the code in the 'i' (invalid)
extension in the current context. If that does not exist, it will try to execute the
'h' extension. If either or neither the 'h' or 'i' extensions have been defined, the
channel is hung up, and the execution of instructions on the channel is terminated.
Remember that this command can set the current context, and if the context specified
does not exist, then it will not be able to find any 'h' or 'i' extensions there, and
the channel and call will both be terminated!

\end{verbatim}


\section{GotoIfTime}
\subsection{Synopsis}
\begin{verbatim}
Conditional Goto based on the current time
\end{verbatim}
\subsection{Description}
\begin{verbatim}
  GotoIfTime(<times>|<weekdays>|<mdays>|<months>?[[context|]exten|]priority):
This application will set the context, extension, and priority in the channel structure
if the current time matches the given time specification. Otherwise, nothing is done.
Further information on the time specification can be found in examples
illustrating how to do time-based context includes in the dialplan.
If the target jump location is bogus, the same actions would be taken as for Goto.

\end{verbatim}


\section{Hangup}
\subsection{Synopsis}
\begin{verbatim}
Hang up the calling channel
\end{verbatim}
\subsection{Description}
\begin{verbatim}
  Hangup([causecode]): This application will hang up the calling channel.
If a causecode is given the channel's hangup cause will be set to the given
value.

\end{verbatim}


\section{IAX2Provision}
\subsection{Synopsis}
\begin{verbatim}
Provision a calling IAXy with a given template
\end{verbatim}
\subsection{Description}
\begin{verbatim}
  IAX2Provision([template]): Provisions the calling IAXy (assuming
the calling entity is in fact an IAXy) with the given template or
default if one is not specified.  Returns -1 on error or 0 on success.

\end{verbatim}


\section{ICES}
\subsection{Synopsis}
\begin{verbatim}
Encode and stream using 'ices'
\end{verbatim}
\subsection{Description}
\begin{verbatim}
  ICES(config.xml) Streams to an icecast server using ices
(available separately).  A configuration file must be supplied
for ices (see examples/asterisk-ices.conf). 

\end{verbatim}


\section{ImportVar}
\subsection{Synopsis}
\begin{verbatim}
Import a variable from a channel into a new variable
\end{verbatim}
\subsection{Description}
\begin{verbatim}
  ImportVar(newvar=channelname|variable): This application imports a variable
from the specified channel (as opposed to the current one) and stores it as
a variable in the current channel (the channel that is calling this
application). Variables created by this application have the same inheritance
properties as those created with the Set application. See the documentation for
Set for more information.

\end{verbatim}


\section{IVRDemo}
\subsection{Synopsis}
\begin{verbatim}
IVR Demo Application
\end{verbatim}
\subsection{Description}
\begin{verbatim}
  This is a skeleton application that shows you the basic structure to create your
own asterisk applications and demonstrates the IVR demo.

\end{verbatim}


\section{JabberSend}
\subsection{Synopsis}
\begin{verbatim}
JabberSend(jabber,screenname,message)
\end{verbatim}
\subsection{Description}
\begin{verbatim}
JabberSend(Jabber,ScreenName,Message)
  Jabber - Client or transport Asterisk uses to connect to Jabber
  ScreenName - User Name to message.
  Message - Message to be sent to the buddy

\end{verbatim}


\section{JabberStatus}
\subsection{Synopsis}
\begin{verbatim}
JabberStatus(Jabber,ScreenName,Variable)
\end{verbatim}
\subsection{Description}
\begin{verbatim}
JabberStatus(Jabber,ScreenName,Variable)
  Jabber - Client or transport Asterisk uses to connect to Jabber
  ScreenName - User Name to retrieve status from.
  Variable - Variable to store presence in will be 1-6.
             In order, Online, Chatty, Away, XAway, DND, Offline
             If not in roster variable will = 7

\end{verbatim}


\section{KeepAlive}
\subsection{Synopsis}
\begin{verbatim}
returns AST_PBX_KEEPALIVE value
\end{verbatim}
\subsection{Description}
\begin{verbatim}
  KeepAlive(): This application is chiefly meant for internal use with Gosubs.
Please do not run it alone from the dialplan!

\end{verbatim}


\section{Log}
\subsection{Synopsis}
\begin{verbatim}
Send arbitrary text to a selected log level
\end{verbatim}
\subsection{Description}
\begin{verbatim}
Log(<level>|<message>)
  level must be one of ERROR, WARNING, NOTICE, DEBUG, VERBOSE, DTMF

\end{verbatim}


\section{Macro}
\subsection{Synopsis}
\begin{verbatim}
Macro Implementation
\end{verbatim}
\subsection{Description}
\begin{verbatim}
  Macro(macroname|arg1|arg2...): Executes a macro using the context
'macro-<macroname>', jumping to the 's' extension of that context and
executing each step, then returning when the steps end. 
The calling extension, context, and priority are stored in ${MACRO_EXTEN}, 
${MACRO_CONTEXT} and ${MACRO_PRIORITY} respectively.  Arguments become
${ARG1}, ${ARG2}, etc in the macro context.
If you Goto out of the Macro context, the Macro will terminate and control
will be returned at the location of the Goto.
If ${MACRO_OFFSET} is set at termination, Macro will attempt to continue
at priority MACRO_OFFSET + N + 1 if such a step exists, and N + 1 otherwise.
Extensions: While a macro is being executed, it becomes the current context.
            This means that if a hangup occurs, for instance, that the macro
            will be searched for an 'h' extension, NOT the context from which
            the macro was called. So, make sure to define all appropriate
            extensions in your macro! (Note: AEL does not use macros)
WARNING: Because of the way Macro is implemented (it executes the priorities
         contained within it via sub-engine), and a fixed per-thread
         memory stack allowance, macros are limited to 7 levels
         of nesting (macro calling macro calling macro, etc.); It
         may be possible that stack-intensive applications in deeply nested macros
         could cause asterisk to crash earlier than this limit. It is advised that
         if you need to deeply nest macro calls, that you use the Gosub application
         (now allows arguments like a Macro) with explict Return() calls instead.

\end{verbatim}


\section{MacroExclusive}
\subsection{Synopsis}
\begin{verbatim}
Exclusive Macro Implementation
\end{verbatim}
\subsection{Description}
\begin{verbatim}
  MacroExclusive(macroname|arg1|arg2...):
Executes macro defined in the context 'macro-macroname'
Only one call at a time may run the macro.
(we'll wait if another call is busy executing in the Macro)
Arguments and return values as in application Macro()

\end{verbatim}


\section{MacroExit}
\subsection{Synopsis}
\begin{verbatim}
Exit From Macro
\end{verbatim}
\subsection{Description}
\begin{verbatim}
  MacroExit():
Causes the currently running macro to exit as if it had
ended normally by running out of priorities to execute.
If used outside a macro, will likely cause unexpected
behavior.

\end{verbatim}


\section{MacroIf}
\subsection{Synopsis}
\begin{verbatim}
Conditional Macro Implementation
\end{verbatim}
\subsection{Description}
\begin{verbatim}
  MacroIf(<expr>?macroname_a[|arg1][:macroname_b[|arg1]])
Executes macro defined in <macroname_a> if <expr> is true
(otherwise <macroname_b> if provided)
Arguments and return values as in application macro()

\end{verbatim}


\section{MailboxExists}
\subsection{Synopsis}
\begin{verbatim}
Check to see if Voicemail mailbox exists
\end{verbatim}
\subsection{Description}
\begin{verbatim}
  MailboxExists(mailbox[@context][|options]): Check to see if the specified
mailbox exists. If no voicemail context is specified, the 'default' context
will be used.
  This application will set the following channel variable upon completion:
    VMBOXEXISTSSTATUS - This will contain the status of the execution of the
                        MailboxExists application. Possible values include:
                        SUCCESS | FAILED

  Options: (none)

\end{verbatim}


\section{MeetMe}
\subsection{Synopsis}
\begin{verbatim}
MeetMe conference bridge
\end{verbatim}
\subsection{Description}
\begin{verbatim}
  MeetMe([confno][,[options][,pin]]): Enters the user into a specified MeetMe
conference.  If the conference number is omitted, the user will be prompted
to enter one.  User can exit the conference by hangup, or if the 'p' option
is specified, by pressing '#'.
Please note: The Zaptel kernel modules and at least one hardware driver (or ztdummy)
             must be present for conferencing to operate properly. In addition, the chan_zap
             channel driver must be loaded for the 'i' and 'r' options to operate at all.

The option string may contain zero or more of the following characters:
      'a' -- set admin mode
      'A' -- set marked mode
      'b' -- run AGI script specified in ${MEETME_AGI_BACKGROUND}
             Default: conf-background.agi  (Note: This does not work with
             non-Zap channels in the same conference)
      'c' -- announce user(s) count on joining a conference
      'C' -- continue in dialplan when kicked out of conference
      'd' -- dynamically add conference
      'D' -- dynamically add conference, prompting for a PIN
      'e' -- select an empty conference
      'E' -- select an empty pinless conference
      'F' -- Pass DTMF through the conference.  DTMF used to activate any
             conference features will not be passed through.
      'i' -- announce user join/leave with review
      'I' -- announce user join/leave without review
      'l' -- set listen only mode (Listen only, no talking)
      'm' -- set initially muted
      'M' -- enable music on hold when the conference has a single caller
      'o' -- set talker optimization - treats talkers who aren't speaking as
             being muted, meaning (a) No encode is done on transmission and
             (b) Received audio that is not registered as talking is omitted
             causing no buildup in background noise
      'p' -- allow user to exit the conference by pressing '#'
      'P' -- always prompt for the pin even if it is specified
      'q' -- quiet mode (don't play enter/leave sounds)
      'r' -- Record conference (records as ${MEETME_RECORDINGFILE}
             using format ${MEETME_RECORDINGFORMAT}). Default filename is
             meetme-conf-rec-${CONFNO}-${UNIQUEID} and the default format is
             wav.
      's' -- Present menu (user or admin) when '*' is received ('send' to menu)
      't' -- set talk only mode. (Talk only, no listening)
      'T' -- set talker detection (sent to manager interface and meetme list)
      'w[(<secs>)]'
          -- wait until the marked user enters the conference
      'x' -- close the conference when last marked user exits
      'X' -- allow user to exit the conference by entering a valid single
             digit extension ${MEETME_EXIT_CONTEXT} or the current context
             if that variable is not defined.
      '1' -- do not play message when first person enters

\end{verbatim}


\section{MeetMeAdmin}
\subsection{Synopsis}
\begin{verbatim}
MeetMe conference Administration
\end{verbatim}
\subsection{Description}
\begin{verbatim}
  MeetMeAdmin(confno,command[,user]): Run admin command for conference
      'e' -- Eject last user that joined
      'k' -- Kick one user out of conference
      'K' -- Kick all users out of conference
      'l' -- Unlock conference
      'L' -- Lock conference
      'm' -- Unmute one user
      'M' -- Mute one user
      'n' -- Unmute all users in the conference
      'N' -- Mute all non-admin users in the conference
      'r' -- Reset one user's volume settings
      'R' -- Reset all users volume settings
      's' -- Lower entire conference speaking volume
      'S' -- Raise entire conference speaking volume
      't' -- Lower one user's talk volume
      'T' -- Raise one user's talk volume
      'u' -- Lower one user's listen volume
      'U' -- Raise one user's listen volume
      'v' -- Lower entire conference listening volume
      'V' -- Raise entire conference listening volume

\end{verbatim}


\section{MeetMeChannelAdmin}
\subsection{Synopsis}
\begin{verbatim}
MeetMe conference Administration (channel specific)
\end{verbatim}
\subsection{Description}
\begin{verbatim}
  MeetMeChannelAdmin(channel|command): Run admin command for a specific
channel in any coference.
      'k' -- Kick the specified user out of the conference he is in
      'm' -- Unmute the specified user
      'M' -- Mute the specified user

\end{verbatim}


\section{MeetMeCount}
\subsection{Synopsis}
\begin{verbatim}
MeetMe participant count
\end{verbatim}
\subsection{Description}
\begin{verbatim}
  MeetMeCount(confno[|var]): Plays back the number of users in the specified
MeetMe conference. If var is specified, playback will be skipped and the value
will be returned in the variable. Upon app completion, MeetMeCount will hangup
the channel, unless priority n+1 exists, in which case priority progress will
continue.
A ZAPTEL INTERFACE MUST BE INSTALLED FOR CONFERENCING FUNCTIONALITY.

\end{verbatim}


\section{Milliwatt}
\subsection{Synopsis}
\begin{verbatim}
Generate a Constant 1000Hz tone at 0dbm (mu-law)
\end{verbatim}
\subsection{Description}
\begin{verbatim}
Milliwatt(): Generate a Constant 1000Hz tone at 0dbm (mu-law)

\end{verbatim}


\section{MinivmAccMess}
\subsection{Synopsis}
\begin{verbatim}
Record account specific messages
\end{verbatim}
\subsection{Description}
\begin{verbatim}
Syntax: MinivmAccmess(username@domain,option)
This application is part of the Mini-Voicemail system, configured in minivm.conf.
Use this application to record account specific audio/video messages for
busy, unavailable and temporary messages.
Account specific directories will be created if they do not exist.

The option selects message to be recorded:
   u      Unavailable
   b      Busy
   t      Temporary (overrides busy and unavailable)
   n      Account name

Result is given in channel variable MINIVM_ACCMESS_STATUS
        The possible values are:     SUCCESS |  FAILED
	 FAILED is set if the file can't be created.


\end{verbatim}


\section{MinivmDelete}
\subsection{Synopsis}
\begin{verbatim}
Delete Mini-Voicemail voicemail messages
\end{verbatim}
\subsection{Description}
\begin{verbatim}
Syntax: MinivmDelete(filename)
This application is part of the Mini-Voicemail system, configured in minivm.conf.
It deletes voicemail file set in MVM_FILENAME or given filename.

Result is given in channel variable MINIVM_DELETE_STATUS
        The possible values are:     SUCCESS |  FAILED
	 FAILED is set if the file does not exist or can't be deleted.


\end{verbatim}


\section{MinivmGreet}
\subsection{Synopsis}
\begin{verbatim}
Play Mini-Voicemail prompts
\end{verbatim}
\subsection{Description}
\begin{verbatim}
Syntax: MinivmGreet(username@domain[,options])
This application is part of the Mini-Voicemail system, configured in minivm.conf.
MinivmGreet() plays default prompts or user specific prompts for an account.
Busy and unavailable messages can be choosen, but will be overridden if a temporary
message exists for the account.

Result is given in channel variable MINIVM_GREET_STATUS
        The possible values are:     SUCCESS | USEREXIT | FAILED

  Options:
    b    - Play the 'busy' greeting to the calling party.
    s    - Skip the playback of instructions for leaving a message to the
           calling party.
    u    - Play the 'unavailable greeting.


\end{verbatim}


\section{MinivmNotify}
\subsection{Synopsis}
\begin{verbatim}
Notify voicemail owner about new messages.
\end{verbatim}
\subsection{Description}
\begin{verbatim}
Syntax: MinivmNotify(username@domain[,template])
This application is part of the Mini-Voicemail system, configured in minivm.conf.
MiniVMnotify forwards messages about new voicemail to e-mail and pager.
If there's no user account for that address, a temporary account will
be used with default options (set in minivm.conf).
The recorded file name and path will be read from MVM_FILENAME and the 
duration of the message will be accessed from MVM_DURATION (set by MinivmRecord() )
If the channel variable MVM_COUNTER is set, this will be used in the
message file name and available in the template for the message.
If not template is given, the default email template will be used to send email and
default pager template to send paging message (if the user account is configured with
a paging address.

Result is given in channel variable MINIVM_NOTIFY_STATUS
        The possible values are:     SUCCESS | FAILED


\end{verbatim}


\section{MinivmRecord}
\subsection{Synopsis}
\begin{verbatim}
Receive Mini-Voicemail and forward via e-mail
\end{verbatim}
\subsection{Description}
\begin{verbatim}
Syntax: MinivmRecord(username@domain[,options])
This application is part of the Mini-Voicemail system, configured in minivm.conf.
MiniVM records audio file in configured format and forwards message to e-mail and pager.
If there's no user account for that address, a temporary account will
be used with default options.
The recorded file name and path will be stored in MINIVM_FILENAME and the 
duration of the message will be stored in MINIVM_DURATION

Note: If the caller hangs up after the recording, the only way to send
the message and clean up is to execute in the "h" extension.

The application will exit if any of the following DTMF digits are 
received and the requested extension exist in the current context.
    0 - Jump to the 'o' extension in the current dialplan context.
    * - Jump to the 'a' extension in the current dialplan context.

Result is given in channel variable MINIVM_RECORD_STATUS
        The possible values are:     SUCCESS | USEREXIT | FAILED

  Options:
    g(#) - Use the specified amount of gain when recording the voicemail
           message. The units are whole-number decibels (dB).


\end{verbatim}


\section{MixMonitor}
\subsection{Synopsis}
\begin{verbatim}
Record a call and mix the audio during the recording
\end{verbatim}
\subsection{Description}
\begin{verbatim}
  MixMonitor(<file>.<ext>[|<options>[|<command>]])

Records the audio on the current channel to the specified file.
If the filename is an absolute path, uses that path, otherwise
creates the file in the configured monitoring directory from
asterisk.conf.

Valid options:
 a      - Append to the file instead of overwriting it.
 b      - Only save audio to the file while the channel is bridged.
          Note: Does not include conferences or sounds played to each bridged
                party.
 v(<x>) - Adjust the heard volume by a factor of <x> (range -4 to 4)
 V(<x>) - Adjust the spoken volume by a factor of <x> (range -4 to 4)
 W(<x>) - Adjust the both heard and spoken volumes by a factor of <x>
         (range -4 to 4)

<command> will be executed when the recording is over
Any strings matching ^{X} will be unescaped to ${X}.
All variables will be evaluated at the time MixMonitor is called.
The variable MIXMONITOR_FILENAME will contain the filename used to record.

\end{verbatim}


\section{Monitor}
\subsection{Synopsis}
\begin{verbatim}
Monitor a channel
\end{verbatim}
\subsection{Description}
\begin{verbatim}
Monitor([file_format[:urlbase]|[fname_base]|[options]]):
Used to start monitoring a channel. The channel's input and output
voice packets are logged to files until the channel hangs up or
monitoring is stopped by the StopMonitor application.
  file_format		optional, if not set, defaults to "wav"
  fname_base		if set, changes the filename used to the one specified.
  options:
    m   - when the recording ends mix the two leg files into one and
          delete the two leg files.  If the variable MONITOR_EXEC is set, the
          application referenced in it will be executed instead of
          soxmix and the raw leg files will NOT be deleted automatically.
          soxmix or MONITOR_EXEC is handed 3 arguments, the two leg files
          and a target mixed file name which is the same as the leg file names
          only without the in/out designator.
          If MONITOR_EXEC_ARGS is set, the contents will be passed on as
          additional arguements to MONITOR_EXEC
          Both MONITOR_EXEC and the Mix flag can be set from the
          administrator interface

    b   - Don't begin recording unless a call is bridged to another channel

Returns -1 if monitor files can't be opened or if the channel is already
monitored, otherwise 0.

\end{verbatim}


\section{Morsecode}
\subsection{Synopsis}
\begin{verbatim}
Plays morse code
\end{verbatim}
\subsection{Description}
\begin{verbatim}
Usage: Morsecode(<string>)
Plays the Morse code equivalent of the passed string.  If the variable
MORSEDITLEN is set, it will use that value for the length (in ms) of the dit
(defaults to 80).  Additionally, if MORSETONE is set, it will use that tone
(in Hz).  The tone default is 800.

\end{verbatim}


\section{MP3Player}
\subsection{Synopsis}
\begin{verbatim}
Play an MP3 file or stream
\end{verbatim}
\subsection{Description}
\begin{verbatim}
  MP3Player(location) Executes mpg123 to play the given location,
which typically would be a filename or a URL. User can exit by pressing
any key on the dialpad, or by hanging up.
\end{verbatim}


\section{MusicOnHold}
\subsection{Synopsis}
\begin{verbatim}
Play Music On Hold indefinitely
\end{verbatim}
\subsection{Description}
\begin{verbatim}
MusicOnHold(class): Plays hold music specified by class.  If omitted, the default
music source for the channel will be used. Set the default 
class with the SetMusicOnHold() application.
Returns -1 on hangup.
Never returns otherwise.

\end{verbatim}


\section{NBScat}
\subsection{Synopsis}
\begin{verbatim}
Play an NBS local stream
\end{verbatim}
\subsection{Description}
\begin{verbatim}
  NBScat: Executes nbscat to listen to the local NBS stream.
User can exit by pressing any key
.
\end{verbatim}


\section{NoCDR}
\subsection{Synopsis}
\begin{verbatim}
Tell Asterisk to not maintain a CDR for the current call
\end{verbatim}
\subsection{Description}
\begin{verbatim}
  NoCDR(): This application will tell Asterisk not to maintain a CDR for the
current call.

\end{verbatim}


\section{NoOp}
\subsection{Synopsis}
\begin{verbatim}
Do Nothing
\end{verbatim}
\subsection{Description}
\begin{verbatim}
  NoOp(): This applicatiion does nothing. However, it is useful for debugging
purposes. Any text that is provided as arguments to this application can be
viewed at the Asterisk CLI. This method can be used to see the evaluations of
variables or functions without having any effect.
\end{verbatim}


\section{ODBCFinish}
\subsection{Synopsis}
\begin{verbatim}
Clear the resultset of a successful multirow query
\end{verbatim}
\subsection{Description}
\begin{verbatim}
ODBCFinish(<result-id>)
  Clears any remaining rows of the specified resultset

\end{verbatim}


\section{Page}
\subsection{Synopsis}
\begin{verbatim}
Pages phones
\end{verbatim}
\subsection{Description}
\begin{verbatim}
Page(Technology/Resource&Technology2/Resource2[|options])
  Places outbound calls to the given technology / resource and dumps
them into a conference bridge as muted participants.  The original
caller is dumped into the conference as a speaker and the room is
destroyed when the original caller leaves.  Valid options are:
        d - full duplex audio
        q - quiet, do not play beep to caller
        r - record the page into a file (see 'r' for app_meetme)
        s - only dial channel if devicestate says it is not in use

\end{verbatim}


\section{Park}
\subsection{Synopsis}
\begin{verbatim}
Park yourself
\end{verbatim}
\subsection{Description}
\begin{verbatim}
Park():Used to park yourself (typically in combination with a supervised
transfer to know the parking space). This application is always
registered internally and does not need to be explicitly added
into the dialplan, although you should include the 'parkedcalls'
context (or the context specified in features.conf).

If you set the PARKINGEXTEN variable to an extension in your
parking context, park() will park the call on that extension, unless
it already exists. In that case, execution will continue at next
priority.

\end{verbatim}


\section{ParkAndAnnounce}
\subsection{Synopsis}
\begin{verbatim}
Park and Announce
\end{verbatim}
\subsection{Description}
\begin{verbatim}
  ParkAndAnnounce(announce:template|timeout|dial|[return_context]):
Park a call into the parkinglot and announce the call to another channel.

announce template: Colon-separated list of files to announce.  The word PARKED
                   will be replaced by a say_digits of the extension in which
                   the call is parked.
timeout:           Time in seconds before the call returns into the return
                   context.
dial:              The app_dial style resource to call to make the
                   announcement.  Console/dsp calls the console.
return_context:    The goto-style label to jump the call back into after
                   timeout.  Default <priority+1>.

The variable ${PARKEDAT} will contain the parking extension into which the
call was placed.  Use with the Local channel to allow the dialplan to make
use of this information.

\end{verbatim}


\section{ParkedCall}
\subsection{Synopsis}
\begin{verbatim}
Answer a parked call
\end{verbatim}
\subsection{Description}
\begin{verbatim}
ParkedCall(exten):Used to connect to a parked call.  This application is always
registered internally and does not need to be explicitly added
into the dialplan, although you should include the 'parkedcalls'
context.

\end{verbatim}


\section{PauseMonitor}
\subsection{Synopsis}
\begin{verbatim}
Pause monitoring of a channel
\end{verbatim}
\subsection{Description}
\begin{verbatim}
PauseMonitor
Pauses monitoring of a channel until it is re-enabled by a call to UnpauseMonitor.

\end{verbatim}


\section{PauseQueueMember}
\subsection{Synopsis}
\begin{verbatim}
Pauses a queue member
\end{verbatim}
\subsection{Description}
\begin{verbatim}
   PauseQueueMember([queuename]|interface[|options]):
Pauses (blocks calls for) a queue member.
The given interface will be paused in the given queue.  This prevents
any calls from being sent from the queue to the interface until it is
unpaused with UnpauseQueueMember or the manager interface.  If no
queuename is given, the interface is paused in every queue it is a
member of. The application will fail if the interface is not found.
  This application sets the following channel variable upon completion:
     PQMSTATUS      The status of the attempt to pause a queue member as a
                     text string, one of
           PAUSED | NOTFOUND
Example: PauseQueueMember(|SIP/3000)

\end{verbatim}


\section{Pickup}
\subsection{Synopsis}
\begin{verbatim}
Directed Call Pickup
\end{verbatim}
\subsection{Description}
\begin{verbatim}
  Pickup(extension[@context][&extension2@context...]): This application can pickup any ringing channel
that is calling the specified extension. If no context is specified, the current
context will be used. If you use the special string "PICKUPMARK" for the context parameter, for example
10@PICKUPMARK, this application tries to find a channel which has defined a channel variable with the same content
as "extension".
\end{verbatim}


\section{Playback}
\subsection{Synopsis}
\begin{verbatim}
Play a file
\end{verbatim}
\subsection{Description}
\begin{verbatim}
  Playback(filename[&filename2...][|option]):  Plays back given filenames (do not put
extension). Options may also be included following a pipe symbol. The 'skip'
option causes the playback of the message to be skipped if the channel
is not in the 'up' state (i.e. it hasn't been  answered  yet). If 'skip' is 
specified, the application will return immediately should the channel not be
off hook.  Otherwise, unless 'noanswer' is specified, the channel will
be answered before the sound is played. Not all channels support playing
messages while still on hook.
This application sets the following channel variable upon completion:
 PLAYBACKSTATUS    The status of the playback attempt as a text string, one of
               SUCCESS | FAILED

\end{verbatim}


\section{PlayTones}
\subsection{Synopsis}
\begin{verbatim}
Play a tone list
\end{verbatim}
\subsection{Description}
\begin{verbatim}
PlayTones(arg): Plays a tone list. Execution will continue with the next step immediately,
while the tones continue to play.
Arg is either the tone name defined in the indications.conf configuration file, or a directly
specified list of frequencies and durations.
See the sample indications.conf for a description of the specification of a tonelist.

Use the StopPlayTones application to stop the tones playing. 

\end{verbatim}


\section{PrivacyManager}
\subsection{Synopsis}
\begin{verbatim}
Require phone number to be entered, if no CallerID sent
\end{verbatim}
\subsection{Description}
\begin{verbatim}
  PrivacyManager([maxretries[|minlength[|options]]]): If no Caller*ID 
is sent, PrivacyManager answers the channel and asks the caller to
enter their phone number. The caller is given 3 attempts to do so.
The application does nothing if Caller*ID was received on the channel.
  Configuration file privacy.conf contains two variables:
   maxretries  default 3  -maximum number of attempts the caller is allowed 
               to input a callerid.
   minlength   default 10 -minimum allowable digits in the input callerid number.
If you don't want to use the config file and have an i/o operation with
every call, you can also specify maxretries and minlength as application
parameters. Doing so supercedes any values set in privacy.conf.
The application sets the following channel variable upon completion: 
PRIVACYMGRSTATUS  The status of the privacy manager's attempt to collect 
                  a phone number from the user. A text string that is either:
          SUCCESS | FAILED 

\end{verbatim}


\section{Progress}
\subsection{Synopsis}
\begin{verbatim}
Indicate progress
\end{verbatim}
\subsection{Description}
\begin{verbatim}
  Progress(): This application will request that in-band progress information
be provided to the calling channel.

\end{verbatim}


\section{Queue}
\subsection{Synopsis}
\begin{verbatim}
Queue a call for a call queue
\end{verbatim}
\subsection{Description}
\begin{verbatim}
  Queue(queuename[|options[|URL][|announceoverride][|timeout][|AGI][|macro][|gosub]):
Queues an incoming call in a particular call queue as defined in queues.conf.
This application will return to the dialplan if the queue does not exist, or
any of the join options cause the caller to not enter the queue.
The option string may contain zero or more of the following characters:
      'c' -- continue in the dialplan if the callee hangs up.
      'd' -- data-quality (modem) call (minimum delay).
      'h' -- allow callee to hang up by pressing *.
      'H' -- allow caller to hang up by pressing *.
      'n' -- no retries on the timeout; will exit this application and 
             go to the next step.
      'i' -- ignore call forward requests from queue members and do nothing
             when they are requested.
      'r' -- ring instead of playing MOH.
      't' -- allow the called user to transfer the calling user.
      'T' -- allow the calling user to transfer the call.
      'w' -- allow the called user to write the conversation to disk via Monitor.
      'W' -- allow the calling user to write the conversation to disk via Monitor.
  In addition to transferring the call, a call may be parked and then picked
up by another user.
  The optional URL will be sent to the called party if the channel supports
it.
  The optional AGI parameter will setup an AGI script to be executed on the 
calling party's channel once they are connected to a queue member.
  The optional macro parameter will run a macro on the 
calling party's channel once they are connected to a queue member.
  The optional gosub parameter will run a gosub on the 
calling party's channel once they are connected to a queue member.
  The timeout will cause the queue to fail out after a specified number of
seconds, checked between each queues.conf 'timeout' and 'retry' cycle.
  This application sets the following channel variable upon completion:
      QUEUESTATUS    The status of the call as a text string, one of
             TIMEOUT | FULL | JOINEMPTY | LEAVEEMPTY | JOINUNAVAIL | LEAVEUNAVAIL | CONTINUE

\end{verbatim}


\section{QueueLog}
\subsection{Synopsis}
\begin{verbatim}
Writes to the queue_log
\end{verbatim}
\subsection{Description}
\begin{verbatim}
   QueueLog(queuename|uniqueid|agent|event[|additionalinfo]):
Allows you to write your own events into the queue log
Example: QueueLog(101|${UNIQUEID}|${AGENT}|WENTONBREAK|600)

\end{verbatim}


\section{Read}
\subsection{Synopsis}
\begin{verbatim}
Read a variable
\end{verbatim}
\subsection{Description}
\begin{verbatim}
  Read(variable[|filename[&filename2...]][|maxdigits][|option][|attempts][|timeout])

Reads a #-terminated string of digits a certain number of times from the
user in to the given variable.
  filename   -- file(s) to play before reading digits or tone with option i
  maxdigits  -- maximum acceptable number of digits. Stops reading after
                maxdigits have been entered (without requiring the user to
                press the '#' key).
                Defaults to 0 - no limit - wait for the user press the '#' key.
                Any value below 0 means the same. Max accepted value is 255.
  option     -- options are 's' , 'i', 'n'
                's' to return immediately if the line is not up,
                'i' to play  filename as an indication tone from your indications.conf
                'n' to read digits even if the line is not up.
  attempts   -- if greater than 1, that many attempts will be made in the 
                event no data is entered.
  timeout    -- The number of seconds to wait for a digit response. If greater
                than 0, that value will override the default timeout. Can be floating point.

Read should disconnect if the function fails or errors out.

\end{verbatim}


\section{ReadFile}
\subsection{Synopsis}
\begin{verbatim}
ReadFile(varname=file,length)
\end{verbatim}
\subsection{Description}
\begin{verbatim}
ReadFile(varname=file,length)
  Varname - Result stored here.
  File - The name of the file to read.
  Length - Maximum number of characters to capture.

\end{verbatim}


\section{Record}
\subsection{Synopsis}
\begin{verbatim}
Record to a file
\end{verbatim}
\subsection{Description}
\begin{verbatim}
  Record(filename.format|silence[|maxduration][|options])

Records from the channel into a given filename. If the file exists it will
be overwritten.
- 'format' is the format of the file type to be recorded (wav, gsm, etc).
- 'silence' is the number of seconds of silence to allow before returning.
- 'maxduration' is the maximum recording duration in seconds. If missing
or 0 there is no maximum.
- 'options' may contain any of the following letters:
     'a' : append to existing recording rather than replacing
     'n' : do not answer, but record anyway if line not yet answered
     'q' : quiet (do not play a beep tone)
     's' : skip recording if the line is not yet answered
     't' : use alternate '*' terminator key (DTMF) instead of default '#'
     'x' : ignore all terminator keys (DTMF) and keep recording until hangup

If filename contains '%d', these characters will be replaced with a number
incremented by one each time the file is recorded. A channel variable
named RECORDED_FILE will also be set, which contains the final filemname.

Use 'core show file formats' to see the available formats on your system

User can press '#' to terminate the recording and continue to the next priority.

If the user should hangup during a recording, all data will be lost and the
application will teminate. 

\end{verbatim}


\section{RemoveQueueMember}
\subsection{Synopsis}
\begin{verbatim}
Dynamically removes queue members
\end{verbatim}
\subsection{Description}
\begin{verbatim}
   RemoveQueueMember(queuename[|interface[|options]]):
Dynamically removes interface to an existing queue
If the interface is NOT in the queue it will return an error.
  This application sets the following channel variable upon completion:
     RQMSTATUS      The status of the attempt to remove a queue member as a
                     text string, one of
           REMOVED | NOTINQUEUE | NOSUCHQUEUE 
Example: RemoveQueueMember(techsupport|SIP/3000)

\end{verbatim}


\section{ResetCDR}
\subsection{Synopsis}
\begin{verbatim}
Resets the Call Data Record
\end{verbatim}
\subsection{Description}
\begin{verbatim}
  ResetCDR([options]):  This application causes the Call Data Record to be
reset.
  Options:
    w -- Store the current CDR record before resetting it.
    a -- Store any stacked records.
    v -- Save CDR variables.

\end{verbatim}


\section{RetryDial}
\subsection{Synopsis}
\begin{verbatim}
Place a call, retrying on failure allowing optional exit extension.
\end{verbatim}
\subsection{Description}
\begin{verbatim}
  RetryDial(announce|sleep|retries|dialargs): This application will attempt to
place a call using the normal Dial application. If no channel can be reached,
the 'announce' file will be played. Then, it will wait 'sleep' number of
seconds before retying the call. After 'retires' number of attempts, the
calling channel will continue at the next priority in the dialplan. If the
'retries' setting is set to 0, this application will retry endlessly.
  While waiting to retry a call, a 1 digit extension may be dialed. If that
extension exists in either the context defined in ${EXITCONTEXT} or the current
one, The call will jump to that extension immediately.
  The 'dialargs' are specified in the same format that arguments are provided
to the Dial application.

\end{verbatim}


\section{Return}
\subsection{Synopsis}
\begin{verbatim}
Return from gosub routine
\end{verbatim}
\subsection{Description}
\begin{verbatim}
Return([return-value])
  Jumps to the last label on the stack, removing it.  The return value, if
any, is saved in the channel variable GOSUB_RETVAL.

\end{verbatim}


\section{Ringing}
\subsection{Synopsis}
\begin{verbatim}
Indicate ringing tone
\end{verbatim}
\subsection{Description}
\begin{verbatim}
  Ringing(): This application will request that the channel indicate a ringing
tone to the user.

\end{verbatim}


\section{Rpt}
\subsection{Synopsis}
\begin{verbatim}
Radio Repeater/Remote Base Control System
\end{verbatim}
\subsection{Description}
\begin{verbatim}
  Rpt(nodename[|options]):  Radio Remote Link or Remote Base Link Endpoint Process.

    Not specifying an option puts it in normal endpoint mode (where source
    IP and nodename are verified).

    Options are as follows:

        X - Normal endpoint mode WITHOUT security check. Only specify
            this if you have checked security already (like with an IAX2
            user/password or something).

        Rannounce-string[|timeout[|timeout-destination]] - Amateur Radio
            Reverse Autopatch. Caller is put on hold, and announcement (as
            specified by the 'announce-string') is played on radio system.
            Users of radio system can access autopatch, dial specified
            code, and pick up call. Announce-string is list of names of
            recordings, or "PARKED" to substitute code for un-parking,
            or "NODE" to substitute node number.

        P - Phone Control mode. This allows a regular phone user to have
            full control and audio access to the radio system. For the
            user to have DTMF control, the 'phone_functions' parameter
            must be specified for the node in 'rpt.conf'. An additional
            function (cop,6) must be listed so that PTT control is available.

        D - Dumb Phone Control mode. This allows a regular phone user to
            have full control and audio access to the radio system. In this
            mode, the PTT is activated for the entire length of the call.
            For the user to have DTMF control (not generally recomended in
            this mode), the 'dphone_functions' parameter must be specified
            for the node in 'rpt.conf'. Otherwise no DTMF control will be
            available to the phone user.


\end{verbatim}


\section{SayAlpha}
\subsection{Synopsis}
\begin{verbatim}
Say Alpha
\end{verbatim}
\subsection{Description}
\begin{verbatim}
  SayAlpha(string): This application will play the sounds that correspond to
the letters of the given string.

\end{verbatim}


\section{SayDigits}
\subsection{Synopsis}
\begin{verbatim}
Say Digits
\end{verbatim}
\subsection{Description}
\begin{verbatim}
  SayDigits(digits): This application will play the sounds that correspond
to the digits of the given number. This will use the language that is currently
set for the channel. See the LANGUAGE function for more information on setting
the language for the channel.

\end{verbatim}


\section{SayNumber}
\subsection{Synopsis}
\begin{verbatim}
Say Number
\end{verbatim}
\subsection{Description}
\begin{verbatim}
  SayNumber(digits[,gender]): This application will play the sounds that
correspond to the given number. Optionally, a gender may be specified.
This will use the language that is currently set for the channel. See the
LANGUAGE function for more information on setting the language for the channel.

\end{verbatim}


\section{SayPhonetic}
\subsection{Synopsis}
\begin{verbatim}
Say Phonetic
\end{verbatim}
\subsection{Description}
\begin{verbatim}
  SayPhonetic(string): This application will play the sounds from the phonetic
alphabet that correspond to the letters in the given string.

\end{verbatim}


\section{SayUnixTime}
\subsection{Synopsis}
\begin{verbatim}
Says a specified time in a custom format
\end{verbatim}
\subsection{Description}
\begin{verbatim}
SayUnixTime([unixtime][|[timezone][|format]])
  unixtime: time, in seconds since Jan 1, 1970.  May be negative.
              defaults to now.
  timezone: timezone, see /usr/share/zoneinfo for a list.
              defaults to machine default.
  format:   a format the time is to be said in.  See voicemail.conf.
              defaults to "ABdY 'digits/at' IMp"

\end{verbatim}


\section{SendDTMF}
\subsection{Synopsis}
\begin{verbatim}
Sends arbitrary DTMF digits
\end{verbatim}
\subsection{Description}
\begin{verbatim}
 SendDTMF(digits[|timeout_ms]): Sends DTMF digits on a channel. 
 Accepted digits: 0-9, *#abcd, w (.5s pause)
 The application will either pass the assigned digits or terminate if it
 encounters an error.

\end{verbatim}


\section{SendImage}
\subsection{Synopsis}
\begin{verbatim}
Send an image file
\end{verbatim}
\subsection{Description}
\begin{verbatim}
  SendImage(filename): Sends an image on a channel. 
If the channel supports image transport but the image send
fails, the channel will be hung up. Otherwise, the dialplan
continues execution.
This application sets the following channel variable upon completion:
	SENDIMAGESTATUS		The status is the result of the attempt as a text string, one of
		OK | NOSUPPORT 

\end{verbatim}


\section{SendText}
\subsection{Synopsis}
\begin{verbatim}
Send a Text Message
\end{verbatim}
\subsection{Description}
\begin{verbatim}
  SendText(text[|options]): Sends text to current channel (callee).
Result of transmission will be stored in the SENDTEXTSTATUS
channel variable:
      SUCCESS      Transmission succeeded
      FAILURE      Transmission failed
      UNSUPPORTED  Text transmission not supported by channel

At this moment, text is supposed to be 7 bit ASCII in most channels.

\end{verbatim}


\section{SendURL}
\subsection{Synopsis}
\begin{verbatim}
Send a URL
\end{verbatim}
\subsection{Description}
\begin{verbatim}
  SendURL(URL[|option]): Requests client go to URL (IAX2) or sends the 
URL to the client (other channels).
Result is returned in the SENDURLSTATUS channel variable:
    SUCCESS       URL successfully sent to client
    FAILURE       Failed to send URL
    NOLOAD        Client failed to load URL (wait enabled)
    UNSUPPORTED   Channel does not support URL transport

If the option 'wait' is specified, execution will wait for an
acknowledgement that the URL has been loaded before continuing

SendURL continues normally if the URL was sent correctly or if the channel
does not support HTML transport.  Otherwise, the channel is hung up.

\end{verbatim}


\section{Set}
\subsection{Synopsis}
\begin{verbatim}
Set channel variable(s) or function value(s)
\end{verbatim}
\subsection{Description}
\begin{verbatim}
  Set(name1=value1|name2=value2|..[|options])
This function can be used to set the value of channel variables or dialplan
functions. It will accept up to 24 name/value pairs. When setting variables,
if the variable name is prefixed with _, the variable will be inherited into
channels created from the current channel. If the variable name is prefixed
with __, the variable will be inherited into channels created from the current
channel and all children channels.
  Options:
    g - Set variable globally instead of on the channel
        (applies only to variables, not functions)

\end{verbatim}


\section{SetAMAFlags}
\subsection{Synopsis}
\begin{verbatim}
Set the AMA Flags
\end{verbatim}
\subsection{Description}
\begin{verbatim}
  SetAMAFlags([flag]): This application will set the channel's AMA Flags for
  billing purposes.

\end{verbatim}


\section{SetCallerPres}
\subsection{Synopsis}
\begin{verbatim}
Set CallerID Presentation
\end{verbatim}
\subsection{Description}
\begin{verbatim}
  SetCallerPres(presentation): Set Caller*ID presentation on a call.
  Valid presentations are:

      allowed_not_screened    : Presentation Allowed, Not Screened
      allowed_passed_screen   : Presentation Allowed, Passed Screen
      allowed_failed_screen   : Presentation Allowed, Failed Screen
      allowed                 : Presentation Allowed, Network Number
      prohib_not_screened     : Presentation Prohibited, Not Screened
      prohib_passed_screen    : Presentation Prohibited, Passed Screen
      prohib_failed_screen    : Presentation Prohibited, Failed Screen
      prohib                  : Presentation Prohibited, Network Number
      unavailable             : Number Unavailable


\end{verbatim}


\section{SetMusicOnHold}
\subsection{Synopsis}
\begin{verbatim}
Set default Music On Hold class
\end{verbatim}
\subsection{Description}
\begin{verbatim}
SetMusicOnHold(class): Sets the default class for music on hold for a given channel.  When
music on hold is activated, this class will be used to select which
music is played.

\end{verbatim}


\section{SIPAddHeader}
\subsection{Synopsis}
\begin{verbatim}
Add a SIP header to the outbound call
\end{verbatim}
\subsection{Description}
\begin{verbatim}
  SIPAddHeader(Header: Content)
Adds a header to a SIP call placed with DIAL.
Remember to user the X-header if you are adding non-standard SIP
headers, like "X-Asterisk-Accountcode:". Use this with care.
Adding the wrong headers may jeopardize the SIP dialog.
Always returns 0

\end{verbatim}


\section{SIPDtmfMode}
\subsection{Synopsis}
\begin{verbatim}
Change the dtmfmode for a SIP call
\end{verbatim}
\subsection{Description}
\begin{verbatim}
SIPDtmfMode(inband|info|rfc2833): Changes the dtmfmode for a SIP call

\end{verbatim}


\section{Skel}
\subsection{Synopsis}
\begin{verbatim}
Skeleton application.
\end{verbatim}
\subsection{Description}
\begin{verbatim}
This application is a template to build other applications from.
 It shows you the basic structure to create your own Asterisk applications.

\end{verbatim}


\section{SLAStation}
\subsection{Synopsis}
\begin{verbatim}
Shared Line Appearance Station
\end{verbatim}
\subsection{Description}
\begin{verbatim}
  SLAStation(station):
This application should be executed by an SLA station.  The argument depends
on how the call was initiated.  If the phone was just taken off hook, then
the argument "station" should be just the station name.  If the call was
initiated by pressing a line key, then the station name should be preceded
by an underscore and the trunk name associated with that line button.
For example: "station1_line1".  On exit, this application will set the variable SLASTATION_STATUS to
one of the following values:
    FAILURE | CONGESTION | SUCCESS

\end{verbatim}


\section{SLATrunk}
\subsection{Synopsis}
\begin{verbatim}
Shared Line Appearance Trunk
\end{verbatim}
\subsection{Description}
\begin{verbatim}
  SLATrunk(trunk):
This application should be executed by an SLA trunk on an inbound call.
The channel calling this application should correspond to the SLA trunk
with the name "trunk" that is being passed as an argument.
  On exit, this application will set the variable SLATRUNK_STATUS to
one of the following values:
   FAILURE | SUCCESS | UNANSWERED | RINGTIMEOUT

\end{verbatim}


\section{SMS}
\subsection{Synopsis}
\begin{verbatim}
Communicates with SMS service centres and SMS capable analogue phones
\end{verbatim}
\subsection{Description}
\begin{verbatim}
  SMS(name|[a][s][t][p(d)][r][o]|addr|body):
SMS handles exchange of SMS data with a call to/from SMS capable
phone or SMS PSTN service center. Can send and/or receive SMS messages.
Works to ETSI ES 201 912; compatible with BT SMS PSTN service in UK
and Telecom Italia in Italy.
Typical usage is to use to handle calls from the SMS service centre CLI,
or to set up a call using 'outgoing' or manager interface to connect
service centre to SMS()
name is the name of the queue used in /var/spool/asterisk/sms
Arguments:
 a: answer, i.e. send initial FSK packet.
 s: act as service centre talking to a phone.
 t: use protocol 2 (default used is protocol 1).
 p(N): set the initial delay to N ms (default is 300).
addr and body are a deprecated format to send messages out.
 s: set the Status Report Request (SRR) bit.
 o: the body should be coded as octets not 7-bit symbols.
Messages are processed as per text file message queues.
smsq (a separate software) is a command to generate message
queues and send messages.
NOTE: the protocol has tight delay bounds. Please use short frames
and disable/keep short the jitter buffer on the ATA to make sure that
respones (ACK etc.) are received in time.

\end{verbatim}


\section{SoftHangup}
\subsection{Synopsis}
\begin{verbatim}
Soft Hangup Application
\end{verbatim}
\subsection{Description}
\begin{verbatim}
  SoftHangup(Technology/resource|options)
Hangs up the requested channel.  If there are no channels to hangup,
the application will report it.
- 'options' may contain the following letter:
     'a' : hang up all channels on a specified device instead of a single resource

\end{verbatim}


\section{SpeechActivateGrammar}
\subsection{Synopsis}
\begin{verbatim}
Activate a Grammar
\end{verbatim}
\subsection{Description}
\begin{verbatim}
SpeechActivateGrammar(Grammar Name)
This activates the specified grammar to be recognized by the engine. A grammar tells the speech recognition engine what to recognize, 
and how to portray it back to you in the dialplan. The grammar name is the only argument to this application.

\end{verbatim}


\section{SpeechBackground}
\subsection{Synopsis}
\begin{verbatim}
Play a sound file and wait for speech to be recognized
\end{verbatim}
\subsection{Description}
\begin{verbatim}
SpeechBackground(Sound File|Timeout)
This application plays a sound file and waits for the person to speak. Once they start speaking playback of the file stops, and silence is heard.
Once they stop talking the processing sound is played to indicate the speech recognition engine is working.
Once results are available the application returns and results (score and text) are available using dialplan functions.
The first text and score are ${SPEECH_TEXT(0)} AND ${SPEECH_SCORE(0)} while the second are ${SPEECH_TEXT(1)} and ${SPEECH_SCORE(1)}.
The first argument is the sound file and the second is the timeout integer in seconds. Note the timeout will only start once the sound file has stopped playing.

\end{verbatim}


\section{SpeechCreate}
\subsection{Synopsis}
\begin{verbatim}
Create a Speech Structure
\end{verbatim}
\subsection{Description}
\begin{verbatim}
SpeechCreate(engine name)
This application creates information to be used by all the other applications. It must be called before doing any speech recognition activities such as activating a grammar.
It takes the engine name to use as the argument, if not specified the default engine will be used.

\end{verbatim}


\section{SpeechDeactivateGrammar}
\subsection{Synopsis}
\begin{verbatim}
Deactivate a Grammar
\end{verbatim}
\subsection{Description}
\begin{verbatim}
SpeechDeactivateGrammar(Grammar Name)
This deactivates the specified grammar so that it is no longer recognized. The only argument is the grammar name to deactivate.

\end{verbatim}


\section{SpeechDestroy}
\subsection{Synopsis}
\begin{verbatim}
End speech recognition
\end{verbatim}
\subsection{Description}
\begin{verbatim}
SpeechDestroy()
This destroys the information used by all the other speech recognition applications.
If you call this application but end up wanting to recognize more speech, you must call SpeechCreate
again before calling any other application. It takes no arguments.

\end{verbatim}


\section{SpeechLoadGrammar}
\subsection{Synopsis}
\begin{verbatim}
Load a Grammar
\end{verbatim}
\subsection{Description}
\begin{verbatim}
SpeechLoadGrammar(Grammar Name|Path)
Load a grammar only on the channel, not globally.
It takes the grammar name as first argument and path as second.

\end{verbatim}


\section{SpeechProcessingSound}
\subsection{Synopsis}
\begin{verbatim}
Change background processing sound
\end{verbatim}
\subsection{Description}
\begin{verbatim}
SpeechProcessingSound(Sound File)
This changes the processing sound that SpeechBackground plays back when the speech recognition engine is processing and working to get results.
It takes the sound file as the only argument.

\end{verbatim}


\section{SpeechStart}
\subsection{Synopsis}
\begin{verbatim}
Start recognizing voice in the audio stream
\end{verbatim}
\subsection{Description}
\begin{verbatim}
SpeechStart()
Tell the speech recognition engine that it should start trying to get results from audio being fed to it. This has no arguments.

\end{verbatim}


\section{SpeechUnloadGrammar}
\subsection{Synopsis}
\begin{verbatim}
Unload a Grammar
\end{verbatim}
\subsection{Description}
\begin{verbatim}
SpeechUnloadGrammar(Grammar Name)
Unload a grammar. It takes the grammar name as the only argument.

\end{verbatim}


\section{StackPop}
\subsection{Synopsis}
\begin{verbatim}
Remove one address from gosub stack
\end{verbatim}
\subsection{Description}
\begin{verbatim}
StackPop()
  Removes last label on the stack, discarding it.

\end{verbatim}


\section{StartMusicOnHold}
\subsection{Synopsis}
\begin{verbatim}
Play Music On Hold
\end{verbatim}
\subsection{Description}
\begin{verbatim}
StartMusicOnHold(class): Starts playing music on hold, uses default music class for channel.
Starts playing music specified by class.  If omitted, the default
music source for the channel will be used.  Always returns 0.

\end{verbatim}


\section{StopMixMonitor}
\subsection{Synopsis}
\begin{verbatim}
Stop recording a call through MixMonitor
\end{verbatim}
\subsection{Description}
\begin{verbatim}
  StopMixMonitor()

Stops the audio recording that was started with a call to MixMonitor()
on the current channel.

\end{verbatim}


\section{StopMonitor}
\subsection{Synopsis}
\begin{verbatim}
Stop monitoring a channel
\end{verbatim}
\subsection{Description}
\begin{verbatim}
StopMonitor
Stops monitoring a channel. Has no effect if the channel is not monitored

\end{verbatim}


\section{StopMusicOnHold}
\subsection{Synopsis}
\begin{verbatim}
Stop Playing Music On Hold
\end{verbatim}
\subsection{Description}
\begin{verbatim}
StopMusicOnHold: Stops playing music on hold.

\end{verbatim}


\section{StopPlayTones}
\subsection{Synopsis}
\begin{verbatim}
Stop playing a tone list
\end{verbatim}
\subsection{Description}
\begin{verbatim}
Stop playing a tone list
\end{verbatim}


\section{System}
\subsection{Synopsis}
\begin{verbatim}
Execute a system command
\end{verbatim}
\subsection{Description}
\begin{verbatim}
  System(command): Executes a command  by  using  system(). If the command
fails, the console should report a fallthrough. 
Result of execution is returned in the SYSTEMSTATUS channel variable:
   FAILURE	Could not execute the specified command
   SUCCESS	Specified command successfully executed

\end{verbatim}


\section{TestClient}
\subsection{Synopsis}
\begin{verbatim}
Execute Interface Test Client
\end{verbatim}
\subsection{Description}
\begin{verbatim}
TestClient(testid): Executes test client with given testid.
Results stored in /var/log/asterisk/testreports/<testid>-client.txt
\end{verbatim}


\section{TestServer}
\subsection{Synopsis}
\begin{verbatim}
Execute Interface Test Server
\end{verbatim}
\subsection{Description}
\begin{verbatim}
TestServer(): Perform test server function and write call report.
Results stored in /var/log/asterisk/testreports/<testid>-server.txt
\end{verbatim}


\section{Transfer}
\subsection{Synopsis}
\begin{verbatim}
Transfer caller to remote extension
\end{verbatim}
\subsection{Description}
\begin{verbatim}
  Transfer([Tech/]dest[|options]):  Requests the remote caller be transferred
to a given destination. If TECH (SIP, IAX2, LOCAL etc) is used, only
an incoming call with the same channel technology will be transfered.
Note that for SIP, if you transfer before call is setup, a 302 redirect
SIP message will be returned to the caller.

The result of the application will be reported in the TRANSFERSTATUS
channel variable:
       SUCCESS      Transfer succeeded
       FAILURE      Transfer failed
       UNSUPPORTED  Transfer unsupported by channel driver

\end{verbatim}


\section{TryExec}
\subsection{Synopsis}
\begin{verbatim}
Executes dialplan application, always returning
\end{verbatim}
\subsection{Description}
\begin{verbatim}
Usage: TryExec(appname(arguments))
  Allows an arbitrary application to be invoked even when not
hardcoded into the dialplan. To invoke external applications
see the application System.  Always returns to the dialplan.
The channel variable TRYSTATUS will be set to:
    SUCCESS   if the application returned zero
    FAILED    if the application returned non-zero
    NOAPP     if the application was not found or was not specified

\end{verbatim}


\section{TrySystem}
\subsection{Synopsis}
\begin{verbatim}
Try executing a system command
\end{verbatim}
\subsection{Description}
\begin{verbatim}
  TrySystem(command): Executes a command  by  using  system().
on any situation.
Result of execution is returned in the SYSTEMSTATUS channel variable:
   FAILURE	Could not execute the specified command
   SUCCESS	Specified command successfully executed
   APPERROR	Specified command successfully executed, but returned error code

\end{verbatim}


\section{UnpauseMonitor}
\subsection{Synopsis}
\begin{verbatim}
Unpause monitoring of a channel
\end{verbatim}
\subsection{Description}
\begin{verbatim}
UnpauseMonitor
Unpauses monitoring of a channel on which monitoring had
previously been paused with PauseMonitor.

\end{verbatim}


\section{UnpauseQueueMember}
\subsection{Synopsis}
\begin{verbatim}
Unpauses a queue member
\end{verbatim}
\subsection{Description}
\begin{verbatim}
   UnpauseQueueMember([queuename]|interface[|options]):
Unpauses (resumes calls to) a queue member.
This is the counterpart to PauseQueueMember and operates exactly the
same way, except it unpauses instead of pausing the given interface.
  This application sets the following channel variable upon completion:
     UPQMSTATUS       The status of the attempt to unpause a queue 
                      member as a text string, one of
            UNPAUSED | NOTFOUND
Example: UnpauseQueueMember(|SIP/3000)

\end{verbatim}


\section{UserEvent}
\subsection{Synopsis}
\begin{verbatim}
Send an arbitrary event to the manager interface
\end{verbatim}
\subsection{Description}
\begin{verbatim}
  UserEvent(eventname[|body]): Sends an arbitrary event to the manager
interface, with an optional body representing additional arguments.  The
body may be specified as a | delimeted list of headers. Each additional
argument will be placed on a new line in the event. The format of the
event will be:
    Event: UserEvent
    UserEvent: <specified event name>
    [body]
If no body is specified, only Event and UserEvent headers will be present.

\end{verbatim}


\section{Verbose}
\subsection{Synopsis}
\begin{verbatim}
Send arbitrary text to verbose output
\end{verbatim}
\subsection{Description}
\begin{verbatim}
Verbose([<level>|]<message>)
  level must be an integer value.  If not specified, defaults to 0.

\end{verbatim}


\section{VMAuthenticate}
\subsection{Synopsis}
\begin{verbatim}
Authenticate with Voicemail passwords
\end{verbatim}
\subsection{Description}
\begin{verbatim}
  VMAuthenticate([mailbox][@context][|options]): This application behaves the
same way as the Authenticate application, but the passwords are taken from
voicemail.conf.
  If the mailbox is specified, only that mailbox's password will be considered
valid. If the mailbox is not specified, the channel variable AUTH_MAILBOX will
be set with the authenticated mailbox.

  Options:
    s - Skip playing the initial prompts.

\end{verbatim}


\section{VoiceMail}
\subsection{Synopsis}
\begin{verbatim}
Leave a Voicemail message
\end{verbatim}
\subsection{Description}
\begin{verbatim}
  VoiceMail(mailbox[@context][&mailbox[@context]][...][|options]): This
application allows the calling party to leave a message for the specified
list of mailboxes. When multiple mailboxes are specified, the greeting will
be taken from the first mailbox specified. Dialplan execution will stop if the
specified mailbox does not exist.
  The Voicemail application will exit if any of the following DTMF digits are
received:
    0 - Jump to the 'o' extension in the current dialplan context.
    * - Jump to the 'a' extension in the current dialplan context.
  This application will set the following channel variable upon completion:
    VMSTATUS - This indicates the status of the execution of the VoiceMail
               application. The possible values are:
               SUCCESS | USEREXIT | FAILED

  Options:
    b    - Play the 'busy' greeting to the calling party.
    g(#) - Use the specified amount of gain when recording the voicemail
           message. The units are whole-number decibels (dB).
    s    - Skip the playback of instructions for leaving a message to the
           calling party.
    u    - Play the 'unavailable greeting.

\end{verbatim}


\section{VoiceMailMain}
\subsection{Synopsis}
\begin{verbatim}
Check Voicemail messages
\end{verbatim}
\subsection{Description}
\begin{verbatim}
  VoiceMailMain([mailbox][@context][|options]): This application allows the
calling party to check voicemail messages. A specific mailbox, and optional
corresponding context, may be specified. If a mailbox is not provided, the
calling party will be prompted to enter one. If a context is not specified,
the 'default' context will be used.

  Options:
    p    - Consider the mailbox parameter as a prefix to the mailbox that
           is entered by the caller.
    g(#) - Use the specified amount of gain when recording a voicemail
           message. The units are whole-number decibels (dB).
    s    - Skip checking the passcode for the mailbox.
    a(#) - Skip folder prompt and go directly to folder specified.
           Defaults to INBOX

\end{verbatim}


\section{Wait}
\subsection{Synopsis}
\begin{verbatim}
Waits for some time
\end{verbatim}
\subsection{Description}
\begin{verbatim}
  Wait(seconds): This application waits for a specified number of seconds.
Then, dialplan execution will continue at the next priority.
  Note that the seconds can be passed with fractions of a second. For example,
'1.5' will ask the application to wait for 1.5 seconds.

\end{verbatim}


\section{WaitExten}
\subsection{Synopsis}
\begin{verbatim}
Waits for an extension to be entered
\end{verbatim}
\subsection{Description}
\begin{verbatim}
  WaitExten([seconds][|options]): This application waits for the user to enter
a new extension for a specified number of seconds.
  Note that the seconds can be passed with fractions of a second. For example,
'1.5' will ask the application to wait for 1.5 seconds.
  Options:
    m[(x)] - Provide music on hold to the caller while waiting for an extension.
               Optionally, specify the class for music on hold within parenthesis.

\end{verbatim}


\section{WaitForRing}
\subsection{Synopsis}
\begin{verbatim}
Wait for Ring Application
\end{verbatim}
\subsection{Description}
\begin{verbatim}
  WaitForRing(timeout)
Returns 0 after waiting at least timeout seconds. and
only after the next ring has completed.  Returns 0 on
success or -1 on hangup

\end{verbatim}


\section{WaitForSilence}
\subsection{Synopsis}
\begin{verbatim}
Waits for a specified amount of silence
\end{verbatim}
\subsection{Description}
\begin{verbatim}
  WaitForSilence(silencerequired[|iterations][|timeout]) 
Wait for Silence: Waits for up to 'silencerequired' 
milliseconds of silence, 'iterations' times or once if omitted.
An optional timeout specified the number of seconds to return
after, even if we do not receive the specified amount of silence.
Use 'timeout' with caution, as it may defeat the purpose of this
application, which is to wait indefinitely until silence is detected
on the line.  This is particularly useful for reverse-911-type
call broadcast applications where you need to wait for an answering
machine to complete its spiel before playing a message.
The timeout parameter is specified only to avoid an infinite loop in
cases where silence is never achieved.  Typically you will want to
include two or more calls to WaitForSilence when dealing with an answering
machine; first waiting for the spiel to finish, then waiting for the beep, etc.

Examples:
  - WaitForSilence(500|2) will wait for 1/2 second of silence, twice
  - WaitForSilence(1000) will wait for 1 second of silence, once
  - WaitForSilence(300|3|10) will wait for 300ms silence, 3 times,
     and returns after 10 sec, even if silence is not detected

Sets the channel variable WAITSTATUS with to one of these values:
SILENCE - if exited with silence detected
TIMEOUT - if exited without silence detected after timeout

\end{verbatim}


\section{WaitMusicOnHold}
\subsection{Synopsis}
\begin{verbatim}
Wait, playing Music On Hold
\end{verbatim}
\subsection{Description}
\begin{verbatim}
WaitMusicOnHold(delay): Plays hold music specified number of seconds.  Returns 0 when
done, or -1 on hangup.  If no hold music is available, the delay will
still occur with no sound.

\end{verbatim}


\section{While}
\subsection{Synopsis}
\begin{verbatim}
Start a while loop
\end{verbatim}
\subsection{Description}
\begin{verbatim}
Usage:  While(<expr>)
Start a While Loop.  Execution will return to this point when
EndWhile is called until expr is no longer true.

\end{verbatim}


\section{Zapateller}
\subsection{Synopsis}
\begin{verbatim}
Block telemarketers with SIT
\end{verbatim}
\subsection{Description}
\begin{verbatim}
  Zapateller(options):  Generates special information tone to block
telemarketers from calling you.  Options is a pipe-delimited list of
options.  The following options are available:
'answer' causes the line to be answered before playing the tone,
'nocallerid' causes Zapateller to only play the tone if there
is no callerid information available.  Options should be separated by |
characters

\end{verbatim}


\section{ZapBarge}
\subsection{Synopsis}
\begin{verbatim}
Barge in (monitor) Zap channel
\end{verbatim}
\subsection{Description}
\begin{verbatim}
  ZapBarge([channel]): Barges in on a specified zap
channel or prompts if one is not specified.  Returns
-1 when caller user hangs up and is independent of the
state of the channel being monitored.
\end{verbatim}


\section{ZapRAS}
\subsection{Synopsis}
\begin{verbatim}
Executes Zaptel ISDN RAS application
\end{verbatim}
\subsection{Description}
\begin{verbatim}
  ZapRAS(args): Executes a RAS server using pppd on the given channel.
The channel must be a clear channel (i.e. PRI source) and a Zaptel
channel to be able to use this function (No modem emulation is included).
Your pppd must be patched to be zaptel aware. Arguments should be
separated by | characters.

\end{verbatim}


\section{ZapScan}
\subsection{Synopsis}
\begin{verbatim}
Scan Zap channels to monitor calls
\end{verbatim}
\subsection{Description}
\begin{verbatim}
  ZapScan([group]) allows a call center manager to monitor Zap channels in
a convenient way.  Use '#' to select the next channel and use '*' to exit
Limit scanning to a channel GROUP by setting the option group argument.

\end{verbatim}


