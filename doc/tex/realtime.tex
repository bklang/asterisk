\subsubsection{Introduction}

The Asterisk Realtime Architecture is a new set of drivers and
functions implemented in Asterisk.

The benefits of this architecture are many, both from a code management
standpoint and from an installation perspective.

The ARA is designed to be independent of storage. Currently, most
drivers are based on SQL, but the architecture should be able to handle
other storage methods in the future, like LDAP.

The main benefit comes in the database support. In Asterisk v1.0 some
functions supported MySQL database, some PostgreSQL and other ODBC.
With the ARA, we have a unified database interface internally in Asterisk,
so if one function supports database integration, all databases that has a
realtime driver will be supported in that function.

Currently there are three realtime database drivers:

\begin{itemize}
  \item ODBC: Support for UnixODBC, integrated into Asterisk
        The UnixODBC subsystem supports many different databases,
        please check \url{www.unixodbc.org} for more information.
  \item MySQL: Found in the asterisk-addons subversion repository on \url{svn.digium.com}
  \item PostgreSQL: Native support for Postgres, integrated into Asterisk
\end{itemize}

\subsubsection{Two modes: Static and Realtime}

The ARA realtime mode is used to dynamically load and update objects.
This mode is used in the SIP and IAX2 channels, as well as in the voicemail
system. For SIP and IAX2 this is similar to the v1.0 MYSQL\_FRIENDS
functionality. With the ARA, we now support many more databases for
dynamic configuration of phones.

The ARA static mode is used to load configuration files. For the Asterisk
modules that read configurations, there's no difference between a static
file in the file system, like extensions.conf, and a configuration loaded
from a database.

You just have to always make sure the var\_metric values are properly set and
ordered as you expect in your database server if you're using the static mode
with ARA (either sequentially or with the same var\_metric value for everybody).

If you have an option that depends on another one in a given configuration
file (i.e, 'musiconhold' depending on 'agent' from agents.conf) but their
var\_metric are not sequential you'll probably get default values being assigned for
those options instead of the desired ones. You can still use the same
var\_metric for all entries in your DB, just make sure the entries
are recorded in an order that does not break the option dependency.

That doesn't happen when you use a static file in the file system. Although
this might be interpreted as a bug or limitation, it is not.

\subsubsection{Realtime SIP friends}

The SIP realtime objects are users and peers that are loaded in memory
when needed, then deleted. This means that Asterisk currently can't handle
voicemail notification and NAT keepalives for these peers. Other than that,
most of the functionality works the same way for realtime friends as for
the ones in static configuration.

With caching, the device stays in memory for a specified time. More
information about this is to be found in the sip.conf sample file.

If you specify a separate family called "sipregs" SIP registration
data will be stored in that table and not in the "sippeers" table.

\subsubsection{Realtime H.323 friends}

Like SIP realtime friends, H.323 friends also can be configured using
dynamic realtime objects.

\subsubsection{New function in the dial plan: The Realtime Switch}

The realtime switch is more than a port of functionality in v1.0 to the
new architecture, this is a new feature of Asterisk based on the
ARA. The realtime switch lets your Asterisk server do database lookups
of extensions in realtime from your dial plan. You can have many Asterisk
servers sharing a dynamically updated dial plan in real time with this
solution.

Note that this switch does NOT support Caller ID matching, only
extension name or pattern matching.

\subsubsection{Capabilities}

The realtime Architecture lets you store all of your configuration in
databases and reload it whenever you want. You can force a reload over
the AMI, Asterisk Manager Interface or by calling Asterisk from a
shell script with

  asterisk -rx "reload"

You may also dynamically add SIP and IAX devices and extensions
and making them available without a reload, by using the realtime
objects and the realtime switch.

\subsubsection{Configuration in extconfig.conf}

You configure the ARA in extconfig.conf (yes, it's a strange name, but
is was defined in the early days of the realtime architecture and kind
of stuck).

The part of Asterisk that connects to the ARA use a well defined family
name to find the proper database driver. The syntax is easy:

\begin{verbatim}
    <family> => <realtime driver>,<db name>[,<table>]
\end{verbatim}

The options following the realtime driver identified depends on the
driver.

Defined well-known family names are:

\begin{itemize}
  \item sippeers, sipusers - SIP peers and users
  \item iaxpeers, iaxusers - IAX2 peers and users
  \item voicemail - Voicemail accounts
  \item queues - Queues
  \item queue\_members - Queue members
  \item extensions - Realtime extensions (switch)
\end{itemize}

Voicemail storage with the support of ODBC described in file
\path{docs/odbcstorage.tex} (\ref{odbcstorage}).

\subsubsection{Limitations}

Currently, realtime extensions do not support realtime hints.  There is
a workaround available by using func\_odbc.  See the sample func\_odbc.conf
for more information.

\subsubsection{FreeTDS supported with connection pooling}

In order to use a FreeTDS-based database with realtime, you need to turn
connection pooling on in res\_odbc.conf.  This is due to a limitation within
the FreeTDS protocol itself.  Please note that this includes databases such
as MS SQL Server and Sybase.  This support is new in the current release.
