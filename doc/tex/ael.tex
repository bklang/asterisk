\section{Introduction}

AEL is a specialized language intended purely for 
describing Asterisk dial plans.

The current version was written by Steve Murphy, and is a rewrite of 
the original version.

This new version further extends AEL, and
provides more flexible syntax, better error messages, and some missing
functionality.

AEL is really the merger of 4 different 'languages', or syntaxes:

\begin{itemize}
    \item The first and most obvious is the AEL syntax itself. A BNF is
      provided near the end of this document.

    \item The second syntax is the Expression Syntax, which is normally
     handled by Asterisk extension engine, as expressions enclosed in
     \$[...]. The right hand side of assignments are wrapped in \$[ ... ] 
     by AEL, and so are the if and while expressions, among others.

    \item The third syntax is the Variable Reference Syntax, the stuff
      enclosed in \${..} curly braces. It's a bit more involved than just
      putting a variable name in there. You can include one of dozens of
      'functions', and their arguments, and there are even some string
      manipulation notation in there.

    \item The last syntax that underlies AEL, and is not used
      directly in AEL, is the Extension Language Syntax. The
      extension language is what you see in extensions.conf, and AEL
      compiles the higher level AEL language into extensions and
      priorities, and passes them via function calls into
      Asterisk. Embedded in this language is the Application/AGI
      commands, of which one application call per step, or priority
      can be made. You can think of this as a "macro assembler"
      language, that AEL will compile into.
\end{itemize}

Any programmer of AEL should be familiar with it's syntax, of course,
as well as the Expression syntax, and the Variable syntax.


\section{Asterisk in a Nutshell}

Asterisk acts as a server. Devices involved in telephony, like Zapata
cards, or Voip phones, all indicate some context that should be
activated in their behalf. See the config file formats for IAX, SIP,
zapata.conf, etc. They all help describe a device, and they all
specify a context to activate when somebody picks up a phone, or a
call comes in from the phone company, or a voip phone, etc.

\subsection{Contexts}

Contexts are a grouping of extensions.

Contexts can also include other contexts. Think of it as a sort of
merge operation at runtime, whereby the included context's extensions
are added to the contexts making the inclusion.

\subsection{Extensions and priorities}

A Context contains zero or more Extensions. There are several
predefined extensions. The "s" extension is the "start" extension, and
when a device activates a context the "s" extension is the one that is
going to be run. Other extensions are the timeout "t" extension, the
invalid response, or "i" extension, and there's a "fax" extension. For
instance, a normal call will activate the "s" extension, but an
incoming FAX call will come into the "fax" extension, if it
exists. (BTW, asterisk can tell it's a fax call by the little "beep"
that the calling fax machine emits every so many seconds.).

Extensions contain several priorities, which are individual
instructions to perform. Some are as simple as setting a variable to a
value. Others are as complex as initiating the Voicemail application,
for instance. Priorities are executed in order.

When the 's" extension completes, asterisk waits until the timeout for
a response. If the response matches an extension's pattern in the
context, then control is transferred to that extension. Usually the
responses are tones emitted when a user presses a button on their
phone. For instance, a context associated with a desk phone might not
have any "s" extension. It just plays a dialtone until someone starts
hitting numbers on the keypad, gather the number, find a matching
extension, and begin executing it. That extension might Dial out over
a connected telephone line for the user, and then connect the two
lines together.

The extensions can also contain "goto" or "jump" commands to skip to
extensions in other contexts. Conditionals provide the ability to
react to different stimuli, and there you have it.

\subsection{Macros}

Think of a macro as a combination of a context with one nameless
extension, and a subroutine. It has arguments like a subroutine
might. A macro call can be made within an extension, and the
individual statements there are executed until it ends. At this point,
execution returns to the next statement after the macro call. Macros
can call other macros. And they work just like function calls.

\subsection{Applications}

Application calls, like "Dial()", or "Hangup()", or "Answer()", are
available for users to use to accomplish the work of the
dialplan. There are over 145 of them at the moment this was written,
and the list grows as new needs and wants are uncovered. Some
applications do fairly simple things, some provide amazingly complex
services.

Hopefully, the above objects will allow you do anything you need to in
the Asterisk environment!

\section{Getting Started}

The AEL parser (pbx\_ael.so) is completely separate from the module
that parses extensions.conf (pbx\_config.so). To use AEL, the only
thing that has to be done is the module pbx\_ael.so must be loaded by
Asterisk. This will be done automatically if using 'autoload=yes' in
/etc/asterisk/modules.conf. When the module is loaded, it will look
for 'extensions.ael' in /etc/asterisk/. extensions.conf and
extensions.ael can be used in conjunction with
each other if that is what is desired. Some users may want to keep
extensions.conf for the features that are configured in the 'general'
section of extensions.conf.

Reloading extensions.ael

To reload extensions.ael, the following command can be issued at the
CLI:

    *CLI> ael reload


\section{Debugging}

Right at this moment, the following commands are available, but do
nothing:

Enable AEL contexts debug
   *CLI> ael debug contexts 

Enable AEL macros debug
   *CLI> ael debug macros 

Enable AEL read debug
   *CLI> ael debug read

Enable AEL tokens debug
   *CLI> ael debug tokens 

Disable AEL debug messages
   *CLI> ael no debug

If things are going wrong in your dialplan, you can use the following
facilities to debug your file:

1. The messages log in /var/log/asterisk. (from the checks done at load time).
2. the "show dialplan" command in asterisk
3. the standalone executable, "aelparse" built in the utils/ dir in the source.


\section{About "aelparse"}

You can use the "aelparse" program to check your extensions.ael
file before feeding it to asterisk. Wouldn't it be nice to eliminate
most errors before giving the file to asterisk?

aelparse is compiled in the utils directory of the asterisk release.
It isn't installed anywhere (yet). You can copy it to your favorite
spot in your PATH.

aelparse has two optional arguments:

\begin{itemize}
  \item -d
  \begin{itemize}
    \item Override the normal location of the config file dir, (usually
       /etc/asterisk), and use the current directory instead as the
       config file dir. Aelparse will then expect to find the file
       "./extensions.ael" in the current directory, and any included
       files in the current directory as well.
  \end{itemize}
  \item -n
  \begin{itemize}
    \item don't show all the function calls to set priorities and contexts
       within asterisk. It will just show the errors and warnings from
       the parsing and semantic checking phases.
  \end{itemize}
\end{itemize}

\section{General Notes about Syntax}

Note that the syntax and style are now a little more free-form. The
opening '{' (curly-braces) do not have to be on the same line as the
keyword that precedes them. Statements can be split across lines, as
long as tokens are not broken by doing so. More than one statement can
be included on a single line. Whatever you think is best!

You can just as easily say,

\begin{verbatim}
if(${x}=1) { NoOp(hello!); goto s|3; } else { NoOp(Goodbye!); goto s|12; }
\end{verbatim}

as you can say:

\begin{verbatim}
if(${x}=1)
{
       NoOp(hello!);
   goto s|3;
}
else
{
       NoOp(Goodbye!);
       goto s|12;
}
\end{verbatim}

or:

\begin{verbatim}
if(${x}=1) {
       NoOp(hello!);
   goto s|3;
} else {
       NoOp(Goodbye!);
       goto s|12;
}
\end{verbatim}

or:

\begin{verbatim}
if (${x}=1) {
       NoOp(hello!); goto s|3;
} else {
       NoOp(Goodbye!); goto s|12;
}
\end{verbatim}

\section{Keywords}

The AEL keywords are case-sensitive. If an application name and a
keyword overlap, there is probably good reason, and you should
consider replacing the application call with an AEL statement. If you
do not wish to do so, you can still use the application, by using a
capitalized letter somewhere in its name. In the Asterisk extension
language, application names are NOT case-sensitive.

The following are keywords in the AEL language:
\begin{itemize}
    \item abstract
    \item context
    \item macro
    \item globals
    \item ignorepat
    \item switch
    \item if
    \item ifTime
    \item else
    \item random
    \item goto
    \item jump
    \item local
    \item return
    \item break
    \item continue
    \item regexten
    \item hint
    \item for
    \item while
    \item case
    \item pattern
    \item default   NOTE: the "default" keyword can be used as a context name, 
                      for those who would like to do so.
    \item catch
    \item switches
    \item eswitches
    \item includes 
\end{itemize}


\section{Procedural Interface and Internals}

AEL first parses the extensions.ael file into a memory structure representing the file.
The entire file is represented by a tree of "pval" structures linked together.

This tree is then handed to the semantic check routine. 

Then the tree is handed to the compiler. 

After that, it is freed from memory.

A program could be written that could build a tree of pval structures, and
a pretty printing function is provided, that would dump the data to a file,
or the tree could be handed to the compiler to merge the data into the 
asterisk dialplan. The modularity of the design offers several opportunities
for developers to simplify apps to generate dialplan data.


\subsection{AEL version 2 BNF}

(hopefully, something close to bnf).

First, some basic objects

\begin{verbatim}
------------------------
<word>    a lexical token consisting of characters matching this pattern: [-a-zA-Z0-9"_/.\<\>\*\+!$#\[\]][-a-zA-Z0-9"_/.!\*\+\<\>\{\}$#\[\]]*

<word3-list>  a concatenation of up to 3 <word>s.

<collected-word>  all characters encountered until the character that follows the <collected-word> in the grammar.
-------------------------

<file> :== <objects>

<objects> :== <object>
           | <objects> <object>


<object> :==  <context>
         | <macro>
         | <globals>
         | ';'


<context> :==  'context' <word> '{' <elements> '}'
            | 'context' <word> '{' '}'
            | 'context' 'default' '{' <elements> '}'
            | 'context' 'default' '{' '}'
            | 'abstract'  'context' <word> '{' <elements> '}'
            | 'abstract'  'context' <word> '{' '}'
            | 'abstract'  'context' 'default' '{' <elements> '}'
            | 'abstract'  'context' 'default' '{' '}'


<macro> :== 'macro' <word> '(' <arglist> ')' '{' <macro_statements> '}'
       | 'macro' <word> '(' <arglist> ')' '{'  '}'
       | 'macro' <word> '(' ')' '{' <macro_statements> '}'
       | 'macro' <word> '(' ')' '{'  '}'


<globals> :== 'globals' '{' <global_statements> '}'
         | 'globals' '{' '}'


<global_statements> :== <global_statement>
                   | <global_statements> <global_statement>


<global_statement> :== <word> '=' <collected-word> ';'


<arglist> :== <word>
         | <arglist> ',' <word>


<elements> :==  <element>
             | <elements> <element>


<element> :== <extension>
         | <includes>
         | <switches>
         | <eswitches>
         | <ignorepat>
         | <word> '='  <collected-word> ';'
         | 'local' <word> '='  <collected-word> ';'
         | ';'


<ignorepat> :== 'ignorepat' '=>' <word> ';'


<extension> :== <word> '=>' <statement>
           | 'regexten' <word> '=>' <statement>
           | 'hint' '(' <word3-list> ')' <word> '=>' <statement>
           | 'regexten' 'hint' '(' <word3-list> ')' <word> '=>' <statement>


<statements> :== <statement>
            | <statements> <statement>

<if_head> :== 'if' '('  <collected-word> ')'

<random_head> :== 'random' '(' <collected-word> ')'

<ifTime_head> :== 'ifTime' '(' <word3-list> ':' <word3-list> ':' <word3-list> '|' <word3-list> '|' <word3-list> '|' <word3-list> ')'
                       | 'ifTime' '(' <word> '|' <word3-list> '|' <word3-list> '|' <word3-list> ')'


<word3-list> :== <word>
       | <word> <word>
       | <word> <word> <word>

<switch_head> :== 'switch' '(' <collected-word> ')'  '{'


<statement> :== '{' <statements> '}'
       | <word> '='  <collected-word> ';'
       | 'local' <word> '='  <collected-word> ';'
       | 'goto' <target> ';'
       | 'jump' <jumptarget> ';'
       | <word> ':'
       | 'for' '('  <collected-word> ';'  <collected-word> ';' <collected-word> ')' <statement>
       | 'while' '('  <collected-word> ')' <statement>
       | <switch_head> '}'
       | <switch_head> <case_statements> '}'
       | '&' macro_call ';'
       | <application_call> ';'
       | <application_call> '='  <collected-word> ';'
       | 'break' ';'
       | 'return' ';'
       | 'continue' ';'
       | <random_head> <statement>
       | <random_head> <statement> 'else' <statement>
       | <if_head> <statement>
       | <if_head> <statement> 'else' <statement>
       | <ifTime_head> <statement>
       | <ifTime_head> <statement> 'else' <statement>
       | ';'

<target> :== <word>
       | <word> '|' <word>
       | <word> '|' <word> '|' <word>
       | 'default' '|' <word> '|' <word>
       | <word> ',' <word>
       | <word> ',' <word> ',' <word>
       | 'default' ',' <word> ',' <word>

<jumptarget> :== <word>
               | <word> ',' <word>
               | <word> ',' <word> '@' <word>
               | <word> '@' <word>
               | <word> ',' <word> '@' 'default'
               | <word> '@' 'default'

<macro_call> :== <word> '(' <eval_arglist> ')'
       | <word> '(' ')'

<application_call_head> :== <word>  '('

<application_call> :== <application_call_head> <eval_arglist> ')'
       | <application_call_head> ')'

<eval_arglist> :==  <collected-word>
       | <eval_arglist> ','  <collected-word>
       |  /* nothing */
       | <eval_arglist> ','  /* nothing */

<case_statements> :== <case_statement>
       | <case_statements> <case_statement>


<case_statement> :== 'case' <word> ':' <statements>
       | 'default' ':' <statements>
       | 'pattern' <word> ':' <statements>
       | 'case' <word> ':'
       | 'default' ':'
       | 'pattern' <word> ':'

<macro_statements> :== <macro_statement>
       | <macro_statements> <macro_statement>

<macro_statement> :== <statement>
       | 'catch' <word> '{' <statements> '}'

<switches> :== 'switches' '{' <switchlist> '}'
       | 'switches' '{' '}'

<eswitches> :== 'eswitches' '{' <switchlist> '}'
       | 'eswitches' '{'  '}'

<switchlist> :== <word> ';'
       | <switchlist> <word> ';'

<includeslist> :== <includedname> ';'
       | <includedname> '|' <word3-list> ':' <word3-list> ':' <word3-list> '|' <word3-list> '|' <word3-list> '|' <word3-list> ';'
       | <includedname> '|' <word> '|' <word3-list> '|' <word3-list> '|' <word3-list> ';'
       | <includeslist> <includedname> ';'
       | <includeslist> <includedname> '|' <word3-list> ':' <word3-list> ':' <word3-list> '|' <word3-list> '|' <word3-list> '|' <word3-list> ';'
       | <includeslist> <includedname> '|' <word> '|' <word3-list> '|' <word3-list> '|' <word3-list> ';'

<includedname> :== <word>
        | 'default'

<includes> :== 'includes' '{' <includeslist> '}'
       | 'includes' '{' '}'

\end{verbatim}


\section{AEL Example USAGE}

\subsection{Comments}

Comments begin with // and end with the end of the line.

Comments are removed by the lexical scanner, and will not be
recognized in places where it is busy gathering expressions to wrap in
\$[] , or inside application call argument lists. The safest place to put
comments is after terminating semicolons, or on otherwise empty lines.


\subsection{Context}

Contexts in AEL represent a set of extensions in the same way that
they do in extensions.conf.

\begin{verbatim}
context default {

}


A context can be declared to be "abstract", in which case, this
declaration expresses the intent of the writer, that this context will
only be included by another context, and not "stand on its own". The
current effect of this keyword is to prevent "goto " statements from
being checked.

\begin{verbatim}
abstract context longdist {
     _1NXXNXXXXXX => NoOp(generic long distance dialing actions in the US);
}
\end{verbatim}


\subsection{Extensions}

To specify an extension in a context, the following syntax is used. If
more than one application is be called in an extension, they can be
listed in order inside of a block.

\begin{verbatim}
context default {
    1234 => Playback(tt-monkeys);
    8000 => {
         NoOp(one);
         NoOp(two);
         NoOp(three);
    };
    _5XXX => NoOp(it's a pattern!);
}
\end{verbatim}

Two optional items have been added to the AEL syntax, that allow the
specification of hints, and a keyword, regexten, that will force the
numbering of priorities to start at 2.

The ability to make extensions match by CID is preserved in
AEL; just use '/' and the CID number in the specification. See below.

\begin{verbatim}
context default {

    regexten _5XXX => NoOp(it's a pattern!);
}
\end{verbatim}

\begin{verbatim}
context default {

    hint(Sip/1) _5XXX => NoOp(it's a pattern!);
}
\end{verbatim}

\begin{verbatim}
context default {

    regexten hint(Sip/1) _5XXX => NoOp(it's a pattern!);
}
\end{verbatim}

The regexten must come before the hint if they are both present.

CID matching is done as with the extensions.conf file. Follow the extension
name/number with a slash (/) and the number to match against the Caller ID:

\begin{verbatim}
context zoombo 
{
	819/7079953345 => { NoOp(hello, 3345); }
}
\end{verbatim}

In the above,  the 819/7079953345 extension will only be matched if the
CallerID is 7079953345, and the dialed number is 819. Hopefully you have
another 819 extension defined for all those who wish 819, that are not so lucky
as to have 7079953345 as their CallerID!


\subsection{Includes}

Contexts can be included in other contexts. All included contexts are
listed within a single block.

\begin{verbatim}
context default {
    includes {
         local;
         longdistance;
         international;
    }
}
\end{verbatim}

Time-limited inclusions can be specified, as in extensions.conf
format, with the fields described in the wiki page Asterisk cmd
GotoIfTime.

\begin{verbatim}
context default {
    includes {
         local;
         longdistance|16:00-23:59|mon-fri|*|*;
         international;
    }
}
\end{verbatim}

\subsection{\#include}

You can include other files with the \#include "filepath" construct.

\begin{verbatim}
   #include "/etc/asterisk/testfor.ael"
\end{verbatim}

An interesting property of the \#include, is that you can use it almost
anywhere in the .ael file. It is possible to include the contents of
a file in a macro, context, or even extension.  The \#include does not
have to occur at the beginning of a line. Included files can include
other files, up to 50 levels deep. If the path provided in quotes is a
relative path, the parser looks in the config file directory for the
file (usually /etc/asterisk).



\subsection{Dialplan Switches}

Switches are listed in their own block within a context. For clues as
to what these are used for, see Asterisk - dual servers, and Asterisk
config extensions.conf.

\begin{verbatim}
context default {
    switches {
         DUNDi/e164;
         IAX2/box5;
    };
    eswitches {
         IAX2/context@${CURSERVER};
    }
}
\end{verbatim}


\subsection{Ignorepat}

ignorepat can be used to instruct channel drivers to not cancel
dialtone upon receipt of a particular pattern. The most commonly used
example is '9'.

\begin{verbatim}
context outgoing {
    ignorepat => 9;
}
\end{verbatim}


\subsection{Variables}

Variables in Asterisk do not have a type, so to define a variable, it
just has to be specified with a value.

Global variables are set in their own block.

\begin{verbatim}
globals {
    CONSOLE=Console/dsp;
    TRUNK=Zap/g2;
}
\end{verbatim}


Variables can be set within extensions as well.

\begin{verbatim}
context foo {
    555 => {
         x=5;
         y=blah;
         divexample=10/2
         NoOp(x is ${x} and y is ${y} !);
    }
}
\end{verbatim}

NOTE: AEL wraps the right hand side of an assignment with \$[ ] to allow 
expressions to be used If this is unwanted, you can protect the right hand 
side from being wrapped by using the Set() application. 
Read the README.variables about the requirements and behavior 
of \$[ ] expressions.

NOTE: These things are wrapped up in a \$[ ] expression: The while() test; 
the if() test; the middle expression in the for( x; y; z) statement 
(the y expression); Assignments - the right hand side, so a = b -> Set(a=\$[b])

Writing to a dialplan function is treated the same as writing to a variable.

\begin{verbatim}
context blah {
    s => {
         CALLERID(name)=ChickenMan;
         NoOp(My name is ${CALLERID(name)} !);
    }
} 
\end{verbatim}

You can declare variables in Macros, as so:

\begin{verbatim}
Macro myroutine(firstarg, secondarg)
{
	Myvar=1;
	NoOp(Myvar is set to ${myvar});
}
\end{verbatim}

\subsection{Local Variables}

In 1.2, and 1.4, ALL VARIABLES are CHANNEL variables, including the function
arguments and associated ARG1, ARG2, etc variables. Sorry.

In trunk (1.6 and higher), we have made all arguments local variables to
a macro call. They will not affect channel variables of the same name.
This includes the ARG1, ARG2, etc variables. 

Users can declare their own local variables by using the keyword 'local'
before setting them to a value;

\begin{verbatim}
Macro myroutine(firstarg, secondarg)
{
	local Myvar=1;
	NoOp(Myvar is set to ${Myvar}, and firstarg is ${firstarg}, and secondarg is ${secondarg});
}
\end{verbatim}

In the above example, Myvar, firstarg, and secondarg are all local variables,
and will not be visible to the calling code, be it an extension, or another Macro. 

If you need to make a local variable within the Set() application, you can do it this way:

\begin{verbatim}
Macro myroutine(firstarg, secondarg)
{
	Set(LOCAL(Myvar)=1);
	NoOp(Myvar is set to ${Myvar}, and firstarg is ${firstarg}, and secondarg is ${secondarg});
}
\end{verbatim}


\subsection{Loops}

AEL has implementations of 'for' and 'while' loops.

\begin{verbatim}
context loops {
    1 => {
         for (x=0; ${x} < 3; x=${x} + 1) {
              Verbose(x is ${x} !);
         }
    }
    2 => {
         y=10;
         while (${y} >= 0) {
              Verbose(y is ${y} !);
              y=${y}-1;
         }
    }
}
\end{verbatim}

NOTE: The conditional expression (the "\${y} >= 0" above) is wrapped in
      \$[ ] so it can be evaluated.  NOTE: The for loop test expression
      (the "\${x} < 3" above) is wrapped in \$[ ] so it can be evaluated.



\subsection{Conditionals}

AEL supports if and switch statements, like AEL, but adds ifTime, and
random. Unlike the original AEL, though, you do NOT need to put curly
braces around a single statement in the "true" branch of an if(), the
random(), or an ifTime() statement. The if(), ifTime(), and random()
statements allow optional else clause.

\begin{verbatim}
context conditional {
    _8XXX => {
         Dial(SIP/${EXTEN});
         if ("${DIALSTATUS}" = "BUSY")
         {
              NoOp(yessir);
              Voicemail(${EXTEN}|b);
         }
         else
              Voicemail(${EXTEN}|u);
         ifTime (14:00-25:00|sat-sun|*|*) 
              Voicemail(${EXTEN}|b);
         else
         {
              Voicemail(${EXTEN}|u);
              NoOp(hi, there!);
         }
         random(51) NoOp(This should appear 51% of the time);

         random( 60 )
         {
                       NoOp( This should appear 60% of the time );
         }
         else
         {
                       random(75)
                       {
                               NoOp( This should appear 30% of the time! );
                       }
                       else
                       {
                               NoOp( This should appear 10% of the time! );
                       }
          }
    }
    _777X => {
         switch (${EXTEN}) {
              case 7771:
                   NoOp(You called 7771!);
                   break;
              case 7772:
                   NoOp(You called 7772!);
                   break;
              case 7773:
                   NoOp(You called 7773!);
                   // fall thru-
              pattern 777[4-9]:
                    NoOp(You called 777 something!);
              default:
                   NoOp(In the default clause!);
         }
    }
}
\end{verbatim}

NOTE: The conditional expression in if() statements (the
      "\${DIALSTATUS}" = "BUSY" above) is wrapped by the compiler in 
      \$[] for evaluation.

NOTE: Neither the switch nor case values are wrapped in \$[ ]; they can
      be constants, or \${var} type references only.

NOTE: AEL generates each case as a separate extension. case clauses
      with no terminating 'break', or 'goto', have a goto inserted, to
      the next clause, which creates a 'fall thru' effect.

NOTE: AEL introduces the ifTime keyword/statement, which works just
      like the if() statement, but the expression is a time value,
      exactly like that used by the application GotoIfTime(). See
      Asterisk cmd GotoIfTime

NOTE: The pattern statement makes sure the new extension that is
      created has an '\_' preceding it to make sure asterisk recognizes
      the extension name as a pattern.

NOTE: Every character enclosed by the switch expression's parenthesis
      are included verbatim in the labels generated. So watch out for
      spaces!

NOTE: NEW: Previous to version 0.13, the random statement used the
      "Random()" application, which has been deprecated. It now uses
      the RAND() function instead, in the GotoIf application.


\subsection{Break, Continue, and Return}

Three keywords, break, continue, and return, are included in the
syntax to provide flow of control to loops, and switches.

The break can be used in switches and loops, to jump to the end of the
loop or switch.

The continue can be used in loops (while and for) to immediately jump
to the end of the loop. In the case of a for loop, the increment and
test will then be performed. In the case of the while loop, the
continue will jump to the test at the top of the loop.

The return keyword will cause an immediate jump to the end of the
context, or macro, and can be used anywhere.



\subsection{goto, jump, and labels}

This is an example of how to do a goto in AEL.

\begin{verbatim}
context gotoexample {
    s => {
begin:
         NoOp(Infinite Loop!  yay!);
         Wait(1);
         goto begin;    // go to label in same extension
    }
    3 => {
            goto s|begin;   // go to label in different extension
     }
     4 => {
            goto gotoexample|s|begin;  // overkill go to label in same context
     }
}

context gotoexample2 {
     s =>  {
   end: 
           goto gotoexample|s|begin;   // go to label in different context
     }
}
\end{verbatim}

You can use the special label of "1" in the goto and jump
statements. It means the "first" statement in the extension. I would
not advise trying to use numeric labels other than "1" in goto's or
jumps, nor would I advise declaring a "1" label anywhere! As a matter
of fact, it would be bad form to declare a numeric label, and it might
conflict with the priority numbers used internally by asterisk.

The syntax of the jump statement is: jump
extension[,priority][@context] If priority is absent, it defaults to
"1". If context is not present, it is assumed to be the same as that
which contains the "jump".

\begin{verbatim}
context gotoexample {
    s => {
begin:
         NoOp(Infinite Loop!  yay!);
         Wait(1);
         jump s;    // go to first extension in same extension
    }
    3 => {
            jump s,begin;   // go to label in different extension
     }
     4 => {
            jump s,begin@gotoexample;  // overkill go to label in same context
     }
}

context gotoexample2 {
     s =>  {
   end: 
           jump s@gotoexample;   // go to label in different context
     }
}
\end{verbatim}

NOTE: goto labels follow the same requirements as the Goto()
      application, except the last value has to be a label. If the
      label does not exist, you will have run-time errors. If the
      label exists, but in a different extension, you have to specify
      both the extension name and label in the goto, as in: goto s|z;
      if the label is in a different context, you specify
      context|extension|label. There is a note about using goto's in a
      switch statement below...

NOTE  AEL introduces the special label "1", which is the beginning
      context number for most extensions.

NOTE: A NEW addition to AEL: you can now use ',' instead of '|' to 
      separate the items in the target address. You can't have a mix,
      though, of '|' and ',' in the target. It's either one, or the other.




\subsection{Macros}

A macro is defined in its own block like this. The arguments to the
macro are specified with the name of the macro. They are then referred
to by that same name. A catch block can be specified to catch special
extensions.

\begin{verbatim}
macro std-exten( ext , dev ) {
       Dial(${dev}/${ext},20);
       switch(${DIALSTATUS) {
       case BUSY:
               Voicemail(b${ext});
               break;
       default:
               Voicemail(u${ext});

       }
       catch a {
               VoiceMailMain(${ext});
               return;
       }
}
\end{verbatim}

A macro is then called by preceding the macro name with an
ampersand. Empty arguments can be passed simply with nothing between
comments(0.11).

\begin{verbatim}
context example {
    _5XXX => &std-exten(${EXTEN}, "IAX2");
    _6XXX => &std-exten(, "IAX2");
    _7XXX => &std-exten(${EXTEN},);
    _8XXX => &std-exten(,);
}
\end{verbatim}


\section{Examples}

\begin{verbatim}
context demo {
    s => {
         Wait(1);
         Answer();
         TIMEOUT(digit)=5;
         TIMEOUT(response)=10;
restart:
         Background(demo-congrats);
instructions:
         for (x=0; ${x} < 3; x=${x} + 1) {
              Background(demo-instruct);
              WaitExten();
         }
    }
    2 => {
         Background(demo-moreinfo);
         goto s|instructions;
    }
    3 => {
         LANGUAGE()=fr;
         goto s|restart;
    }

    500 => {
         Playback(demo-abouttotry);
         Dial(IAX2/guest@misery.digium.com);
         Playback(demo-nogo);
         goto s|instructions;
    }
    600 => {
         Playback(demo-echotest);
         Echo();
         Playback(demo-echodone);
         goto s|instructions;
    }
    # => {
hangup:
         Playback(demo-thanks);
         Hangup();
    }
    t => goto #|hangup;
    i => Playback(invalid);
}
\end{verbatim}


\section{Semantic Checks}


AEL, after parsing, but before compiling, traverses the dialplan
tree, and makes several checks:

\begin{itemize}
    \item Macro calls to non-existent macros.
    \item Macro calls to contexts.
    \item Macro calls with argument count not matching the definition.
    \item application call to macro. (missing the '\&')
    \item application calls to "GotoIf", "GotoIfTime", "while",
      "endwhile", "Random", and "execIf", will generate a message to
      consider converting the call to AEL goto, while, etc. constructs.
    \item goto a label in an empty extension.
    \item goto a non-existent label, either a within-extension,
      within-context, or in a different context, or in any included
      contexts. Will even check "sister" context references.
    \item All the checks done on the time values in the dial plan, are
      done on the time values in the ifTime() and includes times:
          o the time range has to have two times separated by a dash;
          o the times have to be in range of 0 to 24 hours.
          o The weekdays have to match the internal list, if they are provided;
          o the day of the month, if provided, must be in range of 1 to 31;
          o the month name or names have to match those in the internal list. 
    \item (0.5) If an expression is wrapped in \$[ ... ], and the compiler
      will wrap it again, a warning is issued.
    \item (0.5) If an expression had operators (you know,
      +,-,*,/,%,!,etc), but no \${ } variables, a warning is
      issued. Maybe someone forgot to wrap a variable name?
    \item (0.12) check for duplicate context names.
    \item (0.12) check for abstract contexts that are not included by any context.
    \item (0.13) Issue a warning if a label is a numeric value. 
\end{itemize}

There are a subset of checks that have been removed until the proposed
AAL (Asterisk Argument Language) is developed and incorporated into Asterisk.
These checks will be:

\begin{itemize}
    \item (if the application argument analyzer is working: the presence
      of the 'j' option is reported as error.
    \item if options are specified, that are not available in an
      application.
    \item if you specify too many arguments to an application.
    \item a required argument is not present in an application call.
    \item Switch-case using "known" variables that applications set, that
      does not cover all the possible values. (a "default" case will
      solve this problem. Each "unhandled" value is listed.
    \item a Switch construct is used, which is uses a known variable, and
      the application that would set that variable is not called in
      the same extension. This is a warning only...
    \item Calls to applications not in the "applist" database (installed
      in /var/lib/asterisk/applist" on most systems).
    \item In an assignment statement, if the assignment is to a function,
      the function name used is checked to see if it one of the
      currently known functions. A warning is issued if it is not.
\end{itemize}


Differences with the original version of AEL
============================================

\begin{enumerate}
   \item The \$[...] expressions have been enhanced to include the ==, ||,
      and \&\& operators. These operators are exactly equivalent to the
      =, |, and \& operators, respectively. Why? So the C, Java, C++
      hackers feel at home here.
   \item It is more free-form. The newline character means very little,
      and is pulled out of the white-space only for line numbers in
      error messages.
   \item It generates more error messages -- by this I mean that any
      difference between the input and the grammar are reported, by
      file, line number, and column.
   \item It checks the contents of \$[ ] expressions (or what will end up
      being \$[ ] expressions!) for syntax errors. It also does
      matching paren/bracket counts.
   \item It runs several semantic checks after the parsing is over, but
      before the compiling begins, see the list above.
   \item It handles \#include "filepath" directives. -- ALMOST
      anywhere, in fact. You could easily include a file in a context,
      in an extension, or at the root level. Files can be included in
      files that are included in files, down to 50 levels of hierarchy...
   \item Local Goto's inside Switch statements automatically have the
      extension of the location of the switch statement appended to them.
   \item A pretty printer function is available within pbx\_ael.so.
   \item In the utils directory, two standalone programs are supplied for
      debugging AEL files. One is called "aelparse", and it reads in
      the /etc/asterisk/extensions.ael file, and shows the results of
      syntax and semantic checking on stdout, and also shows the
      results of compilation to stdout. The other is "aelparse1",
      which uses the original ael compiler to do the same work,
      reading in "/etc/asterisk/extensions.ael", using the original
      'pbx\_ael.so' instead.
  \item AEL supports the "jump" statement, and the "pattern" statement
      in switch constructs. Hopefully these will be documented in the
      AEL README.
  \item Added the "return" keyword, which will jump to the end of an
      extension/Macro.
  \item Added the ifTime (<time range>|<days of week>|<days of
      month>|<months> ) {} [else {}] construct, which executes much
      like an if () statement, but the decision is based on the
      current time, and the time spec provided in the ifTime. See the
      example above. (Note: all the other time-dependent Applications
      can be used via ifTime)
  \item Added the optional time spec to the contexts in the includes
      construct. See examples above.
  \item You don't have to wrap a single "true" statement in curly
      braces, as in the original AEL. An "else" is attached to the
      closest if. As usual, be careful about nested if statements!
      When in doubt, use curlies!
  \item Added the syntax [regexten] [hint(channel)] to precede an
      extension declaration. See examples above, under
      "Extension". The regexten keyword will cause the priorities in
      the extension to begin with 2 instead of 1. The hint keyword
      will cause its arguments to be inserted in the extension under
      the hint priority. They are both optional, of course, but the
      order is fixed at the moment-- the regexten must come before the
      hint, if they are both present.
  \item Empty case/default/pattern statements will "fall thru" as
      expected. (0.6)
  \item A trailing label in an extension, will automatically have a
      NoOp() added, to make sure the label exists in the extension on
      Asterisk. (0.6)
  \item (0.9) the semicolon is no longer required after a closing brace!
      (i.e. "];" ===> "\}". You can have them there if you like, but
      they are not necessary. Someday they may be rejected as a syntax
      error, maybe.
  \item (0.9) the // comments are not recognized and removed in the
      spots where expressions are gathered, nor in application call
      arguments. You may have to move a comment if you get errors in
      existing files.
  \item (0.10) the random statement has been added. Syntax: random (
      <expr> ) <lucky-statement> [ else <unlucky-statement> ]. The
      probability of the lucky-statement getting executed is <expr>,
      which should evaluate to an integer between 0 and 100. If the
      <lucky-statement> isn't so lucky this time around, then the
      <unlucky-statement> gets executed, if it is present.
\end{enumerate}


\section{Hints and Bugs}

     The safest way to check for a null strings is to say \$[ "\${x}" =
     "" ] The old way would do as shell scripts often do, and append
     something on both sides, like this: \$[ \${x}foo = foo ]. The
     trouble with the old way, is that, if x contains any spaces, then
     problems occur, usually syntax errors. It is better practice and
     safer wrap all such tests with double quotes! Also, there are now
     some functions that can be used in a variable reference,
     ISNULL(), and LEN(), that can be used to test for an empty string:
     \${ISNULL(\${x})} or \$[ \${LEN(\${x}) = 0 ].

      Assignment vs. Set(). Keep in mind that setting a variable to
      value can be done two different ways. If you choose say 'x=y;',
      keep in mind that AEL will wrap the right-hand-side with
      \$[]. So, when compiled into extension language format, the end
      result will be 'Set(x=\$[y])'. If you don't want this effect,
      then say "Set(x=y);" instead.


\section{The Full Power of AEL}

A newcomer to Asterisk will look at the above constructs and
descriptions, and ask, "Where's the string manipulation functions?",
"Where's all the cool operators that other languages have to offer?",
etc.

The answer is that the rich capabilities of Asterisk are made
available through AEL, via:

\begin{itemize}
    \item Applications: See Asterisk - documentation of application
      commands

    \item Functions: Functions were implemented inside \${ .. } variable
      references, and supply many useful capabilities. 

    \item Expressions: An expression evaluation engine handles items
      wrapped inside \$[...]. This includes some string manipulation
      facilities, arithmetic expressions, etc. 

    \item Application Gateway Interface: Asterisk can fork external
      processes that communicate via pipe. AGI applications can be
      written in any language. Very powerful applications can be added
      this way. 

    \item Variables: Channels of communication have variables associated
      with them, and asterisk provides some global variables. These can be
      manipulated and/or consulted by the above mechanisms. 
\end{itemize}
