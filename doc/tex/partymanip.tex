\section{Introduction}

This chapter aims to explain how to use some of the features available to
manipulate party ID information.  It will not delve into specific channel
configuration options described in the respective sample configuration
files.  The party ID information can consist of Caller ID, Connected Line
ID, redirecting to party ID information, and redirecting from party ID
information.  Meticulous control is needed particularly when
interoperating between different channel technologies.

\begin{itemize}

\item Caller ID: The Caller ID information describes who is originating a
call.

\item Connected Line ID: The Connected Line ID information describes who
is connected to the other end of a call while a call is established.
Unlike Caller ID, the connected line information can change over the life
of a call when call transfers are performed.  The connected line
information can also change in either direction because either end could
transfer the call.  For ISDN it is known as Connected Line Identification
Presentation (COLP), Connected Line Identification Restriction (COLR), and
Explicit Call Transfer (ECT).  For SIP it is known either as
P-Asserted-Identity or Remote-Party-Id.

\item Redirecting information: When a call is forwarded, the call
originator is informed that the call is redirecting-to a new destination.
The new destination is also informed that the incoming call is
redirecting-from the forwarding party.  A call can be forwarded repeatedly
until a new destination answers it or a forwarding limit is reached.

\end{itemize}

\section{Tools available}

Asterisk contains several tools for manipulating the party ID information
for a call.  Additional information can be found by using the 'core show
function' or 'core show application' console commands at the Asterisk CLI.
The following list identifies some of the more common tools for
manipulating the party ID information:

\begin{itemize}
\item CALLERID(datatype[,caller-id])

\item CONNECTEDLINE(datatype[,i])

\item REDIRECTING(datatype[,i])

\item Dial() and Queue() dialplan application 'I' option

\item Interception macros

\item Channel driver specific configuration options.
\end{itemize}

\subsection{CALLERID dialplan function}

The CALLERID function has been around for quite a while and its use is
straightforward.  It is used to examine and alter the caller information
that came into the dialplan with the call.  Then the call with it's caller
information passes on to the destination using the Dial() or Queue()
application.

The CALLERID information is passed during the initial call setup.
However, depending on the channel technology, the caller name may be
delayed.  Q.SIG is an example where the caller name may be delayed so your
dialplan may need to wait for it.

\subsection{CONNECTEDLINE dialplan function}

The CONNECTEDLINE function does the opposite of the CALLERID function.
CONNECTEDLINE can be used to setup connected line information to be sent when the
call is answered.  You can use it to send new connected line information
to the remote party on the channel when a call is transferred.  The
CONNECTEDLINE information is passed when the call is answered and when the
call is transferred.

Since the connected line information can be sent while a call is
connected, you may need to prevent the channel driver from acting on a
partial update.  The 'i' option is used to inhibit the channel driver from
sending the changed information immediately.

\subsection{REDIRECTING dialplan function}

The REDIRECTING function allows you to report information about
forwarded/deflected calls to the caller and to the new destination.  The
use of the REDIRECTING function is the most complicated of the party
information functions.

The REDIRECTING information is passed during the initial call setup and
while the call is being routed through the network.  Since the redirecting
information is sent before a call is answered, you need to prevent the
channel driver from acting on a partial update.  The 'i' option is used to
inhibit the channel driver from sending the changed information
immediately.

The incoming call may have already been redirected.  An incoming call has
already been redirected if the REDIRECTING(count) is not zero.  (Alternate
indications are if the REDIRECTING(from-num-valid) is non-zero or if the
REDIRECTING(from-num) is not empty.)

There are several things to do when a call is forwarded by the dialplan:

\begin{itemize}

\item Setup the REDIRECTING(to-xxx) values to be sent to the caller.

\item Setup the REDIRECTING(from-xxx) values to be sent to the new
destination.

\item Increment the REDIRECTING(count).

\item Set the REDIRECTING(reason).

\item Dial() the new destination.

\end{itemize}

\subsection{Special REDIRECTING considerations for ISDN}

Special considerations for Q.SIG and ISDN point-to-point links are needed
to make the DivertingLegInformation1, DivertingLegInformation2, and
DivertingLegInformation3 messages operate properly.

You should manually send the COLR of the redirected-to party for an
incoming redirected call if the incoming call could experience further
redirects.  For chan_misdn, just set the REDIRECTING(to-num,i) =
\$\{EXTEN\} and set the REDIRECTING(to-num-pres) to the COLR.  For
chan_dahdi, just set the REDIRECTING(to-num,i) = CALLERID(dnid) and set
the REDIRECTING(to-num-pres) to the COLR.  (Setting the
REDIRECTING(to-num,i) value may not be necessary since the channel driver
has already attempted to preset that value for automatic generation of the
needed DivertingLegInformation3 message.)

For redirected calls out a trunk line, you need to use the 'i' option on
all of the REDIRECTING statements before dialing the redirected-to party.
The call will update the redirecting-to presentation (COLR) when it
becomes available.

\subsection{Dial() and Queue() dialplan application 'I' option}

In the dialplan applications Dial() and Queue(), the 'I' option is a brute
force option to block connected line and redirecting information updates
while the application is running.  Blocking the updates prevents the
update from overwriting any CONNECTEDLINE or REDIRECTING values you may
have setup before running the application.

The option blocks all redirecting updates since they should only happen
before a call is answered.  The option only blocks the connected line
update from the initial answer.  Connected line updates resulting from
call transfers happen after the application has completed.  Better control
of connected line and redirecting information is obtained using the
interception macros.

\subsection{Interception macros}

The interception macros give the administrator an opportunity to alter
connected line and redirecting information before the channel driver is
given the information.  If the macro does not change a value then that is
what is going to be passed to the channel driver.

The tag string available in CALLERID, CONNECTEDLINE, and REDIRECTING is
useful for the interception macros to provide some information about where
the information originally came from.

The 'i' option of the CONNECTEDLINE dialplan function should always be
used in the CONNECTED_LINE interception macros.  The interception macro
always passes the connected line information on to the channel driver when
the macro exits.  Similarly, the 'i' option of the REDIRECTING dialplan
function should always be used in the REDIRECTING interception macros.

\begin{verbatim}
${REDIRECTING_CALLEE_SEND_MACRO}
  Macro to call before sending a redirecting update to the callee.
  This macro may never be needed since the redirecting updates
  should only go from the callee to the caller direction.  It is
  available for completeness.
${REDIRECTING_CALLEE_SEND_MACRO_ARGS}
  Arguments to pass to ${REDIRECTING_CALLEE_SEND_MACRO}.

${REDIRECTING_CALLER_SEND_MACRO}
  Macro to call before sending a redirecting update to the caller.
${REDIRECTING_CALLER_SEND_MACRO_ARGS}
  Arguments to pass to ${REDIRECTING_CALLER_SEND_MACRO}.

${CONNECTED_LINE_CALLEE_SEND_MACRO}
  Macro to call before sending a connected line update to the callee.
${CONNECTED_LINE_CALLEE_SEND_MACRO_ARGS}
  Arguments to pass to ${CONNECTED_LINE_CALLEE_SEND_MACRO}.

${CONNECTED_LINE_CALLER_SEND_MACRO}
  Macro to call before sending a connected line update to the caller.
${CONNECTED_LINE_CALLER_SEND_MACRO_ARGS}
  Arguments to pass to ${CONNECTED_LINE_CALLER_SEND_MACRO}.
\end{verbatim}

\section{Manipulation examples}

The following examples show several common scenarios in which you may need
to manipulate party ID information from the dialplan.

\subsection{Simple recording playback}

\begin{verbatim}
exten => 1000,1,NoOp
; The CONNECTEDLINE information is sent when the call is answered.
exten => 1000,n,Set(CONNECTEDLINE(name,i)="Company Name")
exten => 1000,n,Set(CONNECTEDLINE(name-pres,i)=allowed)
exten => 1000,n,Set(CONNECTEDLINE(num,i)=5551212)
exten => 1000,n,Set(CONNECTEDLINE(num-pres)=allowed)
exten => 1000,n,Answer
exten => 1000,n,Playback(tt-weasels)
exten => 1000,n,Hangup
\end{verbatim}

\subsection{Straightforward dial through}

\begin{verbatim}
exten => 1000,1,NoOp
; The CONNECTEDLINE information is sent when the call is answered.
exten => 1000,n,Set(CONNECTEDLINE(name,i)="Company Name")
exten => 1000,n,Set(CONNECTEDLINE(name-pres,i)=allowed)
exten => 1000,n,Set(CONNECTEDLINE(num,i)=5551212)
exten => 1000,n,Set(CONNECTEDLINE(num-pres)=allowed)
; The I option prevents overwriting the CONNECTEDLINE information
; set above when the call is answered.
exten => 1000,n,Dial(SIP/1000,20,I)
exten => 1000,n,Hangup
\end{verbatim}

\subsection{Use of interception macro}

\begin{verbatim}
[macro-add_pfx]
; ARG1 is the prefix to add.
; ARG2 is the number of digits at the end to add the prefix to.
; When the macro ends the CONNECTEDLINE data is passed to the
; channel driver.
exten => s,1,NoOp(Add prefix to connected line)
exten => s,n,Set(NOPREFIX=${CONNECTEDLINE(number):-${ARG2}})
exten => s,n,Set(CONNECTEDLINE(num,i)=${ARG1}${NOPREFIX})
exten => s,n,MacroExit

exten => 1000,1,NoOp
exten => 1000,n,Set(__CONNECTED_LINE_CALLER_SEND_MACRO=add_pfx)
exten => 1000,n,Set(__CONNECTED_LINE_CALLER_SEND_MACRO_ARGS=45,4)
exten => 1000,n,Dial(SIP/1000,20)
exten => 1000,n,Hangup
\end{verbatim}

\subsection{Simple redirection}
\begin{verbatim}
exten => 1000,1,NoOp
; For Q.SIG or ISDN point-to-point we should determine the COLR for this
; extension and send it if the call was redirected here.
exten => 1000,n,GotoIf($[${REDIRECTING(count)}>0]?redirected:notredirected)
exten => 1000,n(redirected),Set(REDIRECTING(to-num,i)=${CALLERID(dnid)})
exten => 1000,n,Set(REDIRECTING(to-num-pres)=allowed)
exten => 1000,n(notredirected),NoOp
; Determine that the destination has forwarded the call.
; ...
exten => 1000,n,Set(REDIRECTING(from-num,i)=1000)
exten => 1000,n,Set(REDIRECTING(from-num-pres,i)=allowed)
exten => 1000,n,Set(REDIRECTING(to-num,i)=2000)
; The DivertingLegInformation3 message is needed because at this point
; we do not know the presentation (COLR) setting of the redirecting-to
; party.
exten => 1000,n,Set(REDIRECTING(count,i)=$[${REDIRECTING(count)} + 1])
exten => 1000,n,Set(REDIRECTING(reason,i)=cfu)
; The call will update the redirecting-to presentation (COLR) when it
; becomes available with a redirecting update.
exten => 1000,n,Dial(DAHDI/g1/2000,20)
exten => 1000,n,Hangup
\end{verbatim}

\section{Ideas for usage}

The following is a list of ideas in which the manipulation of party ID
information would be beneficial.

\begin{itemize}

\item IVR that updates connected name on each selection made.

\item Disguise the true number of an individual with a generic company
number.

\item Use interception macros to make outbound connected number E.164
formatted.

\item You can do a lot more in an interception macro than just manipulate
party information...

\end{itemize}

\section{Troubleshooting tips}
\begin{itemize}

\item For CONNECTEDLINE and REDIRECTING, check the usage of the 'i'
option.

\item Check channel configuration settings.  The default settings may not
be what you want or expect.

\item Check packet captures.  Your equipment may not support what Asterisk
sends.

\end{itemize}

\section{For further reading...}
\begin{itemize}

\item Relevant ETSI ISDN redirecting specification: EN 300 207-1

\item Relevant ETSI ISDN COLP specification: EN 300 097-1

\item Relevant ETSI ISDN ECT specification: EN 300 369-1

\item Relevant Q.SIG ISDN redirecting specification: ECMA-174

\item Relevant Q.SIG ISDN COLP specification: ECMA-148

\item Relevant Q.SIG ISDN ECT specification: ECMA-178

\item Relevant SIP RFC for P-Asserted-Id: RFC3325

\item The expired draft (draft-ietf-sip-privacy-04.txt) defines
Remote-Party-Id.  Since Remote-Party-Id has not made it into an RFC at
this time, its use is non-standard by definition.

\end{itemize}
