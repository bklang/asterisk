

ODBC Storage allows you to store voicemail messages within a database 
instead of using a file.  This is \textbf{not} a full realtime engine and 
\textbf{only} supports ODBC.  The table description for the "voicemessages" 
table is as follows:

\begin{verbatim}
+----------------+-------------+------+-----+---------+-------+
| Field          | Type        | Null | Key | Default | Extra |
+----------------+-------------+------+-----+---------+-------+
| msgnum         | int(11)     | YES  |     | NULL    |       |
| dir            | varchar(80) | YES  | MUL | NULL    |       |
| context        | varchar(80) | YES  |     | NULL    |       |
| macrocontext   | varchar(80) | YES  |     | NULL    |       |
| callerid       | varchar(40) | YES  |     | NULL    |       |
| origtime       | varchar(40) | YES  |     | NULL    |       |
| duration       | varchar(20) | YES  |     | NULL    |       |
| mailboxuser    | varchar(80) | YES  |     | NULL    |       |
| mailboxcontext | varchar(80) | YES  |     | NULL    |       |
| recording      | longblob    | YES  |     | NULL    |       |
+----------------+-------------+------+-----+---------+-------+
\end{verbatim}

The database name (from \path{/etc/asterisk/res_odbc.conf}) is in the 
"odbcstorage" variable in the general section of voicemail.conf.

You may modify the voicemessages table name by using 
odbctable=??? in voicemail.conf.


