\section{Introduction}

There are two levels of parameter evaluation done in the Asterisk
dial plan in extensions.conf.
\begin{enumerate}
\item The first, and most frequently used, is the substitution of variable
  references with their values.
\item Then there are the evaluations of expressions done in \$[ .. ].
  This will be discussed below.
\end{enumerate}
Asterisk has user-defined variables and standard variables set
by various modules in Asterisk. These standard variables are
listed at the end of this document.

\section{Parameter Quoting}
\begin{astlisting}
\begin{verbatim}
exten => s,5,BackGround,blabla
\end{verbatim}
\end{astlisting}
The parameter (blabla) can be quoted ("blabla"). In this case, a
comma does not terminate the field. However, the double quotes
will be passed down to the Background command, in this example.

Also, characters special to variable substitution, expression evaluation, etc
(see below), can be quoted. For example, to literally use a \$ on the
string "\$1231", quote it with a preceding \textbackslash. Special characters that must
be quoted to be used, are [ ] \$ " \textbackslash. (to write \textbackslash itself, use \textbackslash).

These Double quotes and escapes are evaluated at the level of the
asterisk config file parser.

Double quotes can also be used inside expressions, as discussed below.

\section{Variables}

Parameter strings can include variables. Variable names are arbitrary strings.
They are stored in the respective channel structure.

To set a variable to a particular value, do:
\begin{astlisting}
\begin{verbatim}
    exten => 1,2,Set(varname=value)
\end{verbatim}
\end{astlisting}
You can substitute the value of a variable everywhere using \$\{variablename\}.
For example, to stringwise append \$lala to \$blabla and store result in \$koko,
do:
\begin{astlisting}
\begin{verbatim}
   exten => 1,2,Set(koko=${blabla}${lala})
\end{verbatim}
\end{astlisting}

There are two reference modes - reference by value and reference by name.
To refer to a variable with its name (as an argument to a function that
requires a variable), just write the name. To refer to the variable's value,
enclose it inside \$\{\}. For example, Set takes as the first argument
(before the =) a variable name, so:
\begin{astlisting}
\begin{verbatim}
  exten => 1,2,Set(koko=lala)
  exten => 1,3,Set(${koko}=blabla)
\end{verbatim}
\end{astlisting}
stores to the variable "koko" the value "lala" and to variable "lala" the
value "blabla".

In fact, everything contained \$\{here\} is just replaced with the value of
the variable "here".

\section{Variable Inheritance}

Variable names which are prefixed by "\_" will be inherited to channels
that are created in the process of servicing the original channel in
which the variable was set.  When the inheritance takes place, the
prefix will be removed in the channel inheriting the variable.  If the
name is prefixed by "\_\_" in the channel, then the variable is
inherited and the "\_\_" will remain intact in the new channel.

In the dialplan, all references to these variables refer to the same
variable, regardless of having a prefix or not.  Note that setting any
version of the variable removes any other version of the variable,
regardless of prefix.

\subsection{Example}
\begin{astlisting}
\begin{verbatim}
Set(__FOO=bar) ; Sets an inherited version of "FOO" variable
Set(FOO=bar)   ; Removes the inherited version and sets a local
               ; variable.
\end{verbatim}
\end{astlisting}

However, NoOp(\$\{\_\_FOO\}) is identical to NoOp(\$\{FOO\})

\section{Selecting Characters from Variables}

The format for selecting characters from a variable can be expressed as:
\begin{astlisting}
\begin{verbatim}
  ${variable_name[:offset[:length]]}
\end{verbatim}
\end{astlisting}
If you want to select the first N characters from the string assigned
to a variable, simply append a colon and the number of characters to
skip from the beginning of the string to the variable name.
\begin{astlisting}
\begin{verbatim}
  ; Remove the first character of extension, save in "number" variable
  exten => _9X.,1,Set(number=${EXTEN:1})
\end{verbatim}
\end{astlisting}
Assuming we've dialed 918005551234, the value saved to the 'number' variable
would be 18005551234. This is useful in situations when we require users to
dial a number to access an outside line, but do not wish to pass the first
digit.

If you use a negative offset number, Asterisk starts counting from the end
of the string and then selects everything after the new position. The following
example will save the numbers 1234 to the 'number' variable, still assuming
we've dialed 918005551234.
\begin{astlisting}
\begin{verbatim}
  ; Remove everything before the last four digits of the dialed string
  exten => _9X.,1,Set(number=${EXTEN:-4})
\end{verbatim}
\end{astlisting}
We can also limit the number of characters from our offset position that we
wish to use. This is done by appending a second colon and length value to the
variable name. The following example will save the numbers 555 to the 'number'
variable.
\begin{astlisting}
\begin{verbatim}
  ; Only save the middle numbers 555 from the string 918005551234
  exten => _9X.,1,Set(number=${EXTEN:5:3})
\end{verbatim}
\end{astlisting}
The length value can also be used in conjunction with a negative offset. This
may be useful if the length of the string is unknown, but the trailing digits
are. The following example will save the numbers 555 to the 'number' variable,
even if the string starts with more characters than expected (unlike the
previous example).
\begin{astlisting}
\begin{verbatim}
  ; Save the numbers 555 to the 'number' variable
  exten => _9X.,1,Set(number=${EXTEN:-7:3})
\end{verbatim}
\end{astlisting}
If a negative length value is entered, Asterisk will remove that many characters
from the end of the string.
\begin{astlisting}
\begin{verbatim}
  ; Set pin to everything but the trailing #.
  exten => _XXXX#,1,Set(pin=${EXTEN:0:-1})
\end{verbatim}
\end{astlisting}

\section{Expressions}

Everything contained inside a bracket pair prefixed by a \$ (like \$[this]) is
considered as an expression and it is evaluated. Evaluation works similar to
(but is done on a later stage than) variable substitution: the expression
(including the square brackets) is replaced by the result of the expression
evaluation.

For example, after the sequence:
\begin{astlisting}
\begin{verbatim}
exten => 1,1,Set(lala=$[1 + 2])
exten => 1,2,Set(koko=$[2 * ${lala}])
\end{verbatim}
\end{astlisting}
the value of variable koko is "6".

and, further:
\begin{astlisting}
\begin{verbatim}
exten => 1,1,Set,(lala=$[  1 +    2   ]);
\end{verbatim}
\end{astlisting}
will parse as intended. Extra spaces are ignored.


\subsection{Spaces Inside Variables Values}

If the variable being evaluated contains spaces, there can be problems.

For these cases, double quotes around text that may contain spaces
will force the surrounded text to be evaluated as a single token.
The double quotes will be counted as part of that lexical token.

As an example:

\begin{astlisting}
\begin{verbatim}
exten => s,6,GotoIf($[ "${CALLERID(name)}" : "Privacy Manager" ]?callerid-liar,s,1:s,7)
\end{verbatim}
\end{astlisting}

The variable CALLERID(name) could evaluate to "DELOREAN MOTORS" (with a space)
but the above will evaluate to:

\begin{verbatim}
"DELOREAN MOTORS" : "Privacy Manager"
\end{verbatim}

and will evaluate to 0.

The above without double quotes would have evaluated to:

\begin{verbatim}
DELOREAN MOTORS : Privacy Manager
\end{verbatim}

and will result in syntax errors, because token DELOREAN is immediately
followed by token MOTORS and the expression parser will not know how to
evaluate this expression, because it does not match its grammar.

\subsection{Operators}

Operators are listed below in order of increasing precedence.  Operators
with equal precedence are grouped within \{ \} symbols.

\begin{itemize}
  \item \verb!expr1 | expr2!

       Return the evaluation of expr1 if it is neither an empty string
       nor zero; otherwise, returns the evaluation of expr2.

  \item \verb!expr1 & expr2!

       Return the evaluation of expr1 if neither expression evaluates to
       an empty string or zero; otherwise, returns zero.

  \item \verb+expr1 {=, >, >=, <, <=, !=} expr2+

       Return the results of floating point comparison if both arguments are
       numbers; otherwise, returns the results of string comparison
       using the locale-specific collation sequence.  The result of each
       comparison is 1 if the specified relation is true, or 0 if the
       relation is false.

  \item \verb!expr1 {+, -} expr2!

       Return the results of addition or subtraction of floating point-valued
       arguments.

  \item \verb!expr1 {*, /, %} expr2!

       Return the results of multiplication, floating point division, or
       remainder of arguments.

  \item \verb!- expr1!

       Return the result of subtracting expr1 from 0.
       This, the unary minus operator, is right associative, and
       has the same precedence as the ! operator.

  \item \verb+! expr1+

       Return the result of a logical complement of expr1.
       In other words, if expr1 is null, 0, an empty string,
       or the string "0", return a 1. Otherwise, return a 0.
       It has the same precedence as the unary minus operator, and
       is also right associative.

  \item \verb!expr1 : expr2!

       The `:' operator matches expr1 against expr2, which must be a
       regular expression.  The regular expression is anchored to the
       beginning of  the string with an implicit `\^'.

       If the match succeeds and the pattern contains at least one regular
       expression subexpression `\(...\)', the string corresponing
       to `\textbackslash1' is returned; otherwise the matching operator
       returns the number of characters matched.  If the match fails and
       the pattern contains a regular expression subexpression the null
       string is returned; otherwise 0.

       Normally, the double quotes wrapping a string are left as part
       of the string. This is disastrous to the : operator. Therefore,
       before the regex match is made, beginning and ending double quote
       characters are stripped from both the pattern and the string.

   \item \verb!expr1 =~ expr2!

       Exactly the same as the ':' operator, except that the match is
       not anchored to the beginning of the string. Pardon any similarity
       to seemingly similar operators in other programming languages!
       The ":" and "=\verb!~!" operators share the same precedence.

   \item \verb!expr1 ? expr2 :: expr3!

       Traditional Conditional operator. If expr1 is a number
       that evaluates to 0 (false), expr3 is result of the this
       expression evaluation.  Otherwise, expr2 is the result.
       If expr1 is a string, and evaluates to an empty string,
       or the two characters (""), then expr3 is the
       result. Otherwise, expr2 is the result.  In Asterisk, all
       3 exprs will be "evaluated"; if expr1 is "true", expr2
       will be the result of the "evaluation" of this
       expression.  expr3 will be the result otherwise. This
       operator has the lowest precedence.

   \item \verb!expr1 ~~ expr2!

	Concatenation operator. The two exprs are evaluated and
	turned into strings, stripped of surrounding double quotes,
        and are turned into a single string with no invtervening spaces. 
        This operator is new to trunk after 1.6.0; it is not needed 
	in existing extensions.conf code. Because of the way asterisk
	evaluates ${ } and $[ ] constructs (recursively, bottom-
	up), no $[] or ${} is ever present when the contents
	of a ${} or $[] is evaluated. Thus, tokens are usually
	already merged at evaluation time. But, in AEL, various
	exprs are evaluated raw, and ${} and $[] are gathered
	and treated as tokens. And in AEL, no two tokens can
	sit side by side without an intervening operator.
	So, in AEL, concatenation must be explicitly specified
	in expressions. This new operator will play well into
	future plans, where expressions ($[] constructs, and
	variable references (${} constructs) are merged into a 
	single grammar. 

\end{itemize}

Parentheses are used for grouping in the usual manner.

Operator precedence is applied as one would expect in any of the C
or C derived languages.

\subsection{Floating Point Numbers}

In 1.6 and above, we shifted the \$[...] expressions to be calculated
via floating point numbers instead of integers. We use 'long double' numbers
when possible, which provide around 16 digits of precision with 12 byte numbers.

To specify a floating point constant, the number has to have this format: D.D, where D is
a string of base 10 digits. So, you can say 0.10, but you can't say .10 or 20.-- we hope
this is not an excessive restriction!

Floating point numbers are turned into strings via the '\%g'/'\%Lg' format of the printf
function set. This allows numbers to still 'look' like integers to those counting
on integer behavior. If you were counting on 1/4 evaluating to 0, you need to now say
TRUNC(1/4). For a list of all the truncation/rounding capabilities, see the next section.


\subsection{Functions}

In 1.6 and above, we upgraded the \$[] expressions to handle floating point numbers.
Because of this, folks counting on integer behavior would be disrupted. To make
the same results possible, some rounding and integer truncation functions have been
added to the core of the Expr2 parser. Indeed, dialplan functions can be called from
\$[..] expressions without the \$\{...\} operators. The only trouble might be in the fact that
the arguments to these functions must be specified with a comma. If you try to call
the MATH function, for example, and try to say 3 + MATH(7*8), the expression parser will
evaluate 7*8 for you into 56, and the MATH function will most likely complain that its
input doesn't make any sense.

We also provide access to most of the floating point functions in the C library. (but not all of them).

While we don't expect someone to want to do Fourier analysis in the dialplan, we
don't want to preclude it, either.

Here is a list of the 'builtin' functions in Expr2. All other dialplan functions
are available by simply calling them (read-only). In other words, you don't need to
surround function calls in \$[...] expressions with \$\{...\}. Don't jump to conclusions,
though! -- you still need to wrap variable names in curly braces!

\begin{enumerate}
\item COS(x) x is in radians. Results vary from -1 to 1.
\item SIN(x) x is in radians. Results vary from -1 to 1.
\item TAN(x) x is in radians.
\item ACOS(x) x should be a value between -1 and 1.
\item ASIN(x) x should be a value between -1 and 1.
\item ATAN(x) returns the arc tangent in radians; between -PI/2 and PI/2.
\item ATAN2(x,y) returns a result resembling y/x, except that the signs of both args are used to determine the quadrant of the result. Its result is in radians, between -PI and PI.
\item POW(x,y) returns the value of x raised to the power of y.
\item SQRT(x) returns the square root of x.
\item FLOOR(x) rounds x down to the nearest integer.
\item CEIL(x) rounds x up to the nearest integer.
\item ROUND(x) rounds x to the nearest integer, but round halfway cases away from zero.
\item RINT(x) rounds x to the nearest integer, rounding halfway cases to the nearest even integer.
\item TRUNC(x) rounds x to the nearest integer not larger in absolute value.
\item REMAINDER(x,y) computes the remainder of dividing x by y. The return value is x - n*y, where n is the value x/y, rounded to the nearest integer. If this quotient is 1/2, it is rounded to the nearest even number.
\item EXP(x) returns e to the x power.
\item EXP2(x) returns 2 to the x power.
\item LOG(x) returns the natural logarithm of x.
\item LOG2(x) returns the base 2 log of x.
\item LOG10(x) returns the base 10 log of x.
\end{enumerate}

\subsection{Examples}

\begin{astlisting}
\begin{verbatim}
 "One Thousand Five Hundred" =~ "(T[^ ]+)"
   returns: Thousand

 "One Thousand Five Hundred" =~ "T[^ ]+"
   returns: 8

 "One Thousand Five Hundred" : "T[^ ]+"
   returns: 0

 "8015551212" : "(...)"
   returns: 801

 "3075551212":"...(...)"
   returns: 555

 ! "One Thousand Five Hundred" =~ "T[^ ]+"
   returns: 0 (because it applies to the string, which is non-null,
                    which it turns to "0", and then looks for the pattern
                    in the "0", and doesn't find it)

 !( "One Thousand Five Hundred" : "T[^ ]+" )
   returns: 1  (because the string doesn't start with a word starting
                     with T, so the match evals to 0, and the ! operator
                     inverts it to 1 ).

 2 + 8 / 2
   returns 6. (because of operator precedence; the division is done first, then the addition).

 2+8/2
   returns 6. Spaces aren't necessary.

(2+8)/2
   returns 5, of course.

(3+8)/2
   returns 5.5 now.

TRUNC((3+8)/2)
   returns 5.

FLOOR(2.5)
   returns 2

FLOOR(-2.5)
   returns -3

CEIL(2.5)
   returns 3.

CEIL(-2.5)
   returns -2.

ROUND(2.5)
   returns 3.

ROUND(3.5)
   returns 4.

ROUND(-2.5)
   returns -3

RINT(2.5)
   returns 2.

RINT(3.5)
   returns 4.

RINT(-2.5)
   returns -2.

RINT(-3.5)
   returns -4.

TRUNC(2.5)
   returns 2.

TRUNC(3.5)
   returns 3.

TRUNC(-3.5)
   returns -3.
\end{verbatim}
\end{astlisting}

Of course, all of the above examples use constants, but would work the
same if any of the numeric or string constants were replaced with a
variable reference \$\{CALLERID(num)\}, for instance.


\subsection{Numbers Vs. Strings}

Tokens consisting only of numbers are converted to 'long double' if possible, which
are from 80 bits to 128 bits depending on the OS, compiler, and hardware.
This means that overflows can occur when the
numbers get above 18 digits (depending on the number of bits involved).  Warnings will appear in the logs in this
case.

\subsection{Conditionals}

There is one conditional application - the conditional goto :
\begin{astlisting}
\begin{verbatim}
  exten => 1,2,GotoIf(condition?label1:label2)
\end{verbatim}
\end{astlisting}

If condition is true go to label1, else go to label2. Labels are interpreted
exactly as in the normal goto command.

"condition" is just a string. If the string is empty or "0", the condition
is considered to be false, if it's anything else, the condition is true.
This is designed to be used together with the expression syntax described
above, eg :

\begin{astlisting}
\begin{verbatim}
  exten => 1,2,GotoIf($[${CALLERID(all)} = 123456]?2,1:3,1)
\end{verbatim}
\end{astlisting}

Example of use :
\begin{astlisting}
\begin{verbatim}
exten => s,2,Set(vara=1)
exten => s,3,Set(varb=$[${vara} + 2])
exten => s,4,Set(varc=$[${varb} * 2])
exten => s,5,GotoIf($[${varc} = 6]?99,1:s,6)
\end{verbatim}
\end{astlisting}

\subsection{Parse Errors}

Syntax errors are now output with 3 lines.

If the extensions.conf file contains a line like:

\begin{astlisting}
\begin{verbatim}
exten => s,6,GotoIf($[ "${CALLERID(num)}"  = "3071234567" & &  "${CALLERID(name)}" : "Privacy Manager" ]?callerid-liar,s,1:s,7)
\end{verbatim}
\end{astlisting}

You may see an error in \path{/var/log/asterisk/messages} like this:
\begin{astlisting}
\begin{verbatim}
Jul 15 21:27:49 WARNING[1251240752]: ast_yyerror(): syntax error: parse error, unexpected TOK_AND, expecting TOK_MINUS or TOK_LP or TOKEN; Input:
"3072312154"  = "3071234567" & & "Steves Extension" : "Privacy Manager"
               ^
\end{verbatim}
\end{astlisting}

The log line tells you that a syntax error was encountered. It now
also tells you (in grand standard bison format) that it hit an "AND"
(\&) token unexpectedly, and that was hoping for for a MINUS (-), LP
(left parenthesis), or a plain token (a string or number).

The next line shows the evaluated expression, and the line after
that, the position of the parser in the expression when it became confused,
marked with the "\^" character.

\subsection{NULL Strings}
Testing to see if a string is null can be done in one of two different ways:
\begin{astlisting}
\begin{verbatim}
  exten => _XX.,1,GotoIf($["${calledid}" != ""]?3)
    or
  exten => _XX.,1,GotoIf($[foo${calledid} != foo]?3)
\end{verbatim}
\end{astlisting}

The second example above is the way suggested by the WIKI. It will
work as long as there are no spaces in the evaluated value.

The first way should work in all cases, and indeed, might now
be the safest way to handle this situation.

\subsection{Warning}

If you need to do complicated things with strings, asterisk expressions
is most likely NOT the best way to go about it. AGI scripts are an
excellent option to this need, and make available the full power of
whatever language you desire, be it Perl, C, C++, Cobol, RPG, Java,
Snobol, PL/I, Scheme, Common Lisp, Shell scripts, Tcl, Forth, Modula,
Pascal, APL, assembler, etc.

\subsection{Incompatabilities}

The asterisk expression parser has undergone some evolution. It is hoped
that the changes will be viewed as positive.

The "original" expression parser had a simple, hand-written scanner,
and a simple bison grammar. This was upgraded to a more involved bison
grammar, and a hand-written scanner upgraded to allow extra spaces,
and to generate better error diagnostics. This upgrade required bison
1.85, and part of the user community felt the pain of having to
upgrade their bison version.

The next upgrade included new bison and flex input files, and the makefile
was upgraded to detect current version of both flex and bison, conditionally
compiling and linking the new files if the versions of flex and bison would
allow it.

If you have not touched your extensions.conf files in a year or so, the
above upgrades may cause you some heartburn in certain circumstances, as
several changes have been made, and these will affect asterisk's behavior on
legacy extension.conf constructs.  The changes have been engineered
to minimize these conflicts, but there are bound to be problems.

The following list gives some (and most likely, not all) of areas
of possible concern with "legacy" extension.conf files:

\begin{enumerate}
\item Tokens separated by space(s).
   Previously, tokens were separated by spaces. Thus, ' 1 + 1 ' would evaluate
   to the value '2', but '1+1' would evaluate to the string '1+1'. If this
   behavior was depended on, then the expression evaluation will break. '1+1'
   will now evaluate to '2', and something is not going to work right.
   To keep such strings from being evaluated, simply wrap them in double
   quotes: '  "1+1" '

\item The colon operator. In versions previous to double quoting, the
   colon operator takes the right hand string, and using it as a
   regex pattern, looks for it in the left hand string. It is given
   an implicit \^ operator at the beginning, meaning the pattern
   will match only at the beginning of the left hand string.
   If the pattern or the matching string had double quotes around
   them, these could get in the way of the pattern match. Now,
   the wrapping double quotes are stripped from both the pattern
   and the left hand string before applying the pattern. This
   was done because it recognized that the new way of
   scanning the expression doesn't use spaces to separate tokens,
   and the average regex expression is full of operators that
   the scanner will recognize as expression operators. Thus, unless
   the pattern is wrapped in double quotes, there will be trouble.
   For instance,      \$\{VAR1\} : (Who$|$What*)+
   may have have worked before, but unless you wrap the pattern
   in double quotes now, look out for trouble! This is better:
         "\$\{VAR1\}" : "(Who$|$What*)+"
   and should work as previous.

\item Variables and Double Quotes
   Before these changes, if a variable's value contained one or more double
   quotes, it was no reason for concern. It is now!

\item LE, GE, NE operators removed. The code supported these operators,
   but they were not documented. The symbolic operators, $<$=, $>$=, and !=
   should be used instead.

\item  Added the unary '-' operator. So you can 3+ -4 and get -1.

\item  Added the unary '!' operator, which is a logical complement.
    Basically, if the string or number is null, empty, or '0',
    a '1' is returned. Otherwise a '0' is returned.

\item  Added the '=\verb!~!' operator, just in case someone is just looking for
    match anywhere in the string. The only diff with the ':' is that
    match doesn't have to be anchored to the beginning of the string.

\item  Added the conditional operator  'expr1 ? true\_expr :: false\_expr'
    First, all 3 exprs are evaluated, and if expr1 is false, the 'false\_expr'
    is returned as the result. See above for details.

\item  Unary operators '-' and '!' were made right associative.
\end{enumerate}

\subsection{Debugging Hints}

There are two utilities you can build to help debug the \$[ ] in
your extensions.conf file.

The first, and most simplistic, is to issue the command:
\begin{astlisting}
\begin{verbatim}
make testexpr2
\end{verbatim}
\end{astlisting}
in the top level asterisk source directory. This will build a small
executable, that is able to take the first command line argument, and
run it thru the expression parser. No variable substitutions will be
performed. It might be safest to wrap the expression in single
quotes...
\begin{astlisting}
\begin{verbatim}
testexpr2 '2*2+2/2'
\end{verbatim}
\end{astlisting}
is an example.

And, in the utils directory, you can say:
\begin{astlisting}
\begin{verbatim}
make check_expr
\end{verbatim}
\end{astlisting}
and a small program will be built, that will check the file mentioned
in the first command line argument, for any expressions that might be
have problems when you move to flex-2.5.31.  It was originally
designed to help spot possible incompatibilities when moving from the
pre-2.5.31 world to the upgraded version of the lexer.

But one more capability has been added to check\_expr, that might make
it more generally useful. It now does a simple minded evaluation of
all variables, and then passes the \$[] exprs to the parser. If there
are any parse errors, they will be reported in the log file. You can
use check\_expr to do a quick sanity check of the expressions in your
extensions.conf file, to see if they pass a crude syntax check.

The "simple-minded" variable substitution replaces \$\{varname\} variable
references with '555'. You can override the 555 for variable values,
by entering in var=val arguments after the filename on the command
line.  So...
\begin{astlisting}
\begin{verbatim}
 check_expr /etc/asterisk/extensions.conf CALLERID(num)=3075551212 DIALSTATUS=TORTURE EXTEN=121
\end{verbatim}
\end{astlisting}
will substitute any \$\{CALLERID(num)\} variable references with
3075551212, any \$\{DIALSTATUS\} variable references with 'TORTURE', and
any \$\{EXTEN\} references with '121'.  If there is any fancy stuff
going on in the reference, like \$\{EXTEN:2\}, then the override will
not work. Everything in the \$\{...\} has to match. So, to substitute
\$\{EXTEN:2\} references, you'd best say:
\begin{astlisting}
\begin{verbatim}
 check_expr /etc/asterisk/extensions.conf CALLERID(num)=3075551212 DIALSTATUS=TORTURE EXTEN:2=121
\end{verbatim}
\end{astlisting}
on stdout, you will see something like:

\begin{astlisting}
\begin{verbatim}
 OK -- $[ "${DIALSTATUS}"  = "TORTURE" | "${DIALSTATUS}" = "DONTCALL" ] at line 416
\end{verbatim}
\end{astlisting}

In the expr2\_log file that is generated, you will see:

\begin{astlisting}
\begin{verbatim}
 line 416, evaluation of $[  "TORTURE"  = "TORTURE" | "TORTURE" = "DONTCALL"  ] result: 1
\end{verbatim}
\end{astlisting}

check\_expr is a very simplistic algorithm, and it is far from being
guaranteed to work in all cases, but it is hoped that it will be
useful.

\section{Asterisk standard channel variables}

There are a number of variables that are defined or read
by Asterisk. Here is a list of them. More information is
available in each application's help text. All these variables
are in UPPER CASE only.

Variables marked with a * are builtin functions and can't be set,
only read in the dialplan.  Writes to such variables are silently
ignored.

\begin{verbatim}
${CDR(accountcode)}    * Account code (if specified)
${BLINDTRANSFER}         The name of the channel on the other side of a blind transfer
${BRIDGEPEER}            Bridged peer
${BRIDGEPVTCALLID}       Bridged peer PVT call ID (SIP Call ID if a SIP call)
${CALLERID(ani)}       * Caller ANI (PRI channels)
${CALLERID(ani2)}      * ANI2 (Info digits) also called Originating line information or OLI
${CALLERID(all)}       * Caller ID
${CALLERID(dnid)}      * Dialed Number Identifier
${CALLERID(name)}      * Caller ID Name only
${CALLERID(num)}       * Caller ID Number only
${CALLERID(rdnis)}     * Redirected Dial Number ID Service
${CALLINGANI2}         * Caller ANI2 (PRI channels)
${CALLINGPRES}         * Caller ID presentation for incoming calls (PRI channels)
${CALLINGTNS}          * Transit Network Selector (PRI channels)
${CALLINGTON}          * Caller Type of Number (PRI channels)
${CHANNEL}             * Current channel name
${CONTEXT}             * Current context
${DATETIME}            * Current date time in the format: DDMMYYYY-HH:MM:SS
                         (Deprecated; use ${STRFTIME(${EPOCH},,%d%m%Y-%H:%M:%S)})
${DB_RESULT}             Result value of DB_EXISTS() dial plan function
${EPOCH}               * Current unix style epoch
${EXTEN}               * Current extension
${ENV(VAR)}              Environmental variable VAR
${GOTO_ON_BLINDXFR}      Transfer to the specified context/extension/priority
                         after a blind transfer (use ^ characters in place of
                         | to separate context/extension/priority when setting
                         this variable from the dialplan)
${HANGUPCAUSE}         * Asterisk cause of hangup (inbound/outbound)
${HINT}                * Channel hints for this extension
${HINTNAME}            * Suggested Caller*ID name for this extension
${INVALID_EXTEN}         The invalid called extension (used in the "i" extension)
${LANGUAGE}            * Current language (Deprecated; use ${LANGUAGE()})
${LEN(VAR)}            * String length of VAR (integer)
${PRIORITY}            * Current priority in the dialplan
${PRIREDIRECTREASON}     Reason for redirect on PRI, if a call was directed
${TIMESTAMP}           * Current date time in the format: YYYYMMDD-HHMMSS
                         (Deprecated; use ${STRFTIME(${EPOCH},,%Y%m%d-%H%M%S)})
${TRANSFER_CONTEXT}      Context for transferred calls
${FORWARD_CONTEXT}       Context for forwarded calls
${DYNAMIC_PEERNAME}      The name of the channel on the other side when a dynamic
                         feature is used.
${DYNAMIC_FEATURENAME}   The name of the last triggered dynamic feature.
${UNIQUEID}            * Current call unique identifier
${SYSTEMNAME}          * value of the systemname option of asterisk.conf
${ENTITYID}            * Global Entity ID set automatically, or from asterisk.conf
\end{verbatim}

\subsection{Application return values}

Many applications return the result in a variable that you read to
get the result of the application. These status fields are unique
for each application.
For the various status values, see each application's help text.
\begin{verbatim}
${AGISTATUS}         * agi()
${AQMSTATUS}         * addqueuemember()
${AVAILSTATUS}       * chanisavail()
${CHECKGROUPSTATUS}  * checkgroup()
${CHECKMD5STATUS}    * checkmd5()
${CPLAYBACKSTATUS}   * controlplayback()
${DIALSTATUS}        * dial()
${DBGETSTATUS}       * dbget()
${ENUMSTATUS}        * enumlookup()
${HASVMSTATUS}       * hasnewvoicemail()
${LOOKUPBLSTATUS}    * lookupblacklist()
${OSPAUTHSTATUS}     * ospauth()
${OSPLOOKUPSTATUS}   * osplookup()
${OSPNEXTSTATUS}     * ospnext()
${OSPFINISHSTATUS}   * ospfinish()
${PARKEDAT}          * parkandannounce()
${PLAYBACKSTATUS}    * playback()
${PQMSTATUS}         * pausequeuemember()
${PRIVACYMGRSTATUS}  * privacymanager()
${QUEUESTATUS}       * queue()
${RQMSTATUS}         * removequeuemember()
${SENDIMAGESTATUS}   * sendimage()
${SENDTEXTSTATUS}    * sendtext()
${SENDURLSTATUS}     * sendurl()
${SYSTEMSTATUS}      * system()
${TRANSFERSTATUS}    * transfer()
${TXTCIDNAMESTATUS}  * txtcidname()
${UPQMSTATUS}        * unpausequeuemember()
${VMSTATUS}          * voicmail()
${VMBOXEXISTSSTATUS} * vmboxexists()
${WAITSTATUS}        * waitforsilence()
\end{verbatim}

\subsection{Various application variables}
\begin{verbatim}
${CURL}                 * Resulting page content for curl()
${ENUM}                 * Result of application EnumLookup
${EXITCONTEXT}            Context to exit to in IVR menu (app background())
                          or in the RetryDial() application
${MONITOR}              * Set to "TRUE" if the channel is/has been monitored (app monitor())
${MONITOR_EXEC}           Application to execute after monitoring a call
${MONITOR_EXEC_ARGS}      Arguments to application
${MONITOR_FILENAME}       File for monitoring (recording) calls in queue
${QUEUE_PRIO}             Queue priority
${QUEUE_MAX_PENALTY}      Maximum member penalty allowed to answer caller
${QUEUE_MIN_PENALTY}      Minimum member penalty allowed to answer caller
${QUEUESTATUS}            Status of the call, one of:
                          (TIMEOUT | FULL | JOINEMPTY | LEAVEEMPTY | JOINUNAVAIL | LEAVEUNAVAIL)
${QUEUEPOSITION}        * When a caller is removed from a queue, his current position is logged
                          in this variable. If the value is 0, then this means that the caller was
						  serviced by a queue member. If non-zero, then this was the position in the
						  queue the caller was in when he left.
${RECORDED_FILE}        * Recorded file in record()
${TALK_DETECTED}        * Result from talkdetect()
${TOUCH_MONITOR}          The filename base to use with Touch Monitor (auto record)
${TOUCH_MONITOR_PREF}   * The prefix for automonitor recording filenames.
${TOUCH_MONITOR_FORMAT}   The audio format to use with Touch Monitor (auto record)
${TOUCH_MONITOR_OUTPUT} * Recorded file from Touch Monitor (auto record)
${TOUCH_MONITOR_MESSAGE_START} Recorded file to play for both channels at start of monitoring session
${TOUCH_MONITOR_MESSAGE_STOP} Recorded file to play for both channels at end of monitoring session
${TXTCIDNAME}           * Result of application TXTCIDName
${VPB_GETDTMF}            chan_vpb
\end{verbatim}

\subsection{The MeetMe Conference Bridge}
\begin{verbatim}
${MEETME_RECORDINGFILE}      Name of file for recording a conference with
                             the "r" option
${MEETME_RECORDINGFORMAT}    Format of file to be recorded
${MEETME_EXIT_CONTEXT}       Context for exit out of meetme meeting
${MEETME_AGI_BACKGROUND}     AGI script for Meetme (DAHDI only)
${MEETMESECS}              * Number of seconds a user participated in a MeetMe conference
${CONF_LIMIT_TIMEOUT_FILE}   File to play when time is up.  Used with the L() option.
${CONF_LIMIT_WARNING_FILE}   File to play as warning if 'y' is defined.
                             The default is to say the time remaining.  Used with the L() option.
${MEETMEBOOKID}            * This variable exposes the bookid column for a realtime configured conference bridge.
\end{verbatim}

\subsection{The VoiceMail() application}
\begin{verbatim}
${VM_CATEGORY}      Sets voicemail category
${VM_NAME}        * Full name in voicemail
${VM_DUR}         * Voicemail duration
${VM_MSGNUM}      * Number of voicemail message in mailbox
${VM_CALLERID}    * Voicemail Caller ID (Person leaving vm)
${VM_CIDNAME}     * Voicemail Caller ID Name
${VM_CIDNUM}      * Voicemail Caller ID Number
${VM_DATE}        * Voicemail Date
${VM_MESSAGEFILE} * Path to message left by caller
\end{verbatim}

\subsection{The VMAuthenticate() application}
\begin{verbatim}
${AUTH_MAILBOX}   * Authenticated mailbox
${AUTH_CONTEXT}   * Authenticated mailbox context
\end{verbatim}

\subsection{DUNDiLookup()}
\begin{verbatim}
${DUNDTECH}       * The Technology of the result from a call to DUNDiLookup()
${DUNDDEST}       * The Destination of the result from a call to DUNDiLookup()
\end{verbatim}

\subsection{chan\_dahdi}
\begin{verbatim}
${ANI2}               * The ANI2 Code provided by the network on the incoming call.
                        (ie, Code 29 identifies call as a Prison/Inmate Call)
${CALLTYPE}           * Type of call (Speech, Digital, etc)
${CALLEDTON}          * Type of number for incoming PRI extension
                        i.e. 0=unknown, 1=international, 2=domestic, 3=net_specific,
                        4=subscriber, 6=abbreviated, 7=reserved
${CALLINGSUBADDR}     * Caller's PRI Subaddress
${FAXEXTEN}           * The extension called before being redirected to "fax"
${PRIREDIRECTREASON}  * Reason for redirect, if a call was directed
${SMDI_VM_TYPE}       * When an call is received with an SMDI message, the 'type'
                        of message 'b' or 'u'
\end{verbatim}

\subsection{chan\_sip}
\begin{verbatim}
${SIPCALLID}         * SIP Call-ID: header verbatim (for logging or CDR matching)
${SIPDOMAIN}         * SIP destination domain of an inbound call (if appropriate)
${SIPFROMDOMAIN}       Set SIP domain on outbound calls
${SIPUSERAGENT}      * SIP user agent (deprecated)
${SIPURI}            * SIP uri
${SIP_MAX_FORWARDS}    Set the value of the Max-Forwards header for outbound call
${SIP_CODEC}           Set the SIP codec for an inbound call
${SIP_CODEC_INBOUND}   Set the SIP codec for an inbound call
${SIP_CODEC_OUTBOUND}  Set the SIP codec for an outbound call
${SIP_URI_OPTIONS}   * additional options to add to the URI for an outgoing call
${RTPAUDIOQOS}         RTCP QoS report for the audio of this call
${RTPVIDEOQOS}         RTCP QoS report for the video of this call
\end{verbatim}

\subsection{chan\_agent}
\begin{verbatim}
${AGENTMAXLOGINTRIES}  Set the maximum number of failed logins
${AGENTUPDATECDR}      Whether to update the CDR record with Agent channel data
${AGENTGOODBYE}        Sound file to use for "Good Bye" when agent logs out
${AGENTACKCALL}        Whether the agent should acknowledge the incoming call
${AGENTAUTOLOGOFF}     Auto logging off for an agent
${AGENTWRAPUPTIME}     Setting the time for wrapup between incoming calls
${AGENTNUMBER}       * Agent number (username) set at login
${AGENTSTATUS}       * Status of login ( fail | on | off )
${AGENTEXTEN}        * Extension for logged in agent
\end{verbatim}


\subsection{The Dial() application}
\begin{verbatim}
${DIALEDPEERNAME}     * Dialed peer name
${DIALEDPEERNUMBER}   * Dialed peer number
${DIALEDTIME}         * Time for the call (seconds). Is only set if call was answered.
${ANSWEREDTIME}       * Time from answer to hangup (seconds)
${DIALSTATUS}         * Status of the call, one of:
                        (CHANUNAVAIL | CONGESTION | BUSY | NOANSWER
                        | ANSWER | CANCEL | DONTCALL | TORTURE)
${DYNAMIC_FEATURES}   * The list of features (from the [applicationmap] section of
                        features.conf) to activate during the call, with feature
                        names separated by '#' characters
${LIMIT_PLAYAUDIO_CALLER}  Soundfile for call limits
${LIMIT_PLAYAUDIO_CALLEE}  Soundfile for call limits
${LIMIT_WARNING_FILE}      Soundfile for call limits
${LIMIT_TIMEOUT_FILE}      Soundfile for call limits
${LIMIT_CONNECT_FILE}      Soundfile for call limits
${OUTBOUND_GROUP}          Default groups for peer channels (as in SetGroup)
                           * See "show application dial" for more information
\end{verbatim}

\subsection{The chanisavail() application}
\begin{verbatim}
${AVAILCHAN}          * the name of the available channel if one was found
${AVAILORIGCHAN}      * the canonical channel name that was used to create the channel
${AVAILSTATUS}        * Status of requested channel
\end{verbatim}

\subsection{Dialplan Macros}
\begin{verbatim}
${MACRO_EXTEN}        * The calling extensions
${MACRO_CONTEXT}      * The calling context
${MACRO_PRIORITY}     * The calling priority
${MACRO_OFFSET}         Offset to add to priority at return from macro
\end{verbatim}

\subsection{The ChanSpy() application}
\begin{verbatim}
${SPYGROUP}           * A ':' (colon) separated list of group names.
                        (To be set on spied on channel and matched against the g(grp) option)
\end{verbatim}

\subsection{OSP}
\begin{verbatim}
${OSPINHANDLE}          The inbound call OSP transaction handle.
${OSPINTOKEN}           The inbound OSP token.
${OSPINTIMELIMIT}       The inbound call duration limit in seconds.
${OSPINPEERIP}          The last hop IP address.
${OSPINNETWORKID}       The inbound source network ID.
${OSPINNPRN}            The inbound routing number.
${OSPINNPCIC}           The inbound carrier identification code.
${OSPINNPDI}            The inbound number portability database dip indicator.
${OSPINSPID}            The inbound service provider identity.
${OSPINOCN}             The inbound operator company number.
${OSPINSPN}             The inbound service provider name.
${OSPINALTSPN}          The inbound alternate service provider name.
${OSPINMCC}             The inbound mobile country code.
${OSPINMNC}             The inbound mobile network code.
${OSPINDIVUSER}         The inbound Diversion header user part.
${OSPINDIVHOST}         The inbound Diversion header host part.
${OSPINTOHOST}          The inbound To header host part.
${OSPINCUSTOMINFOn}     The inbound custom information.
                        Where n is the index beginning with 1 upto 8.
${OSPOUTHANDLE}         The outbound call OSP transaction handle.
${OSPOUTTOKEN}          The outbound OSP token.
${OSPOUTTIMELIMIT}      The outbound call duration limit in seconds.
${OSPOUTTECH}           The outbound channel technology.
${OSPOUTCALLIDTYPES}    The outbound Call-ID types.
${OSPOUTCALLID}         The outbound Call-ID. Only for H.323.
${OSPDESTINATION}       The destination IP address.
${OSPDESTREMAILS}       The number of remained destinations.
${OSPOUTCALLING}        The outbound calling number.
${OSPOUTCALLED}         The outbound called number.
${OSPOUTNETWORKID}      The outbound destination network ID.
${OSPOUTNPRN}           The outbound routing number.
${OSPOUTNPCIC}          The outbound carrier identification code.
${OSPOUTNPDI}           The outbound number portability database dip indicator.
${OSPOUTSPID}           The outbound service provider identity.
${OSPOUTOCN}            The outbound operator company number.
${OSPOUTSPN}            The outbound service provider name.
${OSPOUTALTSPN}         The outbound alternate service provider name.
${OSPOUTMCC}            The outbound mobile country code.
${OSPOUTMNC}            The outbound mobile network code.
${OSPDIALSTR}           The outbound Dial command string.
${OSPINAUDIOQOS}        The inbound call leg audio QoS string.
${OSPOUTAUDIOQOS}       The outbound call leg audio QoS string.
\end{verbatim}

\subsection{Digit manipulation}
\begin{verbatim}
${REDIRECTING_CALLEE_SEND_MACRO}
  Macro to call before sending a redirecting update to the callee
${REDIRECTING_CALLEE_SEND_MACRO_ARGS}
  Arguments to pass to ${REDIRECTING_CALLEE_SEND_MACRO}
${REDIRECTING_CALLER_SEND_MACRO}
  Macro to call before sending a redirecting update to the caller
${REDIRECTING_CALLER_SEND_MACRO_ARGS}
  Arguments to pass to ${REDIRECTING_CALLER_SEND_MACRO}

${CONNECTED_LINE_CALLEE_SEND_MACRO}
  Macro to call before sending a connected line update to the callee
${CONNECTED_LINE_CALLEE_SEND_MACRO_ARGS}
  Arguments to pass to ${CONNECTED_LINE_CALLEE_SEND_MACRO}
${CONNECTED_LINE_CALLER_SEND_MACRO}
  Macro to call before sending a connected line update to the caller
${CONNECTED_LINE_CALLER_SEND_MACRO_ARGS}
  Arguments to pass to ${CONNECTED_LINE_CALLER_SEND_MACRO}
\end{verbatim}
