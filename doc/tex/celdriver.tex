\section{Storage Backends}

Right now, the CEL package will support CSV, Customized CSV, ODBC, PGSQL, TDS,
Sqlite3, and Radius back ends.  See the doc/celdriver.tex file
for how to use these back ends.

\subsection{Microsoft SQL Server}

	Asterisk can currently store Channel Events into an MSSQL database in
	two different ways:  cel_odbc or cel_tds

	Channel Event Records can be stored using unixODBC (which requires
	the FreeTDS package) [cel_odbc] or directly by using just the
	FreeTDS package [cel_tds]  The following provide some
	examples known to get asterisk working with mssql.

	NOTE:  Only choose one db connector.

\subsubsection{ODBC using cel_odbc}
	Compile, configure, and install the latest unixODBC package:
\begin{verbatim}
	tar -zxvf unixODBC-2.2.9.tar.gz &&
	cd unixODBC-2.2.9 &&
	./configure --sysconfdir=/etc --prefix=/usr --disable-gui &&
	make &&
	make install
\end{verbatim}

	Compile, configure, and install the latest FreeTDS package:
\begin{verbatim}
	tar -zxvf freetds-0.62.4.tar.gz &&
	cd freetds-0.62.4 &&
	./configure --prefix=/usr --with-tdsver=7.0 \
                 --with-unixodbc=/usr/lib &&
	make && make install
\end{verbatim}

	Compile, or recompile, asterisk so that it will now add support
	for cel_odbc.
\begin{verbatim}
	make clean && ./configure --with-odbc &&
	make update &&
	make &&
	make install
\end{verbatim}

	Setup odbc configuration files.  These are working examples
	from my system.  You will need to modify for your setup.
	You are not required to store usernames or passwords here.

\begin{verbatim}
	/etc/odbcinst.ini
	   [FreeTDS]
	   Description    = FreeTDS ODBC driver for MSSQL
	   Driver         = /usr/lib/libtdsodbc.so
	   Setup          = /usr/lib/libtdsS.so
	   FileUsage      = 1

	/etc/odbc.ini
	   [MSSQL-asterisk]
	   description         = Asterisk ODBC for MSSQL
	   driver              = FreeTDS
	   server              = 192.168.1.25
	   port                = 1433
	   database            = voipdb
	   tds_version         = 7.0
	   language            = us_english
\end{verbatim}

		Only install one database connector.  Do not confuse asterisk
		by using both ODBC (cel_odbc) and FreeTDS (cel_tds).
		This command will erase the contents of cel_tds.conf 
\begin{verbatim}
		[ -f /etc/asterisk/cel_tds.conf ] > /etc/asterisk/cel_tds.conf
\end{verbatim}
		NOTE:  unixODBC requires the freeTDS package, but asterisk does
		not call freeTDS directly.

		Now set up cel_odbc configuration files.  These are working samples
		from my system.  You will need to modify for your setup. Define
		your usernames and passwords here, secure file as well.
\begin{verbatim}
		/etc/asterisk/cel_odbc.conf
		   [global]
		   dsn=MSSQL-asterisk
		   username=voipdbuser
		   password=voipdbpass
		   loguniqueid=yes
\end{verbatim}
		And finally, create the 'cel' table in your mssql database.
\begin{verbatim}
		CREATE TABLE cel (
				[eventtype]     [varchar] (30)          NOT NULL ,
		        [eventtime]     [datetime]              NOT NULL ,
		        [cidname]       [varchar] (80)          NOT NULL ,
		        [cidnum]        [varchar] (80)          NOT NULL ,
		        [cidani]        [varchar] (80)          NOT NULL ,
		        [cidrdnis]      [varchar] (80)          NOT NULL ,
		        [ciddnid]       [varchar] (80)          NOT NULL ,
		        [exten]         [varchar] (80)          NOT NULL ,
		        [context]       [varchar] (80)          NOT NULL ,
		        [channame]      [varchar] (80)          NOT NULL ,
		        [appname]       [varchar] (80)          NOT NULL ,
		        [appdata]       [varchar] (80)          NOT NULL ,
		        [amaflags]      [int]                   NOT NULL ,
		        [accountcode]   [varchar] (20)          NOT NULL ,
		        [uniqueid]      [varchar] (32)          NOT NULL ,
		        [peer]          [varchar] (80)          NOT NULL ,
		        [userfield]     [varchar] (255)         NOT NULL
		)
\end{verbatim}
		Start asterisk in verbose mode, you should see that asterisk
		logs a connection to the database and will now record every
		desired channel event at the moment it occurs.

\subsubsection{FreeTDS, using cel_tds}
		Compile, configure, and install the latest FreeTDS package:
\begin{verbatim}
		   tar -zxvf freetds-0.62.4.tar.gz &&
		   cd freetds-0.62.4 &&
		   ./configure --prefix=/usr --with-tdsver=7.0
		   make &&
		   make install
\end{verbatim}
                Compile, or recompile, asterisk so that it will now add support
                for cel_tds.
\begin{verbatim}
                   make clean && ./configure --with-tds &&
                   make update &&
                   make &&
                   make install
\end{verbatim}
                Only install one database connector.  Do not confuse asterisk
                by using both ODBC (cel_odbc) and FreeTDS (cel_tds).
                This command will erase the contents of cel_odbc.conf
\begin{verbatim}
		[ -f /etc/asterisk/cel_odbc.conf ] > /etc/asterisk/cel_odbc.conf
\end{verbatim}
                Setup cel_tds configuration files.  These are working samples
                from my system.  You will need to modify for your setup. Define
                your usernames and passwords here, secure file as well.
\begin{verbatim}
                /etc/asterisk/cel_tds.conf
		   [global]
		   hostname=192.168.1.25
		   port=1433
		   dbname=voipdb
		   user=voipdbuser
		   password=voipdpass
		   charset=BIG5
\end{verbatim}
                And finally, create the 'cel' table in your mssql database.
\begin{verbatim}
		CREATE TABLE cel (
				[eventtype]     [varchar] (30)          NULL ,
		        [eventtime]     [datetime]              NULL ,
		        [cidname]       [varchar] (80)          NULL ,
		        [cidnum]        [varchar] (80)          NULL ,
		        [cidani]        [varchar] (80)          NULL ,
		        [cidrdnis]      [varchar] (80)          NULL ,
		        [ciddnid]       [varchar] (80)          NULL ,
		        [exten]         [varchar] (80)          NULL ,
		        [context]       [varchar] (80)          NULL ,
		        [channame]      [varchar] (80)          NULL ,
		        [appname]       [varchar] (80)          NULL ,
		        [appdata]       [varchar] (80)          NULL ,
		        [amaflags]      [varchar] (16)          NULL ,
		        [accountcode]   [varchar] (20)          NULL ,
		        [uniqueid]      [varchar] (32)          NULL ,
		        [userfield]     [varchar] (255)         NULL ,
		        [peer]          [varchar] (80)          NULL
		)
\end{verbatim}
                Start asterisk in verbose mode, you should see that asterisk
                logs a connection to the database and will now record every
                call to the database when it's complete.


\subsection{MySQL}

Using MySQL for Channel Event records is supported by using ODBC and the cel_odbc module.

\subsection{PostreSQL}
        If you want to go directly to postgresql database, and have the cel_pgsql.so
        compiled you can use the following sample setup.
        On Debian, before compiling asterisk, just install libpqxx-dev.
        Other distros will likely have a similiar package.

        Once you have the compile done,
        copy the sample cel_pgsql.conf file or create your own.

        Here is a sample:
\begin{verbatim}
        /etc/asterisk/cel_pgsql.conf
          ; Sample Asterisk config file for CEL logging to PostgresSQL
          [global]
          hostname=localhost
          port=5432
          dbname=asterisk
          password=password
          user=postgres
          table=cel
\end{verbatim}
        Now create a table in postgresql for your cels

\begin{verbatim}
        CREATE TABLE cel (
                id            serial ,
                eventtype     varchar (30)          NOT NULL ,
                eventtime     timestamp             NOT NULL ,
                userdeftype   varchar(255)          NOT NULL ,
                cid_name      varchar (80)          NOT NULL ,
                cid_num       varchar (80)          NOT NULL ,
                cid_ani       varchar (80)          NOT NULL ,
                cid_rdnis     varchar (80)          NOT NULL ,
                cid_dnid      varchar (80)          NOT NULL ,
                exten         varchar (80)          NOT NULL ,
                context       varchar (80)          NOT NULL ,
                channame      varchar (80)          NOT NULL ,
                appname       varchar (80)          NOT NULL ,
                appdata       varchar (80)          NOT NULL ,
                amaflags      int                   NOT NULL ,
                accountcode   varchar (20)          NOT NULL ,
                peeraccount   varchar (20)          NOT NULL ,
                uniqueid      varchar (150)         NOT NULL ,
                linkedid      varchar (150)         NOT NULL ,
                userfield     varchar (255)         NOT NULL ,
                peer          varchar (80)          NOT NULL
        );
\end{verbatim}

\subsection{SQLite 3}

SQLite version 3 is supported in cel_sqlite3_custom.

\subsection{RADIUS}

\subsubsection{What is needed}

\begin{itemize}
	\item FreeRADIUS server
	\item Radiusclient-ng library
	\item Asterisk PBX
\end{itemize}

\begin{figure}[h]
\begin{center}
\setlength{\unitlength}{4cm}
\begin{picture}(3,.75)
\put(0,0){\line(0,1){.75}}
\put(0,.75){\line(1,0){1.5}}
\put(1.5,0){\line(0,1){.75}}
\put(0,0){\line(1,0){1.5}}
\put(.1,.4){\makebox(1.3,.3){Asterisk PBX}}
\put(.1,.4){\line(1,0){1.3}}
\put(.1,.1){\line(1,0){1.3}}
\put(.1,.1){\line(0,1){.3}}
\put(1.4,.1){\line(0,1){.3}}
\put(.1,.1){\makebox(1.3,.3){RADIUS Client}}
\put(1.8,0){\line(0,1){.5}}
\put(1.8,.5){\line(1,0){1.1}}
\put(1.8,0){\line(1,0){1.1}}
\put(2.9,0){\line(0,1){.5}}
\put(1.8,.275){\makebox(1.1,.1){RADIUS Server}}
\put(1.8,.125){\makebox(1.1,.1){$(FreeRADIUS)$}}
\thicklines
\put(1.4,.3){\vector(1,0){.4}}
\put(1.8,.2){\vector(-1,0){.4}}
\thinlines
\end{picture}
\end{center}
\caption{Asterisk/RADIUS Integration}
\end{figure}

\subsubsection{Installation of the Radiusclient library}
   Installation:
\begin{verbatim}
	Download the sources from:

	http://developer.berlios.de/projects/radiusclient-ng/

	Untar the source tarball.
		root@localhost:/usr/local/src# tar xvfz radiusclient-ng-0.5.5.1.tar.gz

	Compile and install the library.
		root@localhost:/usr/local/src# cd radiusclient-ng-0.5.5.1
		root@localhost:/usr/local/src/radiusclient-ng-0.5.5.1# ./configure
		root@localhost:/usr/local/src/radiusclient-ng-0.5.5.1# make
		root@localhost:/usr/local/src/radiusclient-ng-0.5.5.1# make install
\end{verbatim}

\subsubsection{Configuration of the Radiusclient library}

	By default all the configuration files of the radiusclient library will
	be in /usr/local/etc/radiusclient-ng directory.

	File "radiusclient.conf"
		Open the file and find lines containing the following:

			authserver      localhost

	This is the hostname or IP address of the RADIUS server used for 
	authentication. You will have to change this unless the server is 
	running on the same host as your Asterisk PBX.

			acctserver      localhost
	This is the hostname or IP address of the RADIUS server used for 
	accounting. You will have to change this unless the server is running
	on the same host as your Asterisk PBX.

	File "servers" 

	RADIUS protocol uses simple access control mechanism based on shared
	secrets that allows RADIUS servers to limit access from RADIUS clients.

	A RADIUS server is configured with a secret string and only RADIUS 
	clients that have the same secret will be accepted.

	You need to configure a shared secret for each server you have 
	configured in radiusclient.conf file in the previous step. The shared 
	secrets are stored in /usr/local/etc/radiusclient-ng/servers file.

	Each line contains hostname of a RADIUS server and shared secret 
	used in communication with that server. The two values are separated 
	by white spaces. Configure shared secrets for every RADIUS server you 
	are going to use.

	File "dictionary"

	Asterisk uses some attributes that are not included in the 
	dictionary of radiusclient library, therefore it is necessary to add 
	them. A file called dictionary.digium (kept in the contrib dir)
	was created to list all new attributes used by Asterisk. 
	Add to the end of the main dictionary file
	/usr/local/etc/radiusclient-ng/dictionary
	the line:
\begin{verbatim}
		\$INCLUDE /path/to/dictionary.digium
\end{verbatim}

\subsubsection{Install FreeRADIUS Server (Version 1.1.1)}
 
	Download sources tarball from:

		http://freeradius.org/

	Untar, configure, build, and install the server:

\begin{verbatim}
	root@localhost:/usr/local/src# tar xvfz freeradius-1.1.1.tar.gz
	root@localhost:/usr/local/src# cd freeradius-1.1.1
	root@localhost"/usr/local/src/freeradius-1.1.1# ./configure
	root@localhost"/usr/local/src/freeradius-1.1.1# make
	root@localhost"/usr/local/src/freeradius-1.1.1# make install
\end{verbatim}

	All the configuration files of FreeRADIUS server will be in 
	/usr/local/etc/raddb directory. 


\subsubsection{Configuration of the FreeRADIUS Server}

	There are several files that have to be modified to configure the
	RADIUS server. These are presented next.

	File "clients.conf"

	File /usr/local/etc/raddb/clients.conf contains description of 
	RADIUS clients that are allowed to use the server. For each of the 
	clients you need to specify its hostname or IP address and also a 
	shared secret. The shared secret must be the same string you configured
	in radiusclient library.

	Example:
\begin{verbatim}
		client myhost {
		    secret = mysecret
		    shortname = foo
		}
\end{verbatim}	

	This fragment allows access from RADIUS clients on "myhost" if they use 
	"mysecret" as the shared secret.	 
	The file already contains an entry for localhost (127.0.0.1), so if you
	are running the RADIUS server on the same host as your Asterisk server,
	then modify the existing entry instead, replacing the default password.

	File "dictionary"

	Note : as of version 1.1.2, the dictionary.digium file ships with FreeRADIUS. 
	The following procedure brings the dictionary.digium file to previous versions 
	of FreeRADIUS.

	File /usr/local/etc/raddb/dictionary contains the dictionary of 
	FreeRADIUS server. You have to add the same dictionary file 
	(dictionary.digium), which you added to the dictionary of radiusclient-ng
	library. You can include it into the main file, adding the following line at the
	end of file '/usr/local/etc/raddb/dictionary':

	\$INCLUDE /path/to/dictionary.digium

	That will include the same new attribute definitions that are used 
	in radiusclient-ng library so the client and server will understand each 
	other. 


\subsubsection{Asterisk Accounting Configuration}

	Compilation and installation:

        The module will be compiled as long as the radiusclient-ng
        library has been detected on your system.

	By default FreeRADIUS server will log all accounting requests into 
	/usr/local/var/log/radius/radacct directory in form of plain text files. 
	The server will create one file for each hostname in the directory. The 
	following example shows how the log files look like. 

	Asterisk now generates Call Detail Records. See /include/asterisk/cel.h
	for all the fields which are recorded. By default, records in comma 
	separated values will be created in /var/log/asterisk/cel-csv. 

	The configuration file for cel_radius.so module is :

        /etc/asterisk/cel.conf 
	This is where you can set CEL related parameters as well as the path to
	the radiusclient-ng library configuration file.


\subsubsection{Logged Values}
\begin{verbatim}
  "Asterisk-Acc-Code",          The account name of detail records
  "Asterisk-CidName",
  "Asterisk-CidNum",
  "Asterisk-Cidani",
  "Asterisk-Cidrdnis",
  "Asterisk-Ciddnid",
  "Asterisk-Exten",
  "Asterisk-Context",           The destination context
  "Asterisk-Channame",          The channel name
  "Asterisk-Appname",	        Last application run on the channel 
  "Asterisk-App-Data",          Argument to the last channel 
  "Asterisk-Event-Time",        
  "Asterisk-Event-Type",        
  "Asterisk-AMA-Flags",       	DOCUMENTATION, BILL, IGNORE etc, specified on 
                                a per channel basis like accountcode. 
  "Asterisk-Unique-ID",         Unique call identifier 
  "Asterisk-User-Field"	        User field set via SetCELUserField 
  "Asterisk-Peer"	            Name of the Peer for 2-channel events (like bridge)
  
\end{verbatim}
