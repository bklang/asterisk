% This file is automatically generated by the "core dump clidocs" CLI command.  Any manual edits will be lost.
\section{!}
\subsection{Summary}
\begin{verbatim}
Execute a shell command
\end{verbatim}
\subsection{Usage}
\begin{verbatim}
Usage: !<command>
       Executes a given shell command

\end{verbatim}


\section{abort halt}
\subsection{Summary}
\begin{verbatim}
Cancel a running halt
\end{verbatim}
\subsection{Usage}
\begin{verbatim}
Usage: abort shutdown
       Causes Asterisk to abort an executing shutdown or restart, and resume normal
       call operations.

\end{verbatim}


\section{ael debug contexts}
\subsection{Summary}
\begin{verbatim}
Enable AEL contexts debug (does nothing)
\end{verbatim}
\subsection{Usage}
\begin{verbatim}
(null)
\end{verbatim}


\section{ael debug macros}
\subsection{Summary}
\begin{verbatim}
Enable AEL macros debug (does nothing)
\end{verbatim}
\subsection{Usage}
\begin{verbatim}
(null)
\end{verbatim}


\section{ael debug read}
\subsection{Summary}
\begin{verbatim}
Enable AEL read debug (does nothing)
\end{verbatim}
\subsection{Usage}
\begin{verbatim}
(null)
\end{verbatim}


\section{ael debug tokens}
\subsection{Summary}
\begin{verbatim}
Enable AEL tokens debug (does nothing)
\end{verbatim}
\subsection{Usage}
\begin{verbatim}
(null)
\end{verbatim}


\section{ael nodebug}
\subsection{Summary}
\begin{verbatim}
Disable AEL debug messages
\end{verbatim}
\subsection{Usage}
\begin{verbatim}
(null)
\end{verbatim}


\section{ael reload}
\subsection{Summary}
\begin{verbatim}
Reload AEL configuration
\end{verbatim}
\subsection{Usage}
\begin{verbatim}
(null)
\end{verbatim}


\section{agent logoff}
\subsection{Summary}
\begin{verbatim}
Sets an agent offline
\end{verbatim}
\subsection{Usage}
\begin{verbatim}
Usage: agent logoff <channel> [soft]
       Sets an agent as no longer logged in.
       If 'soft' is specified, do not hangup existing calls.

\end{verbatim}


\section{agent show}
\subsection{Summary}
\begin{verbatim}
Show status of agents
\end{verbatim}
\subsection{Usage}
\begin{verbatim}
Usage: agent show
       Provides summary information on agents.

\end{verbatim}


\section{agent show online}
\subsection{Summary}
\begin{verbatim}
Show all online agents
\end{verbatim}
\subsection{Usage}
\begin{verbatim}
Usage: agent show online
	Provides a list of all online agents.

\end{verbatim}


\section{agi debug}
\subsection{Summary}
\begin{verbatim}
Enable AGI debugging
\end{verbatim}
\subsection{Usage}
\begin{verbatim}
Usage: agi debug
       Enables dumping of AGI transactions for debugging purposes

\end{verbatim}


\section{agi debug off}
\subsection{Summary}
\begin{verbatim}
Disable AGI debugging
\end{verbatim}
\subsection{Usage}
\begin{verbatim}
Usage: agi debug off
       Disables dumping of AGI transactions for debugging purposes

\end{verbatim}


\section{agi dump commanddocs}
\subsection{Summary}
\begin{verbatim}
Dump agi command documentation in LaTeX format
\end{verbatim}
\subsection{Usage}
\begin{verbatim}
Usage: agi dump commanddocs [command]
       Dump manager action documentation to /tmp/ast_agi_commands.tex.

\end{verbatim}


\section{agi dumphtml}
\subsection{Summary}
\begin{verbatim}
Dumps a list of agi commands in html format
\end{verbatim}
\subsection{Usage}
\begin{verbatim}
Usage: agi dumphtml <filename>
	Dumps the agi command list in html format to given filename

\end{verbatim}


\section{agi show}
\subsection{Summary}
\begin{verbatim}
List AGI commands or specific help
\end{verbatim}
\subsection{Usage}
\begin{verbatim}
Usage: agi show [topic]
       When called with a topic as an argument, displays usage
       information on the given command.  If called without a
       topic, it provides a list of AGI commands.

\end{verbatim}


\section{cdr status}
\subsection{Summary}
\begin{verbatim}
Display the CDR status
\end{verbatim}
\subsection{Usage}
\begin{verbatim}
Usage: cdr status
	Displays the Call Detail Record engine system status.

\end{verbatim}


\section{console active}
\subsection{Summary}
\begin{verbatim}
Sets/displays active console
\end{verbatim}
\subsection{Usage}
\begin{verbatim}
Usage: console active [device]
       If used without a parameter, displays which device is the current
console.  If a device is specified, the console sound device is changed to
the device specified.

\end{verbatim}


\section{console answer}
\subsection{Summary}
\begin{verbatim}
Answer an incoming console call
\end{verbatim}
\subsection{Usage}
\begin{verbatim}
Usage: console answer
       Answers an incoming call on the console (OSS) channel.

\end{verbatim}


\section{console autoanswer [on|off]}
\subsection{Summary}
\begin{verbatim}
Sets/displays autoanswer
\end{verbatim}
\subsection{Usage}
\begin{verbatim}
Usage: console autoanswer [on|off]
       Enables or disables autoanswer feature.  If used without
       argument, displays the current on/off status of autoanswer.
       The default value of autoanswer is in 'oss.conf'.

\end{verbatim}


\section{console boost}
\subsection{Summary}
\begin{verbatim}
Sets/displays mic boost in dB
\end{verbatim}
\subsection{Usage}
\begin{verbatim}
(null)
\end{verbatim}


\section{console dial}
\subsection{Summary}
\begin{verbatim}
Dial an extension on the console
\end{verbatim}
\subsection{Usage}
\begin{verbatim}
Usage: console dial [extension[@context]]
       Dials a given extension (and context if specified)

\end{verbatim}


\section{console flash}
\subsection{Summary}
\begin{verbatim}
Flash a call on the console
\end{verbatim}
\subsection{Usage}
\begin{verbatim}
Usage: console flash
       Flashes the call currently placed on the console.

\end{verbatim}


\section{console hangup}
\subsection{Summary}
\begin{verbatim}
Hangup a call on the console
\end{verbatim}
\subsection{Usage}
\begin{verbatim}
Usage: console hangup
       Hangs up any call currently placed on the console.

\end{verbatim}


\section{console {mute|unmute}}
\subsection{Summary}
\begin{verbatim}
Disable/Enable mic input
\end{verbatim}
\subsection{Usage}
\begin{verbatim}
Usage: console {mute|unmute}
       Mute/unmute the microphone.

\end{verbatim}


\section{console send text}
\subsection{Summary}
\begin{verbatim}
Send text to the remote device
\end{verbatim}
\subsection{Usage}
\begin{verbatim}
Usage: console send text <message>
       Sends a text message for display on the remote terminal.

\end{verbatim}


\section{console transfer}
\subsection{Summary}
\begin{verbatim}
Transfer a call to a different extension
\end{verbatim}
\subsection{Usage}
\begin{verbatim}
Usage: console transfer <extension>[@context]
       Transfers the currently connected call to the given extension (and
context if specified)

\end{verbatim}


\section{core clear profile}
\subsection{Summary}
\begin{verbatim}
Clear profiling info
\end{verbatim}
\subsection{Usage}
\begin{verbatim}
(null)
\end{verbatim}


\section{core dump appdocs}
\subsection{Summary}
\begin{verbatim}
Dump application documentation in LaTeX format
\end{verbatim}
\subsection{Usage}
\begin{verbatim}
Usage: core dump appdocs [application]
       Dump Application documentation to /tmp/ast_appdocs.tex.

\end{verbatim}


\section{core dump clidocs}
\subsection{Summary}
\begin{verbatim}
Dump CLI command documentation in LaTeX format
\end{verbatim}
\subsection{Usage}
\begin{verbatim}
Usage: core dump clidocs
       Dump CLI command documentation to /tmp/ast_cli_commands.tex.

\end{verbatim}


\section{core dump funcdocs}
\subsection{Summary}
\begin{verbatim}
Dump function documentation in LaTeX format
\end{verbatim}
\subsection{Usage}
\begin{verbatim}
Usage: core dump funcdocs [function]
       Dump Application documentation to /tmp/ast_funcdocs.tex.

\end{verbatim}


\section{core set debug channel}
\subsection{Summary}
\begin{verbatim}
Enable/disable debugging on a channel
\end{verbatim}
\subsection{Usage}
\begin{verbatim}
Usage: core set debug channel <all|channel> [off]
       Enables/disables debugging on all or on a specific channel.

\end{verbatim}


\section{core set {debug|verbose} [off|atleast]}
\subsection{Summary}
\begin{verbatim}
Set level of debug/verbose chattiness
\end{verbatim}
\subsection{Usage}
\begin{verbatim}
Usage: core set {debug|verbose} [atleast] <level>
       core set {debug|verbose} off
       Sets level of debug or verbose messages to be displayed.
	0 or off means no messages should be displayed.
	Equivalent to -d[d[...]] or -v[v[v...]] on startup

\end{verbatim}


\section{core set global}
\subsection{Summary}
\begin{verbatim}
Set global dialplan variable
\end{verbatim}
\subsection{Usage}
\begin{verbatim}
Usage: core set global <name> <value>
       Set global dialplan variable <name> to <value>

\end{verbatim}


\section{core show applications}
\subsection{Summary}
\begin{verbatim}
Shows registered dialplan applications
\end{verbatim}
\subsection{Usage}
\begin{verbatim}
Usage: core show applications [{like|describing} <text>]
       List applications which are currently available.
       If 'like', <text> will be a substring of the app name
       If 'describing', <text> will be a substring of the description

\end{verbatim}


\section{core show application}
\subsection{Summary}
\begin{verbatim}
Describe a specific dialplan application
\end{verbatim}
\subsection{Usage}
\begin{verbatim}
Usage: core show application <application> [<application> [<application> [...]]]
       Describes a particular application.

\end{verbatim}


\section{core show channels [concise|verbose|count]}
\subsection{Summary}
\begin{verbatim}
Display information on channels
\end{verbatim}
\subsection{Usage}
\begin{verbatim}
Usage: core show channels [concise|verbose|count]
       Lists currently defined channels and some information about them. If
       'concise' is specified, the format is abridged and in a more easily
       machine parsable format. If 'verbose' is specified, the output includes
       more and longer fields. If 'count' is specified only the channel and call
       count is output.

\end{verbatim}


\section{core show channel}
\subsection{Summary}
\begin{verbatim}
Display information on a specific channel
\end{verbatim}
\subsection{Usage}
\begin{verbatim}
Usage: core show channel <channel>
       Shows lots of information about the specified channel.

\end{verbatim}


\section{core show channeltypes}
\subsection{Summary}
\begin{verbatim}
List available channel types
\end{verbatim}
\subsection{Usage}
\begin{verbatim}
Usage: core show channeltypes
       Lists available channel types registered in your Asterisk server.

\end{verbatim}


\section{core show channeltype}
\subsection{Summary}
\begin{verbatim}
Give more details on that channel type
\end{verbatim}
\subsection{Usage}
\begin{verbatim}
Usage: core show channeltype <name>
	Show details about the specified channel type, <name>.

\end{verbatim}


\section{core show codecs}
\subsection{Summary}
\begin{verbatim}
Displays a list of codecs
\end{verbatim}
\subsection{Usage}
\begin{verbatim}
Usage: core show codecs [audio|video|image]
       Displays codec mapping

\end{verbatim}


\section{core show codecs audio}
\subsection{Summary}
\begin{verbatim}
Displays a list of audio codecs
\end{verbatim}
\subsection{Usage}
\begin{verbatim}
Usage: core show codecs [audio|video|image]
       Displays codec mapping

\end{verbatim}


\section{core show codecs image}
\subsection{Summary}
\begin{verbatim}
Displays a list of image codecs
\end{verbatim}
\subsection{Usage}
\begin{verbatim}
Usage: core show codecs [audio|video|image]
       Displays codec mapping

\end{verbatim}


\section{core show codecs video}
\subsection{Summary}
\begin{verbatim}
Displays a list of video codecs
\end{verbatim}
\subsection{Usage}
\begin{verbatim}
Usage: core show codecs [audio|video|image]
       Displays codec mapping

\end{verbatim}


\section{core show codec}
\subsection{Summary}
\begin{verbatim}
Shows a specific codec
\end{verbatim}
\subsection{Usage}
\begin{verbatim}
Usage: core show codec <number>
       Displays codec mapping

\end{verbatim}


\section{core show config mappings}
\subsection{Summary}
\begin{verbatim}
Display config mappings (file names to config engines)
\end{verbatim}
\subsection{Usage}
\begin{verbatim}
Usage: core show config mappings
	Shows the filenames to config engines.

\end{verbatim}


\section{core show file formats}
\subsection{Summary}
\begin{verbatim}
Displays file formats
\end{verbatim}
\subsection{Usage}
\begin{verbatim}
Usage: core show file formats
       Displays currently registered file formats (if any)

\end{verbatim}


\section{core show file version}
\subsection{Summary}
\begin{verbatim}
List versions of files used to build Asterisk
\end{verbatim}
\subsection{Usage}
\begin{verbatim}
Usage: core show file version [like <pattern>]
       Lists the revision numbers of the files used to build this copy of Asterisk.
       Optional regular expression pattern is used to filter the file list.

\end{verbatim}


\section{core show functions}
\subsection{Summary}
\begin{verbatim}
Shows registered dialplan functions
\end{verbatim}
\subsection{Usage}
\begin{verbatim}
Usage: core show functions [like <text>]
       List builtin functions, optionally only those matching a given string

\end{verbatim}


\section{core show function}
\subsection{Summary}
\begin{verbatim}
Describe a specific dialplan function
\end{verbatim}
\subsection{Usage}
\begin{verbatim}
Usage: core show function <function>
       Describe a particular dialplan function.

\end{verbatim}


\section{core show globals}
\subsection{Summary}
\begin{verbatim}
Show global dialplan variables
\end{verbatim}
\subsection{Usage}
\begin{verbatim}
Usage: core show globals
       List current global dialplan variables and their values

\end{verbatim}


\section{core show hints}
\subsection{Summary}
\begin{verbatim}
Show dialplan hints
\end{verbatim}
\subsection{Usage}
\begin{verbatim}
Usage: core show hints
       List registered hints

\end{verbatim}


\section{core show image formats}
\subsection{Summary}
\begin{verbatim}
Displays image formats
\end{verbatim}
\subsection{Usage}
\begin{verbatim}
Usage: core show image formats
       displays currently registered image formats (if any)

\end{verbatim}


\section{core show license}
\subsection{Summary}
\begin{verbatim}
Show the license(s) for this copy of Asterisk
\end{verbatim}
\subsection{Usage}
\begin{verbatim}
Usage: core show license
       Shows the license(s) for this copy of Asterisk.

\end{verbatim}


\section{core show profile}
\subsection{Summary}
\begin{verbatim}
Display profiling info
\end{verbatim}
\subsection{Usage}
\begin{verbatim}
(null)
\end{verbatim}


\section{core show settings}
\subsection{Summary}
\begin{verbatim}
Show some core settings
\end{verbatim}
\subsection{Usage}
\begin{verbatim}
(null)
\end{verbatim}


\section{core show switches}
\subsection{Summary}
\begin{verbatim}
Show alternative switches
\end{verbatim}
\subsection{Usage}
\begin{verbatim}
Usage: core show switches
       List registered switches

\end{verbatim}


\section{core show sysinfo}
\subsection{Summary}
\begin{verbatim}
Show System Information
\end{verbatim}
\subsection{Usage}
\begin{verbatim}
Usage: core show sysinfo
       List current system information.

\end{verbatim}


\section{core show threads}
\subsection{Summary}
\begin{verbatim}
Show running threads
\end{verbatim}
\subsection{Usage}
\begin{verbatim}
Usage: core show threads
       List threads currently active in the system.

\end{verbatim}


\section{core show translation}
\subsection{Summary}
\begin{verbatim}
Display translation matrix
\end{verbatim}
\subsection{Usage}
\begin{verbatim}
Usage: core show translation [recalc] [<recalc seconds>]
       Displays known codec translators and the cost associated
with each conversion.  If the argument 'recalc' is supplied along
with optional number of seconds to test a new test will be performed
as the chart is being displayed.

\end{verbatim}


\section{core show uptime [seconds]}
\subsection{Summary}
\begin{verbatim}
Show uptime information
\end{verbatim}
\subsection{Usage}
\begin{verbatim}
Usage: core show uptime [seconds]
       Shows Asterisk uptime information.
       The seconds word returns the uptime in seconds only.

\end{verbatim}


\section{core show version}
\subsection{Summary}
\begin{verbatim}
Display version info
\end{verbatim}
\subsection{Usage}
\begin{verbatim}
Usage: core show version
       Shows Asterisk version information.

\end{verbatim}


\section{core show warranty}
\subsection{Summary}
\begin{verbatim}
Show the warranty (if any) for this copy of Asterisk
\end{verbatim}
\subsection{Usage}
\begin{verbatim}
Usage: core show warranty
       Shows the warranty (if any) for this copy of Asterisk.

\end{verbatim}


\section{database del}
\subsection{Summary}
\begin{verbatim}
Removes database key/value
\end{verbatim}
\subsection{Usage}
\begin{verbatim}
Usage: database del <family> <key>
       Deletes an entry in the Asterisk database for a given
family and key.

\end{verbatim}


\section{database deltree}
\subsection{Summary}
\begin{verbatim}
Removes database keytree/values
\end{verbatim}
\subsection{Usage}
\begin{verbatim}
Usage: database deltree <family> [keytree]
       Deletes a family or specific keytree within a family
in the Asterisk database.

\end{verbatim}


\section{database get}
\subsection{Summary}
\begin{verbatim}
Gets database value
\end{verbatim}
\subsection{Usage}
\begin{verbatim}
Usage: database get <family> <key>
       Retrieves an entry in the Asterisk database for a given
family and key.

\end{verbatim}


\section{database put}
\subsection{Summary}
\begin{verbatim}
Adds/updates database value
\end{verbatim}
\subsection{Usage}
\begin{verbatim}
Usage: database put <family> <key> <value>
       Adds or updates an entry in the Asterisk database for
a given family, key, and value.

\end{verbatim}


\section{database show}
\subsection{Summary}
\begin{verbatim}
Shows database contents
\end{verbatim}
\subsection{Usage}
\begin{verbatim}
Usage: database show [family [keytree]]
       Shows Asterisk database contents, optionally restricted
to a given family, or family and keytree.

\end{verbatim}


\section{database showkey}
\subsection{Summary}
\begin{verbatim}
Shows database contents
\end{verbatim}
\subsection{Usage}
\begin{verbatim}
Usage: database showkey <keytree>
       Shows Asterisk database contents, restricted to a given key.

\end{verbatim}


\section{dialplan add extension}
\subsection{Summary}
\begin{verbatim}
Add new extension into context
\end{verbatim}
\subsection{Usage}
\begin{verbatim}
Usage: dialplan add extension <exten>,<priority>,<app>,<app-data>
       into <context> [replace]

       This command will add new extension into <context>. If there is an
       existence of extension with the same priority and last 'replace'
       arguments is given here we simply replace this extension.

Example: dialplan add extension 6123,1,Dial,IAX/216.207.245.56/6123 into local
         Now, you can dial 6123 and talk to Markster :)

\end{verbatim}


\section{dialplan add ignorepat}
\subsection{Summary}
\begin{verbatim}
Add new ignore pattern
\end{verbatim}
\subsection{Usage}
\begin{verbatim}
Usage: dialplan add ignorepat <pattern> into <context>
       This command adds a new ignore pattern into context <context>

Example: dialplan add ignorepat _3XX into local

\end{verbatim}


\section{dialplan add include}
\subsection{Summary}
\begin{verbatim}
Include context in other context
\end{verbatim}
\subsection{Usage}
\begin{verbatim}
Usage: dialplan add include <context> into <context>
       Include a context in another context.

\end{verbatim}


\section{dialplan reload}
\subsection{Summary}
\begin{verbatim}
Reload extensions and *only* extensions
\end{verbatim}
\subsection{Usage}
\begin{verbatim}
Usage: dialplan reload
       reload extensions.conf without reloading any other modules
       This command does not delete global variables unless
       clearglobalvars is set to yes in extensions.conf

\end{verbatim}


\section{dialplan remove extension}
\subsection{Summary}
\begin{verbatim}
Remove a specified extension
\end{verbatim}
\subsection{Usage}
\begin{verbatim}
Usage: dialplan remove extension exten@context [priority]
       Remove an extension from a given context. If a priority
       is given, only that specific priority from the given extension
       will be removed.

\end{verbatim}


\section{dialplan remove ignorepat}
\subsection{Summary}
\begin{verbatim}
Remove ignore pattern from context
\end{verbatim}
\subsection{Usage}
\begin{verbatim}
Usage: dialplan remove ignorepat <pattern> from <context>
       This command removes an ignore pattern from context <context>

Example: dialplan remove ignorepat _3XX from local

\end{verbatim}


\section{dialplan remove include}
\subsection{Summary}
\begin{verbatim}
Remove a specified include from context
\end{verbatim}
\subsection{Usage}
\begin{verbatim}
Usage: dialplan remove include <context> from <context>
       Remove an included context from another context.

\end{verbatim}


\section{dialplan save}
\subsection{Summary}
\begin{verbatim}
Save dialplan
\end{verbatim}
\subsection{Usage}
\begin{verbatim}
Usage: dialplan save [/path/to/extension/file]
       Save dialplan created by pbx_config module.

Example: dialplan save                 (/etc/asterisk/extensions.conf)
         dialplan save /home/markster  (/home/markster/extensions.conf)

\end{verbatim}


\section{dialplan show}
\subsection{Summary}
\begin{verbatim}
Show dialplan
\end{verbatim}
\subsection{Usage}
\begin{verbatim}
Usage: core show dialplan [exten@][context]
       Show dialplan

\end{verbatim}


\section{dnsmgr reload}
\subsection{Summary}
\begin{verbatim}
Reloads the DNS manager configuration
\end{verbatim}
\subsection{Usage}
\begin{verbatim}
Usage: dnsmgr reload
       Reloads the DNS manager configuration.

\end{verbatim}


\section{dnsmgr status}
\subsection{Summary}
\begin{verbatim}
Display the DNS manager status
\end{verbatim}
\subsection{Usage}
\begin{verbatim}
Usage: dnsmgr status
       Displays the DNS manager status.

\end{verbatim}


\section{dundi debug}
\subsection{Summary}
\begin{verbatim}
Enable DUNDi debugging
\end{verbatim}
\subsection{Usage}
\begin{verbatim}
Usage: dundi debug
       Enables dumping of DUNDi packets for debugging purposes

\end{verbatim}


\section{dundi flush}
\subsection{Summary}
\begin{verbatim}
Flush DUNDi cache
\end{verbatim}
\subsection{Usage}
\begin{verbatim}
Usage: dundi flush [stats]
       Flushes DUNDi answer cache, used primarily for debug.  If
'stats' is present, clears timer statistics instead of normal
operation.

\end{verbatim}


\section{dundi lookup}
\subsection{Summary}
\begin{verbatim}
Lookup a number in DUNDi
\end{verbatim}
\subsection{Usage}
\begin{verbatim}
Usage: dundi lookup <number>[@context] [bypass]
       Lookup the given number within the given DUNDi context
(or e164 if none is specified).  Bypasses cache if 'bypass'
keyword is specified.

\end{verbatim}


\section{dundi no debug}
\subsection{Summary}
\begin{verbatim}
Disable DUNDi debugging
\end{verbatim}
\subsection{Usage}
\begin{verbatim}
Usage: dundi no debug
       Disables dumping of DUNDi packets for debugging purposes

\end{verbatim}


\section{dundi no store history}
\subsection{Summary}
\begin{verbatim}
Disable DUNDi historic records
\end{verbatim}
\subsection{Usage}
\begin{verbatim}
Usage: dundi no store history
       Disables storing of DUNDi requests and times for debugging
purposes

\end{verbatim}


\section{dundi precache}
\subsection{Summary}
\begin{verbatim}
Precache a number in DUNDi
\end{verbatim}
\subsection{Usage}
\begin{verbatim}
Usage: dundi precache <number>[@context]
       Lookup the given number within the given DUNDi context
(or e164 if none is specified) and precaches the results to any
upstream DUNDi push servers.

\end{verbatim}


\section{dundi query}
\subsection{Summary}
\begin{verbatim}
Query a DUNDi EID
\end{verbatim}
\subsection{Usage}
\begin{verbatim}
Usage: dundi query <entity>[@context]
       Attempts to retrieve contact information for a specific
DUNDi entity identifier (EID) within a given DUNDi context (or
e164 if none is specified).

\end{verbatim}


\section{dundi show entityid}
\subsection{Summary}
\begin{verbatim}
Display Global Entity ID
\end{verbatim}
\subsection{Usage}
\begin{verbatim}
Usage: dundi show entityid
       Displays the global entityid for this host.

\end{verbatim}


\section{dundi show mappings}
\subsection{Summary}
\begin{verbatim}
Show DUNDi mappings
\end{verbatim}
\subsection{Usage}
\begin{verbatim}
Usage: dundi show mappings
       Lists all known DUNDi mappings.

\end{verbatim}


\section{dundi show peers}
\subsection{Summary}
\begin{verbatim}
Show defined DUNDi peers
\end{verbatim}
\subsection{Usage}
\begin{verbatim}
Usage: dundi show peers
       Lists all known DUNDi peers.

\end{verbatim}


\section{dundi show peer}
\subsection{Summary}
\begin{verbatim}
Show info on a specific DUNDi peer
\end{verbatim}
\subsection{Usage}
\begin{verbatim}
Usage: dundi show peer [peer]
       Provide a detailed description of a specifid DUNDi peer.

\end{verbatim}


\section{dundi show precache}
\subsection{Summary}
\begin{verbatim}
Show DUNDi precache
\end{verbatim}
\subsection{Usage}
\begin{verbatim}
Usage: dundi show precache
       Lists all known DUNDi scheduled precache updates.

\end{verbatim}


\section{dundi show requests}
\subsection{Summary}
\begin{verbatim}
Show DUNDi requests
\end{verbatim}
\subsection{Usage}
\begin{verbatim}
Usage: dundi show requests
       Lists all known pending DUNDi requests.

\end{verbatim}


\section{dundi show trans}
\subsection{Summary}
\begin{verbatim}
Show active DUNDi transactions
\end{verbatim}
\subsection{Usage}
\begin{verbatim}
Usage: dundi show trans
       Lists all known DUNDi transactions.

\end{verbatim}


\section{dundi store history}
\subsection{Summary}
\begin{verbatim}
Enable DUNDi historic records
\end{verbatim}
\subsection{Usage}
\begin{verbatim}
Usage: dundi store history
       Enables storing of DUNDi requests and times for debugging
purposes

\end{verbatim}


\section{feature show}
\subsection{Summary}
\begin{verbatim}
Lists configured features
\end{verbatim}
\subsection{Usage}
\begin{verbatim}
Usage: feature list
       Lists currently configured features.

\end{verbatim}


\section{file convert}
\subsection{Summary}
\begin{verbatim}
Convert audio file
\end{verbatim}
\subsection{Usage}
\begin{verbatim}
Usage: file convert <file_in> <file_out>
    Convert from file_in to file_out. If an absolute path is not given, the
default Asterisk sounds directory will be used.

Example:
    file convert tt-weasels.gsm tt-weasels.ulaw

\end{verbatim}


\section{funcdevstate list}
\subsection{Summary}
\begin{verbatim}
List currently known custom device states
\end{verbatim}
\subsection{Usage}
\begin{verbatim}
Usage: funcdevstate list
       List all custom device states that have been set by using
       the DEVSTATE dialplan function.

\end{verbatim}


\section{group show channels}
\subsection{Summary}
\begin{verbatim}
Display active channels with group(s)
\end{verbatim}
\subsection{Usage}
\begin{verbatim}
Usage: group show channels [pattern]
       Lists all currently active channels with channel group(s) specified.
       Optional regular expression pattern is matched to group names for each
       channel.

\end{verbatim}


\section{gtalk reload}
\subsection{Summary}
\begin{verbatim}
Enable Jabber debugging
\end{verbatim}
\subsection{Usage}
\begin{verbatim}
Usage: gtalk reload
       Reload gtalk channel driver.

\end{verbatim}


\section{gtalk show channels}
\subsection{Summary}
\begin{verbatim}
Show GoogleTalk Channels
\end{verbatim}
\subsection{Usage}
\begin{verbatim}
Usage: gtalk show channels
       Shows current state of the Gtalk channels.

\end{verbatim}


\section{h323 cycle gk}
\subsection{Summary}
\begin{verbatim}
Manually re-register with the Gatekeper
\end{verbatim}
\subsection{Usage}
\begin{verbatim}
Usage: h323 gk cycle
       Manually re-register with the Gatekeper (Currently Disabled)

\end{verbatim}


\section{h323 hangup}
\subsection{Summary}
\begin{verbatim}
Manually try to hang up a call
\end{verbatim}
\subsection{Usage}
\begin{verbatim}
Usage: h323 hangup <token>
       Manually try to hang up call identified by <token>

\end{verbatim}


\section{h323 reload}
\subsection{Summary}
\begin{verbatim}
Reload H.323 configuration
\end{verbatim}
\subsection{Usage}
\begin{verbatim}
Usage: h323 reload
       Reloads H.323 configuration from h323.conf

\end{verbatim}


\section{h323 set debug}
\subsection{Summary}
\begin{verbatim}
Enable H.323 debug
\end{verbatim}
\subsection{Usage}
\begin{verbatim}
Usage: h323 debug
       Enables H.323 debug output

\end{verbatim}


\section{h323 set debug off}
\subsection{Summary}
\begin{verbatim}
Disable H.323 debug
\end{verbatim}
\subsection{Usage}
\begin{verbatim}
Usage: h323 no debug
       Disables H.323 debug output

\end{verbatim}


\section{h323 set trace}
\subsection{Summary}
\begin{verbatim}
Enable H.323 Stack Tracing
\end{verbatim}
\subsection{Usage}
\begin{verbatim}
Usage: h323 trace <level num>
       Enables H.323 stack tracing for debugging purposes

\end{verbatim}


\section{h323 set trace off}
\subsection{Summary}
\begin{verbatim}
Disable H.323 Stack Tracing
\end{verbatim}
\subsection{Usage}
\begin{verbatim}
Usage: h323 no trace
       Disables H.323 stack tracing for debugging purposes

\end{verbatim}


\section{h323 show tokens}
\subsection{Summary}
\begin{verbatim}
Show all active call tokens
\end{verbatim}
\subsection{Usage}
\begin{verbatim}
Usage: h323 show tokens
       Print out all active call tokens

\end{verbatim}


\section{help}
\subsection{Summary}
\begin{verbatim}
Display help list, or specific help on a command
\end{verbatim}
\subsection{Usage}
\begin{verbatim}
Usage: help [topic]
       When called with a topic as an argument, displays usage
       information on the given command. If called without a
       topic, it provides a list of commands.

\end{verbatim}


\section{http show status}
\subsection{Summary}
\begin{verbatim}
Display HTTP server status
\end{verbatim}
\subsection{Usage}
\begin{verbatim}
Usage: http show status
       Lists status of internal HTTP engine

\end{verbatim}


\section{iax2 provision}
\subsection{Summary}
\begin{verbatim}
Provision an IAX device
\end{verbatim}
\subsection{Usage}
\begin{verbatim}
Usage: iax2 provision <host> <template> [forced]
       Provisions the given peer or IP address using a template
       matching either 'template' or '*' if the template is not
       found.  If 'forced' is specified, even empty provisioning
       fields will be provisioned as empty fields.

\end{verbatim}


\section{iax2 prune realtime}
\subsection{Summary}
\begin{verbatim}
Prune a cached realtime lookup
\end{verbatim}
\subsection{Usage}
\begin{verbatim}
Usage: iax2 prune realtime [<peername>|all]
       Prunes object(s) from the cache

\end{verbatim}


\section{iax2 reload}
\subsection{Summary}
\begin{verbatim}
Reload IAX configuration
\end{verbatim}
\subsection{Usage}
\begin{verbatim}
Usage: iax2 reload
       Reloads IAX configuration from iax.conf

\end{verbatim}


\section{iax2 set debug}
\subsection{Summary}
\begin{verbatim}
Enable IAX debugging
\end{verbatim}
\subsection{Usage}
\begin{verbatim}
Usage: iax2 set debug
       Enables dumping of IAX packets for debugging purposes

\end{verbatim}


\section{iax2 set debug jb}
\subsection{Summary}
\begin{verbatim}
Enable IAX jitterbuffer debugging
\end{verbatim}
\subsection{Usage}
\begin{verbatim}
Usage: iax2 set debug jb
       Enables jitterbuffer debugging information

\end{verbatim}


\section{iax2 set debug jb off}
\subsection{Summary}
\begin{verbatim}
Disable IAX jitterbuffer debugging
\end{verbatim}
\subsection{Usage}
\begin{verbatim}
Usage: iax2 set debug jb off
       Disables jitterbuffer debugging information

\end{verbatim}


\section{iax2 set debug off}
\subsection{Summary}
\begin{verbatim}
Disable IAX debugging
\end{verbatim}
\subsection{Usage}
\begin{verbatim}
Usage: iax2 set debug off
       Disables dumping of IAX packets for debugging purposes

\end{verbatim}


\section{iax2 set debug trunk}
\subsection{Summary}
\begin{verbatim}
Enable IAX trunk debugging
\end{verbatim}
\subsection{Usage}
\begin{verbatim}
Usage: iax2 set debug trunk
       Requests current status of IAX trunking

\end{verbatim}


\section{iax2 set debug trunk off}
\subsection{Summary}
\begin{verbatim}
Disable IAX trunk debugging
\end{verbatim}
\subsection{Usage}
\begin{verbatim}
Usage: iax2 set debug trunk off
       Requests current status of IAX trunking

\end{verbatim}


\section{iax2 set mtu}
\subsection{Summary}
\begin{verbatim}
Set the IAX systemwide trunking MTU
\end{verbatim}
\subsection{Usage}
\begin{verbatim}
Usage: iax2 set mtu <value>
       Set the system-wide IAX IP mtu to <value> bytes net or zero to disable.
       Disabling means that the operating system must handle fragmentation of UDP packets
       when the IAX2 trunk packet exceeds the UDP payload size.
       This is substantially below the IP mtu. Try 1240 on ethernets.
       Must be 172 or greater for G.711 samples.

\end{verbatim}


\section{iax2 show cache}
\subsection{Summary}
\begin{verbatim}
Display IAX cached dialplan
\end{verbatim}
\subsection{Usage}
\begin{verbatim}
Usage: iax2 show cache
       Display currently cached IAX Dialplan results.

\end{verbatim}


\section{iax2 show channels}
\subsection{Summary}
\begin{verbatim}
List active IAX channels
\end{verbatim}
\subsection{Usage}
\begin{verbatim}
Usage: iax2 show channels
       Lists all currently active IAX channels.

\end{verbatim}


\section{iax2 show firmware}
\subsection{Summary}
\begin{verbatim}
List available IAX firmwares
\end{verbatim}
\subsection{Usage}
\begin{verbatim}
Usage: iax2 show firmware
       Lists all known IAX firmware images.

\end{verbatim}


\section{iax2 show netstats}
\subsection{Summary}
\begin{verbatim}
List active IAX channel netstats
\end{verbatim}
\subsection{Usage}
\begin{verbatim}
Usage: iax2 show netstats
       Lists network status for all currently active IAX channels.

\end{verbatim}


\section{iax2 show peers}
\subsection{Summary}
\begin{verbatim}
List defined IAX peers
\end{verbatim}
\subsection{Usage}
\begin{verbatim}
Usage: iax2 show peers [registered] [like <pattern>]
       Lists all known IAX2 peers.
       Optional 'registered' argument lists only peers with known addresses.
       Optional regular expression pattern is used to filter the peer list.

\end{verbatim}


\section{iax2 show peer}
\subsection{Summary}
\begin{verbatim}
Show details on specific IAX peer
\end{verbatim}
\subsection{Usage}
\begin{verbatim}
Usage: iax2 show peer <name>
       Display details on specific IAX peer

\end{verbatim}


\section{iax2 show provisioning}
\subsection{Summary}
\begin{verbatim}
Display iax provisioning
\end{verbatim}
\subsection{Usage}
\begin{verbatim}
Usage: iax list provisioning [template]
       Lists all known IAX provisioning templates or a
       specific one if specified.

\end{verbatim}


\section{iax2 show registry}
\subsection{Summary}
\begin{verbatim}
Display IAX registration status
\end{verbatim}
\subsection{Usage}
\begin{verbatim}
Usage: iax2 show registry
       Lists all registration requests and status.

\end{verbatim}


\section{iax2 show stats}
\subsection{Summary}
\begin{verbatim}
Display IAX statistics
\end{verbatim}
\subsection{Usage}
\begin{verbatim}
Usage: iax2 show stats
       Display statistics on IAX channel driver.

\end{verbatim}


\section{iax2 show threads}
\subsection{Summary}
\begin{verbatim}
Display IAX helper thread info
\end{verbatim}
\subsection{Usage}
\begin{verbatim}
Usage: iax2 show threads
       Lists status of IAX helper threads

\end{verbatim}


\section{iax2 show users}
\subsection{Summary}
\begin{verbatim}
List defined IAX users
\end{verbatim}
\subsection{Usage}
\begin{verbatim}
Usage: iax2 show users [like <pattern>]
       Lists all known IAX2 users.
       Optional regular expression pattern is used to filter the user list.

\end{verbatim}


\section{iax2 test losspct}
\subsection{Summary}
\begin{verbatim}
Set IAX2 incoming frame loss percentage
\end{verbatim}
\subsection{Usage}
\begin{verbatim}
Usage: iax2 test losspct <percentage>
       For testing, throws away <percentage> percent of incoming packets

\end{verbatim}


\section{iax2 unregister}
\subsection{Summary}
\begin{verbatim}
Unregister (force expiration) an IAX2 peer from the registry
\end{verbatim}
\subsection{Usage}
\begin{verbatim}
Usage: iax2 unregister <peername>
       Unregister (force expiration) an IAX2 peer from the registry.

\end{verbatim}


\section{indication add}
\subsection{Summary}
\begin{verbatim}
Add the given indication to the country
\end{verbatim}
\subsection{Usage}
\begin{verbatim}
Usage: indication add <country> <indication> "<tonelist>"
       Add the given indication to the country.

\end{verbatim}


\section{indication remove}
\subsection{Summary}
\begin{verbatim}
Remove the given indication from the country
\end{verbatim}
\subsection{Usage}
\begin{verbatim}
Usage: indication remove <country> <indication>
       Remove the given indication from the country.

\end{verbatim}


\section{indication show}
\subsection{Summary}
\begin{verbatim}
Display a list of all countries/indications
\end{verbatim}
\subsection{Usage}
\begin{verbatim}
Usage: indication show [<country> ...]
       Display either a condensed for of all country/indications, or the
       indications for the specified countries.

\end{verbatim}


\section{jabber debug}
\subsection{Summary}
\begin{verbatim}
Enable Jabber debugging
\end{verbatim}
\subsection{Usage}
\begin{verbatim}
Usage: jabber debug
       Enables dumping of Jabber packets for debugging purposes.

\end{verbatim}


\section{jabber debug off}
\subsection{Summary}
\begin{verbatim}
Disable Jabber debug
\end{verbatim}
\subsection{Usage}
\begin{verbatim}
Usage: jabber debug off
       Disables dumping of Jabber packets for debugging purposes.

\end{verbatim}


\section{jabber reload}
\subsection{Summary}
\begin{verbatim}
Reload Jabber configuration
\end{verbatim}
\subsection{Usage}
\begin{verbatim}
Usage: jabber reload
       Enables reloading of Jabber module.

\end{verbatim}


\section{jabber show connected}
\subsection{Summary}
\begin{verbatim}
Show state of clients and components
\end{verbatim}
\subsection{Usage}
\begin{verbatim}
Usage: jabber debug
       Enables dumping of Jabber packets for debugging purposes.

\end{verbatim}


\section{jabber test}
\subsection{Summary}
\begin{verbatim}
Shows roster, but is generally used for mog's debugging.
\end{verbatim}
\subsection{Usage}
\begin{verbatim}
Usage: jabber test [client]
       Sends test message for debugging purposes.  A specific client
       as configured in jabber.conf can be optionally specified.

\end{verbatim}


\section{keys init}
\subsection{Summary}
\begin{verbatim}
Initialize RSA key passcodes
\end{verbatim}
\subsection{Usage}
\begin{verbatim}
Usage: keys init
       Initializes private keys (by reading in pass code from the user)

\end{verbatim}


\section{keys show}
\subsection{Summary}
\begin{verbatim}
Displays RSA key information
\end{verbatim}
\subsection{Usage}
\begin{verbatim}
Usage: keys show
       Displays information about RSA keys known by Asterisk

\end{verbatim}


\section{local show channels}
\subsection{Summary}
\begin{verbatim}
List status of local channels
\end{verbatim}
\subsection{Usage}
\begin{verbatim}
Usage: local show channels
       Provides summary information on active local proxy channels.

\end{verbatim}


\section{logger mute}
\subsection{Summary}
\begin{verbatim}
Toggle logging output to a console
\end{verbatim}
\subsection{Usage}
\begin{verbatim}
Usage: logger mute
       Disables logging output to the current console, making it possible to
       gather information without being disturbed by scrolling lines.

\end{verbatim}


\section{logger reload}
\subsection{Summary}
\begin{verbatim}
Reopens the log files
\end{verbatim}
\subsection{Usage}
\begin{verbatim}
Usage: logger reload
       Reloads the logger subsystem state.  Use after restarting syslogd(8) if you are using syslog logging.

\end{verbatim}


\section{logger rotate}
\subsection{Summary}
\begin{verbatim}
Rotates and reopens the log files
\end{verbatim}
\subsection{Usage}
\begin{verbatim}
Usage: logger rotate
       Rotates and Reopens the log files.

\end{verbatim}


\section{logger show channels}
\subsection{Summary}
\begin{verbatim}
List configured log channels
\end{verbatim}
\subsection{Usage}
\begin{verbatim}
Usage: logger show channels
       List configured logger channels.

\end{verbatim}


\section{manager debug}
\subsection{Summary}
\begin{verbatim}
Show, enable, disable debugging of the manager code
\end{verbatim}
\subsection{Usage}
\begin{verbatim}
Usage: manager debug [on|off]
	Show, enable, disable debugging of the manager code.

\end{verbatim}


\section{manager dump actiondocs}
\subsection{Summary}
\begin{verbatim}
Dump manager action documentation in LaTeX format
\end{verbatim}
\subsection{Usage}
\begin{verbatim}
Usage: manager dump actiondocs [action]
       Dump manager action documentation to /tmp/ast_manager_actiondocs.tex.

\end{verbatim}


\section{manager show command}
\subsection{Summary}
\begin{verbatim}
Show a manager interface command
\end{verbatim}
\subsection{Usage}
\begin{verbatim}
Usage: manager show command <actionname>
	Shows the detailed description for a specific Asterisk manager interface command.

\end{verbatim}


\section{manager show commands}
\subsection{Summary}
\begin{verbatim}
List manager interface commands
\end{verbatim}
\subsection{Usage}
\begin{verbatim}
Usage: manager show commands
	Prints a listing of all the available Asterisk manager interface commands.

\end{verbatim}


\section{manager show connected}
\subsection{Summary}
\begin{verbatim}
List connected manager interface users
\end{verbatim}
\subsection{Usage}
\begin{verbatim}
Usage: manager show connected
	Prints a listing of the users that are currently connected to the
Asterisk manager interface.

\end{verbatim}


\section{manager show eventq}
\subsection{Summary}
\begin{verbatim}
List manager interface queued events
\end{verbatim}
\subsection{Usage}
\begin{verbatim}
Usage: manager show eventq
	Prints a listing of all events pending in the Asterisk manger
event queue.

\end{verbatim}


\section{manager show users}
\subsection{Summary}
\begin{verbatim}
List configured manager users
\end{verbatim}
\subsection{Usage}
\begin{verbatim}
Usage: manager show users
       Prints a listing of all managers that are currently configured on that
 system.

\end{verbatim}


\section{manager show user}
\subsection{Summary}
\begin{verbatim}
Display information on a specific manager user
\end{verbatim}
\subsection{Usage}
\begin{verbatim}
 Usage: manager show user <user>
        Display all information related to the manager user specified.

\end{verbatim}


\section{meetme}
\subsection{Summary}
\begin{verbatim}
Execute a command on a conference or conferee
\end{verbatim}
\subsection{Usage}
\begin{verbatim}
Usage: meetme (un)lock|(un)mute|kick|list [concise] <confno> <usernumber>
       Executes a command for the conference or on a conferee

\end{verbatim}


\section{memory show allocations}
\subsection{Summary}
\begin{verbatim}
Display outstanding memory allocations
\end{verbatim}
\subsection{Usage}
\begin{verbatim}
Usage: memory show allocations [<file>]
       Dumps a list of all segments of allocated memory, optionally
limited to those from a specific file

\end{verbatim}


\section{memory show summary}
\subsection{Summary}
\begin{verbatim}
Summarize outstanding memory allocations
\end{verbatim}
\subsection{Usage}
\begin{verbatim}
Usage: memory show summary [<file>]
       Summarizes heap memory allocations by file, or optionally
by function, if a file is specified

\end{verbatim}


\section{mgcp audit endpoint}
\subsection{Summary}
\begin{verbatim}
Audit specified MGCP endpoint
\end{verbatim}
\subsection{Usage}
\begin{verbatim}
Usage: mgcp audit endpoint <endpointid>
       Lists the capabilities of an endpoint in the MGCP (Media Gateway Control Protocol) subsystem.
       mgcp debug MUST be on to see the results of this command.

\end{verbatim}


\section{mgcp reload}
\subsection{Summary}
\begin{verbatim}
Reload MGCP configuration
\end{verbatim}
\subsection{Usage}
\begin{verbatim}
Usage: mgcp reload
       Reloads MGCP configuration from mgcp.conf
       Deprecated:  please use 'reload chan_mgcp.so' instead.

\end{verbatim}


\section{mgcp set debug}
\subsection{Summary}
\begin{verbatim}
Enable MGCP debugging
\end{verbatim}
\subsection{Usage}
\begin{verbatim}
Usage: mgcp set debug
       Enables dumping of MGCP packets for debugging purposes

\end{verbatim}


\section{mgcp set debug off}
\subsection{Summary}
\begin{verbatim}
Disable MGCP debugging
\end{verbatim}
\subsection{Usage}
\begin{verbatim}
Usage: mgcp set debug off
       Disables dumping of MGCP packets for debugging purposes

\end{verbatim}


\section{mgcp show endpoints}
\subsection{Summary}
\begin{verbatim}
List defined MGCP endpoints
\end{verbatim}
\subsection{Usage}
\begin{verbatim}
Usage: mgcp show endpoints
       Lists all endpoints known to the MGCP (Media Gateway Control Protocol) subsystem.

\end{verbatim}


\section{minivm list accounts}
\subsection{Summary}
\begin{verbatim}
List defined mini-voicemail boxes
\end{verbatim}
\subsection{Usage}
\begin{verbatim}
Usage: minivm list accounts
       Lists all mailboxes currently set up

\end{verbatim}


\section{minivm list templates}
\subsection{Summary}
\begin{verbatim}
List message templates
\end{verbatim}
\subsection{Usage}
\begin{verbatim}
Usage: minivm list templates
       Lists message templates for e-mail, paging and IM

\end{verbatim}


\section{minivm list zones}
\subsection{Summary}
\begin{verbatim}
List zone message formats
\end{verbatim}
\subsection{Usage}
\begin{verbatim}
Usage: minivm list zones
       Lists zone message formats

\end{verbatim}


\section{minivm reload}
\subsection{Summary}
\begin{verbatim}
Reload Mini-voicemail configuration
\end{verbatim}
\subsection{Usage}
\begin{verbatim}
Usage: minivm reload
       Reload mini-voicemail configuration and reset statistics

\end{verbatim}


\section{minivm show settings}
\subsection{Summary}
\begin{verbatim}
Show mini-voicemail general settings
\end{verbatim}
\subsection{Usage}
\begin{verbatim}
Usage: minivm show settings
       Display Mini-Voicemail general settings

\end{verbatim}


\section{minivm show stats}
\subsection{Summary}
\begin{verbatim}
Show some mini-voicemail statistics
\end{verbatim}
\subsection{Usage}
\begin{verbatim}
Usage: minivm show stats
       Display Mini-Voicemail counters

\end{verbatim}


\section{mixmonitor}
\subsection{Summary}
\begin{verbatim}
Execute a MixMonitor command.
\end{verbatim}
\subsection{Usage}
\begin{verbatim}
mixmonitor <start|stop> <chan_name> [args]

The optional arguments are passed to the
MixMonitor application when the 'start' command is used.

\end{verbatim}


\section{module load}
\subsection{Summary}
\begin{verbatim}
Load a module by name
\end{verbatim}
\subsection{Usage}
\begin{verbatim}
Usage: module load <module name>
       Loads the specified module into Asterisk.

\end{verbatim}


\section{module reload}
\subsection{Summary}
\begin{verbatim}
Reload configuration
\end{verbatim}
\subsection{Usage}
\begin{verbatim}
Usage: module reload [module ...]
       Reloads configuration files for all listed modules which support
       reloading, or for all supported modules if none are listed.

\end{verbatim}


\section{module show [like]}
\subsection{Summary}
\begin{verbatim}
List modules and info
\end{verbatim}
\subsection{Usage}
\begin{verbatim}
Usage: module show [like keyword]
       Shows Asterisk modules currently in use, and usage statistics.

\end{verbatim}


\section{module unload}
\subsection{Summary}
\begin{verbatim}
Unload a module by name
\end{verbatim}
\subsection{Usage}
\begin{verbatim}
Usage: module unload [-f|-h] <module_1> [<module_2> ... ]
       Unloads the specified module from Asterisk. The -f
       option causes the module to be unloaded even if it is
       in use (may cause a crash) and the -h module causes the
       module to be unloaded even if the module says it cannot, 
       which almost always will cause a crash.

\end{verbatim}


\section{moh reload}
\subsection{Summary}
\begin{verbatim}
Music On Hold
\end{verbatim}
\subsection{Usage}
\begin{verbatim}
Music On Hold
\end{verbatim}


\section{moh show classes}
\subsection{Summary}
\begin{verbatim}
List MOH classes
\end{verbatim}
\subsection{Usage}
\begin{verbatim}
Lists all MOH classes
\end{verbatim}


\section{moh show files}
\subsection{Summary}
\begin{verbatim}
List MOH file-based classes
\end{verbatim}
\subsection{Usage}
\begin{verbatim}
Lists all loaded file-based MOH classes and their files
\end{verbatim}


\section{no debug channel}
\subsection{Summary}
\begin{verbatim}
Disable debugging on channel(s)
\end{verbatim}
\subsection{Usage}
\begin{verbatim}
Usage: core set debug channel <all|channel> [off]
       Enables/disables debugging on all or on a specific channel.

\end{verbatim}


\section{odbc show}
\subsection{Summary}
\begin{verbatim}
List ODBC DSN(s)
\end{verbatim}
\subsection{Usage}
\begin{verbatim}
Usage: odbc show [<class>]
       List settings of a particular ODBC class.
       or, if not specified, all classes.

\end{verbatim}


\section{originate}
\subsection{Summary}
\begin{verbatim}
Originate a call
\end{verbatim}
\subsection{Usage}
\begin{verbatim}
  There are two ways to use this command. A call can be originated between a
channel and a specific application, or between a channel and an extension in
the dialplan. This is similar to call files or the manager originate action.
Calls originated with this command are given a timeout of 30 seconds.

Usage1: originate <tech/data> application <appname> [appdata]
  This will originate a call between the specified channel tech/data and the
given application. Arguments to the application are optional. If the given
arguments to the application include spaces, all of the arguments to the
application need to be placed in quotation marks.

Usage2: originate <tech/data> extension [exten@][context]
  This will originate a call between the specified channel tech/data and the
given extension. If no context is specified, the 'default' context will be
used. If no extension is given, the 's' extension will be used.

\end{verbatim}


\section{parkedcalls show}
\subsection{Summary}
\begin{verbatim}
List currently parked calls
\end{verbatim}
\subsection{Usage}
\begin{verbatim}
Usage: parkedcalls show
       List currently parked calls

\end{verbatim}


\section{queue add member}
\subsection{Summary}
\begin{verbatim}
Add a channel to a specified queue
\end{verbatim}
\subsection{Usage}
\begin{verbatim}
Usage: queue add member <channel> to <queue> [penalty <penalty>]

\end{verbatim}


\section{queue remove member}
\subsection{Summary}
\begin{verbatim}
Removes a channel from a specified queue
\end{verbatim}
\subsection{Usage}
\begin{verbatim}
Usage: queue remove member <channel> from <queue>

\end{verbatim}


\section{queue show}
\subsection{Summary}
\begin{verbatim}
Show status of a specified queue
\end{verbatim}
\subsection{Usage}
\begin{verbatim}
Usage: queue show
       Provides summary information on a specified queue.

\end{verbatim}


\section{realtime load}
\subsection{Summary}
\begin{verbatim}
Used to print out RealTime variables.
\end{verbatim}
\subsection{Usage}
\begin{verbatim}
Usage: realtime load <family> <colmatch> <value>
       Prints out a list of variables using the RealTime driver.

\end{verbatim}


\section{realtime update}
\subsection{Summary}
\begin{verbatim}
Used to update RealTime variables.
\end{verbatim}
\subsection{Usage}
\begin{verbatim}
Usage: realtime update <family> <colmatch> <value>
       Update a single variable using the RealTime driver.

\end{verbatim}


\section{restart gracefully}
\subsection{Summary}
\begin{verbatim}
Restart Asterisk gracefully
\end{verbatim}
\subsection{Usage}
\begin{verbatim}
Usage: restart gracefully
       Causes Asterisk to stop accepting new calls and exec() itself performing a cold
       restart when all active calls have ended.

\end{verbatim}


\section{restart now}
\subsection{Summary}
\begin{verbatim}
Restart Asterisk immediately
\end{verbatim}
\subsection{Usage}
\begin{verbatim}
Usage: restart now
       Causes Asterisk to hangup all calls and exec() itself performing a cold
       restart.

\end{verbatim}


\section{restart when convenient}
\subsection{Summary}
\begin{verbatim}
Restart Asterisk at empty call volume
\end{verbatim}
\subsection{Usage}
\begin{verbatim}
Usage: restart when convenient
       Causes Asterisk to perform a cold restart when all active calls have ended.

\end{verbatim}


\section{rpt debug level}
\subsection{Summary}
\begin{verbatim}
Enable app_rpt debugging
\end{verbatim}
\subsection{Usage}
\begin{verbatim}
Usage: rpt debug level {0-7}
       Enables debug messages in app_rpt

\end{verbatim}


\section{rpt dump}
\subsection{Summary}
\begin{verbatim}
Dump app_rpt structs for debugging
\end{verbatim}
\subsection{Usage}
\begin{verbatim}
Usage: rpt dump <nodename>
       Dumps struct debug info to log

\end{verbatim}


\section{rpt lstats}
\subsection{Summary}
\begin{verbatim}
Dump link statistics
\end{verbatim}
\subsection{Usage}
\begin{verbatim}
Usage: rpt lstats <nodename>
       Dumps link statistics to console

\end{verbatim}


\section{rpt reload}
\subsection{Summary}
\begin{verbatim}
Reload app_rpt config
\end{verbatim}
\subsection{Usage}
\begin{verbatim}
Usage: rpt reload
       Reloads app_rpt running config parameters

\end{verbatim}


\section{rpt restart}
\subsection{Summary}
\begin{verbatim}
Restart app_rpt
\end{verbatim}
\subsection{Usage}
\begin{verbatim}
Usage: rpt restart
       Restarts app_rpt

\end{verbatim}


\section{rpt stats}
\subsection{Summary}
\begin{verbatim}
Dump node statistics
\end{verbatim}
\subsection{Usage}
\begin{verbatim}
Usage: rpt stats <nodename>
       Dumps node statistics to console

\end{verbatim}


\section{rtcp debug ip}
\subsection{Summary}
\begin{verbatim}
Enable RTCP debugging on IP
\end{verbatim}
\subsection{Usage}
\begin{verbatim}
Usage: rtcp debug [ip host[:port]]
       Enable dumping of all RTCP packets to and from host.

\end{verbatim}


\section{rtcp debug}
\subsection{Summary}
\begin{verbatim}
Enable RTCP debugging
\end{verbatim}
\subsection{Usage}
\begin{verbatim}
Usage: rtcp debug [ip host[:port]]
       Enable dumping of all RTCP packets to and from host.

\end{verbatim}


\section{rtcp debug off}
\subsection{Summary}
\begin{verbatim}
Disable RTCP debugging
\end{verbatim}
\subsection{Usage}
\begin{verbatim}
Usage: rtcp debug off
       Disable all RTCP debugging

\end{verbatim}


\section{rtcp stats}
\subsection{Summary}
\begin{verbatim}
Enable RTCP stats
\end{verbatim}
\subsection{Usage}
\begin{verbatim}
Usage: rtcp stats
       Enable dumping of RTCP stats.

\end{verbatim}


\section{rtcp stats off}
\subsection{Summary}
\begin{verbatim}
Disable RTCP stats
\end{verbatim}
\subsection{Usage}
\begin{verbatim}
Usage: rtcp stats off
       Disable all RTCP stats

\end{verbatim}


\section{rtp debug ip}
\subsection{Summary}
\begin{verbatim}
Enable RTP debugging on IP
\end{verbatim}
\subsection{Usage}
\begin{verbatim}
Usage: rtp debug [ip host[:port]]
       Enable dumping of all RTP packets to and from host.

\end{verbatim}


\section{rtp debug}
\subsection{Summary}
\begin{verbatim}
Enable RTP debugging
\end{verbatim}
\subsection{Usage}
\begin{verbatim}
Usage: rtp debug [ip host[:port]]
       Enable dumping of all RTP packets to and from host.

\end{verbatim}


\section{rtp debug off}
\subsection{Summary}
\begin{verbatim}
Disable RTP debugging
\end{verbatim}
\subsection{Usage}
\begin{verbatim}
Usage: rtp debug off
       Disable all RTP debugging

\end{verbatim}


\section{say load}
\subsection{Summary}
\begin{verbatim}
set/show the say mode
\end{verbatim}
\subsection{Usage}
\begin{verbatim}
say load new|old
\end{verbatim}


\section{sip history}
\subsection{Summary}
\begin{verbatim}
Enable SIP history
\end{verbatim}
\subsection{Usage}
\begin{verbatim}
Usage: sip history
       Enables recording of SIP dialog history for debugging purposes.
Use 'sip show history' to view the history of a call number.

\end{verbatim}


\section{sip history off}
\subsection{Summary}
\begin{verbatim}
Disable SIP history
\end{verbatim}
\subsection{Usage}
\begin{verbatim}
Usage: sip history off
       Disables recording of SIP dialog history for debugging purposes

\end{verbatim}


\section{sip notify}
\subsection{Summary}
\begin{verbatim}
Send a notify packet to a SIP peer
\end{verbatim}
\subsection{Usage}
\begin{verbatim}
Usage: sip notify <type> <peer> [<peer>...]
       Send a NOTIFY message to a SIP peer or peers
       Message types are defined in sip_notify.conf

\end{verbatim}


\section{sip prune realtime}
\subsection{Summary}
\begin{verbatim}
Prune cached Realtime object(s)
\end{verbatim}
\subsection{Usage}
\begin{verbatim}
Usage: sip prune realtime [peer|user] [<name>|all|like <pattern>]
       Prunes object(s) from the cache.
       Optional regular expression pattern is used to filter the objects.

\end{verbatim}


\section{sip prune realtime peer}
\subsection{Summary}
\begin{verbatim}
Prune cached Realtime peer(s)
\end{verbatim}
\subsection{Usage}
\begin{verbatim}
Usage: sip prune realtime [peer|user] [<name>|all|like <pattern>]
       Prunes object(s) from the cache.
       Optional regular expression pattern is used to filter the objects.

\end{verbatim}


\section{sip prune realtime user}
\subsection{Summary}
\begin{verbatim}
Prune cached Realtime user(s)
\end{verbatim}
\subsection{Usage}
\begin{verbatim}
Usage: sip prune realtime [peer|user] [<name>|all|like <pattern>]
       Prunes object(s) from the cache.
       Optional regular expression pattern is used to filter the objects.

\end{verbatim}


\section{sip reload}
\subsection{Summary}
\begin{verbatim}
Reload SIP configuration
\end{verbatim}
\subsection{Usage}
\begin{verbatim}
Usage: sip reload
       Reloads SIP configuration from sip.conf

\end{verbatim}


\section{sip set debug}
\subsection{Summary}
\begin{verbatim}
Enable SIP debugging
\end{verbatim}
\subsection{Usage}
\begin{verbatim}
Usage: sip set debug {on|off|ip <host[:PORT]>|peer <peername>}
       sip set debug on
          Enables dumping of all SIP messages for debugging purposes

       sip set debug off
          Disables dumping of all SIP messages

       sip set debug ip <host[:PORT]>
          Enables dumping of SIP messages to and from host.

       sip set debug peer <peername>
          Enables dumping of SIP messages to and from peer's IP.
          Requires peer to be registered.

\end{verbatim}


\section{sip set debug ip}
\subsection{Summary}
\begin{verbatim}
Enable SIP debugging on IP
\end{verbatim}
\subsection{Usage}
\begin{verbatim}
Usage: sip set debug {on|off|ip <host[:PORT]>|peer <peername>}
       sip set debug on
          Enables dumping of all SIP messages for debugging purposes

       sip set debug off
          Disables dumping of all SIP messages

       sip set debug ip <host[:PORT]>
          Enables dumping of SIP messages to and from host.

       sip set debug peer <peername>
          Enables dumping of SIP messages to and from peer's IP.
          Requires peer to be registered.

\end{verbatim}


\section{sip set debug off}
\subsection{Summary}
\begin{verbatim}
Disable SIP debugging
\end{verbatim}
\subsection{Usage}
\begin{verbatim}
Usage: sip set debug off
       Disables dumping of SIP packets for debugging purposes

\end{verbatim}


\section{sip set debug on}
\subsection{Summary}
\begin{verbatim}
Enable SIP debugging
\end{verbatim}
\subsection{Usage}
\begin{verbatim}
Usage: sip set debug {on|off|ip <host[:PORT]>|peer <peername>}
       sip set debug on
          Enables dumping of all SIP messages for debugging purposes

       sip set debug off
          Disables dumping of all SIP messages

       sip set debug ip <host[:PORT]>
          Enables dumping of SIP messages to and from host.

       sip set debug peer <peername>
          Enables dumping of SIP messages to and from peer's IP.
          Requires peer to be registered.

\end{verbatim}


\section{sip set debug peer}
\subsection{Summary}
\begin{verbatim}
Enable SIP debugging on Peername
\end{verbatim}
\subsection{Usage}
\begin{verbatim}
Usage: sip set debug {on|off|ip <host[:PORT]>|peer <peername>}
       sip set debug on
          Enables dumping of all SIP messages for debugging purposes

       sip set debug off
          Disables dumping of all SIP messages

       sip set debug ip <host[:PORT]>
          Enables dumping of SIP messages to and from host.

       sip set debug peer <peername>
          Enables dumping of SIP messages to and from peer's IP.
          Requires peer to be registered.

\end{verbatim}


\section{sip show channels}
\subsection{Summary}
\begin{verbatim}
List active SIP channels
\end{verbatim}
\subsection{Usage}
\begin{verbatim}
Usage: sip show channels
       Lists all currently active SIP channels.

\end{verbatim}


\section{sip show channel}
\subsection{Summary}
\begin{verbatim}
Show detailed SIP channel info
\end{verbatim}
\subsection{Usage}
\begin{verbatim}
Usage: sip show channel <channel>
       Provides detailed status on a given SIP channel.

\end{verbatim}


\section{sip show domains}
\subsection{Summary}
\begin{verbatim}
List our local SIP domains.
\end{verbatim}
\subsection{Usage}
\begin{verbatim}
Usage: sip show domains
       Lists all configured SIP local domains.
       Asterisk only responds to SIP messages to local domains.

\end{verbatim}


\section{sip show history}
\subsection{Summary}
\begin{verbatim}
Show SIP dialog history
\end{verbatim}
\subsection{Usage}
\begin{verbatim}
Usage: sip show history <channel>
       Provides detailed dialog history on a given SIP channel.

\end{verbatim}


\section{sip show inuse}
\subsection{Summary}
\begin{verbatim}
List all inuse/limits
\end{verbatim}
\subsection{Usage}
\begin{verbatim}
Usage: sip show inuse [all]
       List all SIP users and peers usage counters and limits.
       Add option "all" to show all devices, not only those with a limit.

\end{verbatim}


\section{sip show objects}
\subsection{Summary}
\begin{verbatim}
List all SIP object allocations
\end{verbatim}
\subsection{Usage}
\begin{verbatim}
Usage: sip show objects
       Lists status of known SIP objects

\end{verbatim}


\section{sip show peers}
\subsection{Summary}
\begin{verbatim}
List defined SIP peers
\end{verbatim}
\subsection{Usage}
\begin{verbatim}
Usage: sip show peers [like <pattern>]
       Lists all known SIP peers.
       Optional regular expression pattern is used to filter the peer list.

\end{verbatim}


\section{sip show peer}
\subsection{Summary}
\begin{verbatim}
Show details on specific SIP peer
\end{verbatim}
\subsection{Usage}
\begin{verbatim}
Usage: sip show peer <name> [load]
       Shows all details on one SIP peer and the current status.
       Option "load" forces lookup of peer in realtime storage.

\end{verbatim}


\section{sip show registry}
\subsection{Summary}
\begin{verbatim}
List SIP registration status
\end{verbatim}
\subsection{Usage}
\begin{verbatim}
Usage: sip show registry
       Lists all registration requests and status.

\end{verbatim}


\section{sip show settings}
\subsection{Summary}
\begin{verbatim}
Show SIP global settings
\end{verbatim}
\subsection{Usage}
\begin{verbatim}
Usage: sip show settings
       Provides detailed list of the configuration of the SIP channel.

\end{verbatim}


\section{sip show subscriptions}
\subsection{Summary}
\begin{verbatim}
List active SIP subscriptions
\end{verbatim}
\subsection{Usage}
\begin{verbatim}
Usage: sip show subscriptions
       Lists active SIP subscriptions for extension states

\end{verbatim}


\section{sip show users}
\subsection{Summary}
\begin{verbatim}
List defined SIP users
\end{verbatim}
\subsection{Usage}
\begin{verbatim}
Usage: sip show users [like <pattern>]
       Lists all known SIP users.
       Optional regular expression pattern is used to filter the user list.

\end{verbatim}


\section{sip show user}
\subsection{Summary}
\begin{verbatim}
Show details on specific SIP user
\end{verbatim}
\subsection{Usage}
\begin{verbatim}
Usage: sip show user <name> [load]
       Shows all details on one SIP user and the current status.
       Option "load" forces lookup of peer in realtime storage.

\end{verbatim}


\section{sip unregister}
\subsection{Summary}
\begin{verbatim}
Unregister (force expiration) a SIP peer from the registery

\end{verbatim}
\subsection{Usage}
\begin{verbatim}
Usage: sip unregister <peer>
       Unregister (force expiration) a SIP peer from the registry

\end{verbatim}


\section{skinny reset}
\subsection{Summary}
\begin{verbatim}
Reset Skinny device(s)
\end{verbatim}
\subsection{Usage}
\begin{verbatim}
Usage: skinny reset <DeviceId|all> [restart]
       Causes a Skinny device to reset itself, optionally with a full restart

\end{verbatim}


\section{skinny set debug}
\subsection{Summary}
\begin{verbatim}
Enable Skinny debugging
\end{verbatim}
\subsection{Usage}
\begin{verbatim}
Usage: skinny set debug
       Enables dumping of Skinny packets for debugging purposes

\end{verbatim}


\section{skinny set debug off}
\subsection{Summary}
\begin{verbatim}
Disable Skinny debugging
\end{verbatim}
\subsection{Usage}
\begin{verbatim}
Usage: skinny set debug off
       Disables dumping of Skinny packets for debugging purposes

\end{verbatim}


\section{skinny show devices}
\subsection{Summary}
\begin{verbatim}
List defined Skinny devices
\end{verbatim}
\subsection{Usage}
\begin{verbatim}
Usage: skinny show devices
       Lists all devices known to the Skinny subsystem.

\end{verbatim}


\section{skinny show lines}
\subsection{Summary}
\begin{verbatim}
List defined Skinny lines per device
\end{verbatim}
\subsection{Usage}
\begin{verbatim}
Usage: skinny show lines
       Lists all lines known to the Skinny subsystem.

\end{verbatim}


\section{sla show stations}
\subsection{Summary}
\begin{verbatim}
Show SLA Stations
\end{verbatim}
\subsection{Usage}
\begin{verbatim}
Usage: sla show stations
       This will list all stations defined in sla.conf

\end{verbatim}


\section{sla show trunks}
\subsection{Summary}
\begin{verbatim}
Show SLA Trunks
\end{verbatim}
\subsection{Usage}
\begin{verbatim}
Usage: sla show trunks
       This will list all trunks defined in sla.conf

\end{verbatim}


\section{soft hangup}
\subsection{Summary}
\begin{verbatim}
Request a hangup on a given channel
\end{verbatim}
\subsection{Usage}
\begin{verbatim}
Usage: soft hangup <channel>
       Request that a channel be hung up. The hangup takes effect
       the next time the driver reads or writes from the channel

\end{verbatim}


\section{stop gracefully}
\subsection{Summary}
\begin{verbatim}
Gracefully shut down Asterisk
\end{verbatim}
\subsection{Usage}
\begin{verbatim}
Usage: stop gracefully
       Causes Asterisk to not accept new calls, and exit when all
       active calls have terminated normally.

\end{verbatim}


\section{stop now}
\subsection{Summary}
\begin{verbatim}
Shut down Asterisk immediately
\end{verbatim}
\subsection{Usage}
\begin{verbatim}
Usage: stop now
       Shuts down a running Asterisk immediately, hanging up all active calls .

\end{verbatim}


\section{stop when convenient}
\subsection{Summary}
\begin{verbatim}
Shut down Asterisk at empty call volume
\end{verbatim}
\subsection{Usage}
\begin{verbatim}
Usage: stop when convenient
       Causes Asterisk to perform a shutdown when all active calls have ended.

\end{verbatim}


\section{stun debug}
\subsection{Summary}
\begin{verbatim}
Enable STUN debugging
\end{verbatim}
\subsection{Usage}
\begin{verbatim}
Usage: stun debug
       Enable STUN (Simple Traversal of UDP through NATs) debugging

\end{verbatim}


\section{stun debug off}
\subsection{Summary}
\begin{verbatim}
Disable STUN debugging
\end{verbatim}
\subsection{Usage}
\begin{verbatim}
Usage: stun debug off
       Disable STUN debugging

\end{verbatim}


\section{udptl debug}
\subsection{Summary}
\begin{verbatim}
Enable UDPTL debugging
\end{verbatim}
\subsection{Usage}
\begin{verbatim}
Usage: udptl debug [ip host[:port]]
       Enable dumping of all UDPTL packets to and from host.

\end{verbatim}


\section{udptl debug ip}
\subsection{Summary}
\begin{verbatim}
Enable UDPTL debugging on IP
\end{verbatim}
\subsection{Usage}
\begin{verbatim}
Usage: udptl debug [ip host[:port]]
       Enable dumping of all UDPTL packets to and from host.

\end{verbatim}


\section{udptl debug off}
\subsection{Summary}
\begin{verbatim}
Disable UDPTL debugging
\end{verbatim}
\subsection{Usage}
\begin{verbatim}
Usage: udptl debug off
       Disable all UDPTL debugging

\end{verbatim}


\section{ulimit}
\subsection{Summary}
\begin{verbatim}
Set or show process resource limits
\end{verbatim}
\subsection{Usage}
\begin{verbatim}
Usage: ulimit {-d|-l|-f|-m|-s|-t|-u|-v|-c|-n} [<num>]
       Shows or sets the corresponding resource limit.
         -d  Process data segment [readonly]
         -l  Memory lock size [readonly]
         -f  File size
         -m  Process resident memory [readonly]
         -s  Process stack size [readonly]
         -t  CPU usage [readonly]
         -u  Child processes
         -v  Process virtual memory [readonly]
         -c  Core dump file size
         -n  Number of file descriptors

\end{verbatim}


\section{voicemail show users}
\subsection{Summary}
\begin{verbatim}
List defined voicemail boxes
\end{verbatim}
\subsection{Usage}
\begin{verbatim}
Usage: voicemail show users [for <context>]
       Lists all mailboxes currently set up

\end{verbatim}


\section{voicemail show zones}
\subsection{Summary}
\begin{verbatim}
List zone message formats
\end{verbatim}
\subsection{Usage}
\begin{verbatim}
Usage: voicemail show zones
       Lists zone message formats

\end{verbatim}


\section{zap destroy channel}
\subsection{Summary}
\begin{verbatim}
Destroy a channel
\end{verbatim}
\subsection{Usage}
\begin{verbatim}
Usage: zap destroy channel <chan num>
	DON'T USE THIS UNLESS YOU KNOW WHAT YOU ARE DOING.  Immediately removes a given channel, whether it is in use or not

\end{verbatim}


\section{zap restart}
\subsection{Summary}
\begin{verbatim}
Fully restart zaptel channels
\end{verbatim}
\subsection{Usage}
\begin{verbatim}
Usage: zap restart
	Restarts the zaptel channels: destroys them all and then
	re-reads them from zapata.conf.
	Note that this will STOP any running CALL on zaptel channels.

\end{verbatim}


\section{zap show cadences}
\subsection{Summary}
\begin{verbatim}
List cadences
\end{verbatim}
\subsection{Usage}
\begin{verbatim}
Usage: zap show cadences
       Shows all cadences currently defined

\end{verbatim}


\section{zap show channels}
\subsection{Summary}
\begin{verbatim}
Show active zapata channels
\end{verbatim}
\subsection{Usage}
\begin{verbatim}
Usage: zap show channels [ trunkgroup <trunkgroup> | group <group> | context <context> ]
	Shows a list of available channels with optional filtering
	<group> must be a number between 0 and 63

\end{verbatim}


\section{zap show channel}
\subsection{Summary}
\begin{verbatim}
Show information on a channel
\end{verbatim}
\subsection{Usage}
\begin{verbatim}
Usage: zap show channel <chan num>
	Detailed information about a given channel

\end{verbatim}


\section{zap show status}
\subsection{Summary}
\begin{verbatim}
Show all Zaptel cards status
\end{verbatim}
\subsection{Usage}
\begin{verbatim}
Usage: zap show status
       Shows a list of Zaptel cards with status

\end{verbatim}


\section{zap show version}
\subsection{Summary}
\begin{verbatim}
Show the Zaptel version in use
\end{verbatim}
\subsection{Usage}
\begin{verbatim}
Usage: zap show version
       Shows the Zaptel version in use

\end{verbatim}


