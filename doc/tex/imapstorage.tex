By enabling IMAP Storage,  Asterisk will use native IMAP as the storage
mechanism for voicemail messages instead of using the standard file structure.

Tighter integration of Asterisk voicemail and IMAP email services allows
additional voicemail functionality, including:

\begin{itemize}
 \item Listening to a voicemail on the phone will set its state to "read" in
   a user's mailbox automatically.
 \item Deleting a voicemail on the phone will delete it from the user's
   mailbox automatically.
 \item Accessing a voicemail recording email message will turn off the message
   waiting indicator (MWI) on the user's phone.
 \item Deleting a voicemail recording email will also turn off the message
   waiting indicator, and delete the message from the voicemail system.
\end{itemize}

\subsection{Installation Notes}

\subsubsection{University of Washington IMAP C-Client}

If you do not have the University of Washington's IMAP c-client
installed on your system, you will need to download the c-client
source distribution (\url{http://www.washington.edu/imap/}) and compile it.
Asterisk supports both the 2004 and 2006 versions of c-client, however
mail\_expunge\_full is enabled in the 2006 version.

Note that Asterisk only uses the 'client' portion of the UW IMAP toolkit,
but building it also builds an IMAP server and various other utilities.
Because of this, the build instructions for the IMAP toolkit are somewhat
complicated and can lead to confusion about what is needed.

If you are going to be connecting Asterisk to an existing IMAP server,
then you don't need to care about the server or utilities in the IMAP
toolkit at all. If you want to also install the UW IMAPD server, that
is outside the scope of this document.

Building the c-client library is fairly straightforward; for example, on a
Debian system there are two possibilities:

\begin{enumerate}
    \item If you will not be using SSL to connect to the IMAP server:
   \begin{verbatim}
        $ make slx SSLTYPE=none!
   \end{verbatim}
    \item If you will be using SSL to connect to the IMAP server:
   \begin{verbatim}
   $ make slx EXTRACFLAGS="-I/usr/include/openssl"
   \end{verbatim}
\end{enumerate}

Once this completes you can proceed with the Asterisk build; there is no
need to run 'make install'.

\subsubsection{Compiling Asterisk}

To use the system c-client library, configure Asterisk with
./configure --with-imap=system. If you downloaded the c-client source
and compiled it according to the above instructions, configure
Asterisk with with ./configure --with-imap=/usr/src/imap or where ever
you built the UWashington IMAP Toolkit. When you run 'make
menuselect', choose 'Voicemail Build Options' and the IMAP\_STORAGE
option should be available for selection.

After selecting the IMAP\_STORAGE option, use the 'x' key to exit
menuselect and save your changes, and the build/install Asterisk
normally.

\subsection{Modify voicemail.conf}

The following directives have been added to voicemail.conf:
\begin{astlisting}
\begin{verbatim}
imapserver=<name or IP address of IMAP mail server>
imapport=<IMAP port, defaults to 143>
imapflags=<IMAP flags, "novalidate-cert" for example>
expungeonhangup=<yes or no>
authuser=<username>
authpassword=<password>
\end{verbatim}
\end{astlisting}

The "expungeonhangup" flag is used to determine if the voicemail system should
expunge all messages marked for deletion when the user hangs up the phone.

Each mailbox definition should also have imapuser=$<$imap username$>$.
For example:
\begin{astlisting}
\begin{verbatim}
4123=>4123,James Rothenberger,jar@onebiztone.com,,attach=yes|imapuser=jar
\end{verbatim}
\end{astlisting}

The directives "authuser" and "authpassword" are not needed when using
Kerberos. They are defined to allow Asterisk to authenticate as a single
user that has access to all mailboxes as an alternative to Kerberos.


\subsection{IMAP Folders}

Besides INBOX, users should create "Old", "Work", "Family" and "Friends"
IMAP folders at the same level of hierarchy as the INBOX.  These will be
used as alternate folders for storing voicemail messages to mimic the
behavior of the current (file-based) voicemail system.


\subsection{Separate vs. Shared Email Accounts}

As administrator you will have to decide if you want to send the voicemail
messages to a separate IMAP account or use each user's existing IMAP mailbox
for voicemail storage.  The IMAP storage mechanism will work either way.

By implementing a single IMAP mailbox, the user will see voicemail messages
appear in the same INBOX as other messages.  The disadvantage of this method
is that if the IMAP server does NOT support UIDPLUS, Asterisk voicemail will
expunge ALL messages marked for deletion when the user exits the voicemail
system, not just the VOICEMAIL messages marked for deletion.

By implementing separate IMAP mailboxes for voicemail and email, voicemail
expunges will not remove regular email flagged for deletion.


\subsection{IMAP Server Implementations}

There are various IMAP server implementations, each supports a potentially
different set of features.


\subsubsection{UW IMAP-2005 or earlier}

UIDPLUS is currently NOT supported on these versions of UW-IMAP.  Please note
that without UID\_EXPUNGE, Asterisk voicemail will expunge ALL messages marked
for deletion when a user exits the voicemail system (hangs up the phone).

\subsubsection{UW IMAP-2006 Development Branch}

This version supports UIDPLUS, which allows UID\_EXPUNGE capabilities.  This
feature allow the system to expunge ONLY pertinent messages, instead of the
default behavior, which is to expunge ALL messages marked for deletion when
EXPUNGE is called.  The IMAP storage mechanism is this version of Asterisk
will check if the UID\_EXPUNGE feature is supported by the server, and use it
if possible.

\subsubsection{Cyrus IMAP}

Cyrus IMAP server v2.3.3 has been tested using a hierarchy delimiter of '/'.


\subsection{Quota Support}

If the IMAP server supports quotas, Asterisk will check the quota when
accessing voicemail.  Currently only a warning is given to the user that
their quota is exceeded.


\subsection{Application Notes}

Since the primary storage mechanism is IMAP, all message information that
was previously stored in an associated text file, AND the recording itself,
is now stored in a single email message.  This means that the .gsm recording
will ALWAYS be attached to the message (along with the user's preference of
recording format if different - ie. .WAV).  The voicemail message information
is stored in the email message headers.  These headers include:

\begin{verbatim}
X-Asterisk-VM-Message-Num
X-Asterisk-VM-Server-Name
X-Asterisk-VM-Context
X-Asterisk-VM-Extension
X-Asterisk-VM-Priority
X-Asterisk-VM-Caller-channel
X-Asterisk-VM-Caller-ID-Num
X-Asterisk-VM-Caller-ID-Name
X-Asterisk-VM-Duration
X-Asterisk-VM-Category
X-Asterisk-VM-Orig-date
X-Asterisk-VM-Orig-time
\end{verbatim}
