\subsection{Introduction}

This section is intended as an introduction to the Inter-Asterisk 
eXchange v2 (or simply IAX2) protocol.  It provides both a theoretical 
background and practical information on its use.

\subsection{Why IAX2?}

The first question most people are thinking at this point is "Why do you 
need another VoIP protocol?  Why didn't you just use SIP or H.323?"

Well, the answer is a fairly complicated one, but in a nutshell it's like
this...  Asterisk is intended as a very flexible and powerful
communications tool.  As such, the primary feature we need from a VoIP
protocol is the ability to meet our own goals with Asterisk, and one with
enough flexibility that we could use it as a kind of laboratory for
inventing and implementing new concepts in the field.  Neither H.323 or
SIP fit the roles we needed, so we developed our own protocol, which,
while not standards based, provides a number of advantages over both SIP
and H.323, some of which are:

\begin{itemize}
	\item Interoperability with NAT/PAT/Masquerade firewalls
	\begin{itemize}
	     \item IAX seamlessly interoperates through all sorts of NAT and PAT
             and other firewalls, including the ability to place and 
             receive calls, and transfer calls to other stations.
	\end{itemize}
	\item High performance, low overhead protocol
	\begin{itemize}
	     \item When running on low-bandwidth connections, or when running 
	     large numbers of calls, optimized bandwidth utilization is 
	     imperative.  IAX uses only 4 bytes of overhead
	\end{itemize}
	\item Internationalization support
	\begin{itemize}
	     \item IAX transmits language information, so that remote PBX 
	     content can be delivered in the native language of the
	     calling party.
	\end{itemize}
	\item Remote dialplan polling
	\begin{itemize}
	     \item IAX allows a PBX or IP phone to poll the availability of a 
	     number from a remote server.  This allows PBX dialplans to 
	     be centralized.
	\end{itemize}
	\item Flexible authentication
	\begin{itemize}
	     \item IAX supports cleartext, md5, and RSA authentication, 
	     providing flexible security models for outgoing calls and 
	     registration services.
	\end{itemize}	
	\item Multimedia protocol
	\begin{itemize}
	     \item IAX supports the transmission of voice, video, images, text, 
	     HTML, DTMF, and URL's.  Voice menus can be presented in both
	     audibly and visually.
	\end{itemize}
	\item Call statistic gathering
	\begin{itemize}
	     \item IAX gathers statistics about network performance (including 
	     latency and jitter, as well as providing end-to-end latency
	     measurement.
	\end{itemize}
	\item Call parameter communication
	\begin{itemize}
	     \item Caller*ID, requested extension, requested context, etc are
	     all communicated through the call.
	\end{itemize}
	\item Single socket design
	\begin{itemize}
	     \item IAX's single socket design allows up to 32768 calls to be 
	     multiplexed.
	\end{itemize}
\end{itemize}

While we value the importance of standards based (i.e. SIP) call handling, 
hopefully this will provide a reasonable explanation of why we developed 
IAX rather than starting with SIP.

\subsection{Configuration}

For examples of a configuration, please see the iax.conf.sample in
your the /configs directory of you source code distribution.
